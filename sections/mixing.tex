An application of stochastic localizations is used to prove mixing bounds for Markov processes. 
Expanding on the work of Anari, Liu and Oveis in \cite{Anari_2020}, Chen and Eldan in \cite{Chen_2022} 
established a framework in which mixing bounds of a special class of Markov processes arises, namely Markov 
chains associated with stochastic localizations. We will in this section present this 
framework and describe its application to the Glauber dynamics.\todo{Fix!}

\subsection{Mixing bounds}

The motivation for Markov mixing bounds fundamentally comes from sampling. Suppose we wish to sample 
from some probability distribution $\mu$. A common method to achieve this to through the use of the 
Markov chain Monte Carlo (MCMC):
\begin{theorem}\label{thm:markov_conv}
  Given \((X_n)\) an irreducible positively recurrent homogenous Markov process on \(\mathcal{X}\) with
  stationary distribution \(\mu\), for any \(\phi : \mathcal{X} \to \mathbb{R}\) integrable, 
  \[\lim_{n \to \infty} \frac{1}{n} \sum_{k = 1}^n \phi(X_k) = \int f \dd \mu\]
  almost everywhere.  
\end{theorem}
With this theorem in mind, MCMC allows us to sample \(\mu\) by sampling from \((X_n)\) instead. It is in general 
not difficult to come up with such Markov processes, although it is often difficult to show its rate of convergence. 
This motivates the notion of mixing bounds which quantifies the time for which the Markov process takes before 
its law is approximately stationary.

\begin{definition}[Total variation mixing time]
  Given a probability measure \(\nu \in \mathcal{M}(\mathcal{X})\), a Markov kernel \(K\) with stationary 
  distribution \(\mu\) and some \(\epsilon > 0\), the \(\epsilon\)-total variation mixing time is defined as 
  \[t_{\text{mix}}(P, \epsilon, \mu) := \inf \{t \ge 0 \mid \|P^t\nu - \mu\|_{\text{TV}} < \epsilon\},\]
  Furthermore, we denote 
  \[t_{\text{mix}}(P, \epsilon) = \sup_{x \in \mathcal{X}} t_{\text{mix}}(P, \epsilon, \delta_x)\]
  the worst mixing time starting at a point.
\end{definition}

A standard method of analyzing the mixing times of Markov chains is through the use of the spectral gap. 

\begin{definition}[Spectral gap, \cite{Levin_2017}]
  The spectral gap of a Markov kernel \(K\) is defined to be 
  \[\mathfrak{gap}(K) := 1 - \sup\{\lambda \mid \lambda \text{ is an eigenvalue of } K, \lambda \neq 1\}.\]
\end{definition}

It is not difficult to show that 
\begin{equation}\label{eq:gap}
  \mathfrak{gap}(K) = 
    \inf_{\substack{\phi : \mathcal{X} \to \mathbb{R}\\ \int \phi \dd \mu = 0}} 1 
      - \frac{\int \phi K \phi \dd \mu}{\int \phi^2 \dd \mu}
    = \inf_{\phi : \mathcal{X} \to \mathbb{R}} 
      \frac{1}{2 \text{Var}_\mu[\phi]} \int (\phi(x) - \phi(y))^2 K(x, \dd y) \mu(\dd x),
\end{equation}
where \(\mu\) is the stationary measure of \(K\) and \(K\phi = \int \phi(y) K(\cdot, \dd y)\). We will take 
equation~\eqref{eq:gap} to be the defining property of the spectral gap in the case \(K\) is defined on 
a general state space.

\begin{theorem}[\cite{Levin_2017}]\label{thm:levin}
  Given a reversible and irreducible Markov chain with kernel \(K\) on the state space \(\mathcal{X}\) 
  with stationary distribution \(\mu\), denoting \(\mu_{\min} = \inf_{x \in \mathcal{X}} \mu(x)\), we have 
  \[t_{\text{mix}}(K, \epsilon) \le \left\lceil\frac{1}{\mathfrak{gap}(K)}
    \left(\frac{1}{2}\log\left(\frac{1}{\mu_{\min}}\right) + \log\left(\frac{1}{2\epsilon}\right)\right)\right\rceil.\]
\end{theorem}

We remark that this inequality is only meaningful whenever \(\mu_{\min} > 0\) and thus, this theorem is 
only meaningful for Markov chains on finite state space (and we can replace \(\inf\) with \(\min\)).

A similar bound can also be established using the modified log-Sobolev inequality (MLSI).

\begin{definition}[Entropy, \cite{Liu_2020}]
  Given \(\phi : \mathcal{X} \to \mathbb{R}_{\ge 0}\) and a measure \(\mu\), we define the entropy of \(\phi\) 
  with respect to \(\mu\) to be 
  \[\text{Ent}_\mu[\phi] := \mathbb{E}_\mu\left[\phi \log\left(\frac{1}{\mathbb{E}_\mu[\phi]} \phi\right)\right]
   = \int \phi \log \phi \dd \mu - \int \phi \dd \mu \log\left(\int \phi \dd \mu\right).\]
\end{definition}

\begin{definition}\todo{TODO}
  
\end{definition}

\subsubsection{Ising model and Glauber dynamics}

\begin{definition}[Ising model]
  Given a graph \(G = (V, E)\), \(\beta > 0\) and \(h \in \mathbb{R}\), the Ising model on \(G\) 
  with inverse temperature \(\beta\) and external field \(h\) is the probability measure \(\mu_{\beta, h}\) on 
  \(\{-1, 1\}^V\) defined such that for all \(\sigma \in \{-1, 1\}^V\),
  \[\mu_{\beta, h}(\sigma) := 
    \frac{1}{Z} \exp\left\{\beta \sum_{xy \in E} \sigma_x \sigma_y + h \sum_{x \in V} \sigma_x\right\}\]
  where \(Z > 0\) is the normalizing constant.
\end{definition}

Heuristically, the Ising model measures the probability that a graph is in a specific configuration of 
up and down spins in which neighboring vertices are more likely to have the same spin. This ``likeliness'' 
is controlled by \(\beta\) in which a larger \(\beta\) means that neighboring vertices are more likely to
align. 

As illustrated by theorem~\ref{thm:markov_conv}, in order to sample from the Ising model, we can 
construct a Markov chain which has the Ising model as its stationary distribution. One such Markov chain 
is known as the Glauber dynamics.

\begin{definition}[Glauber dynamics]
  Given a measure \(\mu\) of \(\{-1, 1\}^n\), the Glauber dynamics of \(\mu\) is the Markov chain 
  with kernel 
  \[K(\sigma^1, \sigma^2) := 
    \mathbf{1}_{\{\|\sigma^1 - \sigma^2\|_1 = 1\}}\frac{1}{n}\frac{\mu(\sigma^2)}{\mu(\sigma^1) + \mu(\sigma^2)}
  + \mathbf{1}_{\{\sigma^1 = \sigma^2\}} \frac{1}{n}\sum_{\|\tilde \sigma - \sigma^1\|_1 = 1}
    \frac{\mu(\sigma^1)}{\mu(\sigma^1) + \mu(\tilde \sigma)}.\]  
\end{definition}

Parsing this definition, we see that the Glauber dynamics is the Markov chain such that, starting at a 
configuration \(\sigma_1 \in \{-1, 1\}^n\), the configuration at the next time step either remains the 
same or change at one vertex. Furthermore, the probability of this occurring is weighted according to 
\(\mu\). As we hoped, the Glauber dynamics of the Ising model \(\mu_{\beta, h}\) has \(\mu_{\beta, h}\) 
as its stationary distribution.\todo{Prove this.}

\subsection{Dynamics of stochastic localizations}

As alluded to previously, one may associate a Markov process at each time step of a stochastic 
localization process for which the original process is stationary. We will in this section define 
these Markov processes and show that the Glauber dynamics can be constructed using this method. 

\begin{definition}[Markov process associated with a stochastic localization, \cite{Chen_2022}]
  Let \((\mu_t)_{t \ge 0}\) be a stochastic localization of \(\mu\) such that \(\mu_t\) is absolutely 
  continuous (almost everywhere) with respect to \(\mu\) for all \(t\). For all \(\tau > 0\), we define the dynamics 
  associated with \((\mu_t)_t\) at \(\tau\) to be the Markov process with kernel 
  \[K(x, A) := \mathbb{E}_\mathbb{P}\left[\frac{\dd \mu_\tau}{\dd \mu}(x) \mu_\tau(A)\right]\]
  for all \(x \in \mathcal{X}, A \in \Sigma\).
\end{definition}

As alluded by the notation, rather than a deterministic time \(\tau\), \(\tau\) can be taken to be an 
appropriate stopping time. In this case, the theorems below will remain to hold by invoking the optional stopping 
theorem whenever necessary.

This is indeed a kernel since for each \(x \in \mathcal{X}\), as \(\mathbb{E}_\mathbb{P}[\mu_\tau] = \mu\) 
by \ref{L2},
\[K(x, \Omega) = \mathbb{E}_\mathbb{P}\left[\frac{\dd \mu_\tau}{\dd \mu}(x)\right] 
  = \frac{\dd}{\dd \mu}\mathbb{E}_\mathbb{P}[\mu_\tau](x) = \frac{\dd \mu}{\dd \mu}(x) = 1\]
where the third equality follows by the uniqueness of the Radon-Nikodym derivative as for all \(A \in \Sigma\), 
we have
\[\int_A \mathbb{E}_\mathbb{P}\left[\frac{\dd \mu_\tau}{\dd \mu}\right] \dd \mu = 
  \mathbb{E}_\mathbb{P}\left[\int_A \frac{\dd \mu_\tau}{\dd \mu} \dd \mu \right] = 
  \mathbb{E}_\mathbb{P}[\mu_\tau](A).\]

\begin{proposition}
  The Markov process associated with a stochastic localization \((\mu_t)_{t \ge 0}\) of \(\mu\) is reversible and 
  has stationary distribution \(\mu\). 
\end{proposition}
\begin{proof}
  Taking \(\phi : \mathcal{X}^2 \to \mathbb{R}\) integrable, we have by Fubini's theorem
  \begin{equation}\label{eq:markov_assoc}
    \begin{split}
      \int_{\mathcal{X}^2} \phi(x, y) K(x, \dd y) \mu(\dd x) & = 
      \int \phi(x, y) \mathbb{E}_\mathbb{P}\left[\frac{\dd \mu_\tau}{\dd \mu}(x) \mu_\tau(\dd y)\right]\mu(\dd x)\\
      & = \mathbb{E}_\mathbb{P}\left[\int \phi(x, y) \mu_\tau(\dd y) \frac{\dd \mu_\tau}{\dd \mu}(x) \mu(\dd x)\right]\\
      & = \mathbb{E}_\mathbb{P}\left[\int \phi(x, y) \mu_\tau(\dd y) \mu_\tau(\dd x)\right].
    \end{split}
  \end{equation}
  Similarly, by the same calculation,   
  \[\int_{\mathcal{X}^2} \phi(x, y) K(y, \dd x) \mu(\dd y) = \mathbb{E}_\mathbb{P}\left[\int \phi(x, y) \mu_\tau(\dd y) \mu_\tau(\dd x)\right].\]
  Thus, 
  \[\int_{\mathcal{X}^2} \phi(x, y) K(x, \dd y) \mu(\dd x) = \int_{\mathcal{X}^2} \phi(x, y) K(y, \dd x) \mu(\dd y)\]
  for any integrable \(\phi : \mathcal{X}^2 \to \mathbb{R}\) implying \(K\) is reversible.

  On the other hand, for all \(A \in \Sigma\), we compute using the martingale property
  \begin{align*} 
    K\mu(A) & = \int K(x, A) \mu(\dd x) = \int \mathbb{E}_\mathbb{P}\left[\frac{\dd \mu_\tau}{\dd \mu}(x) \mu_\tau(A)\right] 
    = \mathbb{E}_\mathbb{P}\left[\mu_\tau(A) \int \frac{\dd \mu_\tau}{\dd \mu} \dd \mu\right] \\
    & = \mathbb{E}_\mathbb{P}[\mu_\tau(A) \mu_\tau(\Omega)] = \mathbb{E}_\mathbb{P}[\mu_\tau(A)] = \mu(A)
  \end{align*}
  implying \(\mu\) is the stationary measure of \(K\).
\end{proof}

\begin{proposition}\label{prop:gap_assoc}
  Taking \(K\) to be the kernel of the Markov process associated with a stochastic localization \((\mu_t)_{t \ge 0}\)
  of \(\mu\) at time \(\tau\), we have 
  \[\mathfrak{gap}(K) = \inf_{\phi : \mathcal{X} \to \mathbb{R}} \frac{\mathbb{E}[\text{Var}_{\mu_\tau}[\phi]]}{\text{Var}_\mu[\phi]}.\]
\end{proposition}
\begin{proof}
  By equation~\eqref{eq:markov_assoc} (where we take \(\phi(x, y) = \phi(x)\phi(y)\)), we have 
  \begin{equation}
    \int_{\mathcal{X}^2} \phi(x)\phi(y) K(x, \dd y) \mu(\dd x) = \mathbb{E}_\mathbb{P}\left[\left(\int_\mathcal{X} \phi \dd\mu_\tau\right)^2\right],
  \end{equation}
  for any integrable \(\phi : \mathcal{X} \to \mathbb{R}\). On the other hand, we observe 
  \[\int_{\mathcal{X}^2} \phi(y)^2 K(x, \dd y) \mu(\dd x) = \int (K \phi^2)(x) \mu(\dd x) = 
      \int \phi(x)^2 (K\mu)(\dd x) = \int \phi(x)^2 \mu(\dd x)\]
  as \(\mu\) is the stationary measure of \(K\). Thus, for any integrable \(\phi : \mathcal{X} \to \mathbb{R}\),
  by substituting the above two equations, we have
  \begin{align*}
    \hspace{1cm} & \frac{1}{2\text{Var}_\mu[\phi]} \int_{\mathcal{X}^2}(\phi(x) - \phi(y))^2 K(x, \dd y) \mu(\dd x) \\
    = & \frac{1}{2\text{Var}_\mu[\phi]} \left(\int\phi(x)^2\mu(\dd x) 
      - 2\int \phi(x)\phi(y) K(x, \dd y) \mu(\dd x) + \int \phi(y)^2 K(x, \dd y) \mu(\dd x)\right) \\
    = & \frac{1}{\text{Var}_\mu[\phi]}\left(\int\phi^2 \dd \mu 
      - \mathbb{E}_\mathbb{P}\left[\left(\int_\mathcal{X} \phi \dd\mu_\tau\right)^2\right]\right) \\
    = & \frac{1}{\text{Var}_\mu[\phi]} 
        \mathbb{E}_\mathbb{P}\left[\int\phi^2 \dd \mu_\tau - \left(\int_\mathcal{X} \phi \dd\mu_\tau\right)^2\right]
      = \frac{\mathbb{E}[\text{Var}_{\mu_\tau}[\phi]]}{\text{Var}_\mu[\phi]}.
  \end{align*}
  Hence, recalling the equivalent form of the spectral gap as described by \eqref{eq:gap}, the result follows 
  by taking infimum on both sides.
\end{proof}

This proposition has a nice intuitive interpretation. By recalling that the limit of a stochastic 
localization as \(t \to \infty\) is a Dirac measure, we may imagine a stochastic localization zooms in 
(in \(t\)) towards a region containing the massive point. Then, the spectral gap of the 
associated Markov process at time \(\tau\) is simply the smallest proportion of the local variation around this 
zoomed in region at time \(\tau\) to that of the global variation. This is achieved by a test function 
with minimal variation within this region and maximal variation outside of it.

With regards to theorem~\ref{thm:levin}, in the case \(\mu\) has full support on the finite state space 
\(\mathcal{X}\) (e.g. in the setting of Glauber dynamics) the above proposition provides a method for 
computing an upper bound for the mixing time. In particular, should the stochastic localization \((\mu_k)_k\)
satisfy 
\[\mathbb{E}[\text{Var}_{\mu_{k + 1}}[\phi] \mid \mu_t] \ge (1 - \epsilon)\text{Var}_{\mu_k}[\phi]\]
for given \(\epsilon > 0\) and any integrable function \(\phi : \mathcal{X} \to \mathbb{R}\), we have the 
telescoping product
\[\frac{\mathbb{E}[\text{Var}_{\mu_k}[\phi]]}{\text{Var}_\mu[\phi]} 
  = \mathbb{E}\left[\prod_{i = 1}^k\frac{\mathbb{E}[\text{Var}_{\mu_i}[\phi] \mid \mu_{i - 1}]}{\text{Var}_{\mu_{i - 1}}[\phi]}\right]
  \ge (1 - \epsilon)^k.\]
Hence, we have the bound 
\[\mathfrak{gap}(K)^{-1} \le (1 - \epsilon)^{-k}\]
which immediately provides an upper bound for the mixing time of \(K\) in light of theorem~\ref{thm:levin}.
This motivates the following definition.

\begin{definition}[Approximate conservation of variance, \cite{Chen_2022}]
  A stochastic localization process \((\mu_k)_k\) is said satisfy conserve \((\kappa_k)\)-variance 
  up to time \(n\) if for any integrable function \(\phi : \mathcal{X} \to \mathbb{R}\), \(1 \le k \le n\),
  \[\mathbb{E}[\text{Var}_{\mu_{k + 1}}[\phi] \mid \mu_t] \ge (1 - \kappa_k)\text{Var}_{\mu_k}[\phi].\]
\end{definition}

By the same computation above, if \((\mu_k)_k\) conserves \((\kappa_k)\)-variance up to time \(n\), 
then it associated dynamics has a spectral gap of at least \(\prod_{i = 1}^n (1 - \kappa_i)\).

\subsection{Glauber dynamics as an associated Markov process}
