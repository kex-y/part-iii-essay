\documentclass[]{article}
\usepackage{amsmath}
\usepackage[T1]{fontenc}
\usepackage[bitstream-charter]{mathdesign}
\usepackage[dvipsnames]{xcolor}
\usepackage{hyperref}
\hypersetup{
  pdftitle={WIP Title},
  pdfauthor={Kexing Ying},
  colorlinks=true,
  linkcolor=Maroon,
  filecolor=Maroon,
  citecolor=Maroon,
  urlcolor=Maroon,
  pdfcreator={LaTeX via pandoc}}

\usepackage[margin = 1.5in]{geometry}
\usepackage{graphicx}
\makeatletter
\def\maxwidth{\ifdim\Gin@nat@width>\linewidth\linewidth\else\Gin@nat@width\fi}
\def\maxheight{\ifdim\Gin@nat@height>\textheight\textheight\else\Gin@nat@height\fi}
\makeatother
% Scale images if necessary, so that they will not overflow the page
% margins by default, and it is still possible to overwrite the defaults
% using explicit options in \includegraphics[width, height, ...]{}
\setkeys{Gin}{width=\maxwidth,height=\maxheight,keepaspectratio}

\setlength{\emergencystretch}{3em} % prevent overfull lines
\setcounter{secnumdepth}{5}

\setlength\parindent{0pt} % no indentation for paragraphs
\setlength{\parskip}{3pt} % add a little space between paragraphs

\usepackage{tikz}
\usepackage{physics}
\usepackage{amsthm}
\usepackage{mathtools}
\usepackage{enumitem}
\usepackage{todonotes} % remove for final version

\theoremstyle{definition}
\newtheorem{theorem}{Theorem}
\newtheorem{conjecture}{Conjecture}
\newtheorem{definition}{Definition}[section]
\newtheorem{lemma}{Lemma}[section]
\newtheorem{proposition}[lemma]{Proposition}
\newtheorem{corollary}[lemma]{Corollary}
\newtheorem{example}{Example}[section]
\newtheorem*{remark}{Remark}

\title{WIP Title}
\author{Kexing Ying}

\begin{document}
\maketitle
\thispagestyle{empty}

\newpage
\tableofcontents
\thispagestyle{empty}

\newpage
\section{Introduction}
\label{sec:introduction}
The notion of stochastic localization was first introduced in the 2013 paper by Eldan \cite{Eldan_2013}
in order to make progress regarding an isoperimetric problem known as the 
\textit{Kannan-Lovász-Simonovitz} (KLS) conjecture. As it turns out, stochastic localization has been also 
useful in many other adjacent areas, in particular, in sampling and Markov mixing. This essay will 
provide an introduction to stochastic localization and describe its applications in Markov mixing and 
the KLS conjecture. Furthermore, using stochastic localization, this essay will also provide an alternative 
proof of a known bound for the log-Sobolev constant of log-concave measures. 

Stochastic localization in its most general form describes a sequence of random measures (random variables 
taking values in the space of measures) which begins at a given measure and converges to dirac 
measures almost everywhere, namely it ``localizes'', and moreover, satisfy a certain martingale 
condition. These sequences of random measures are useful in studying specific measures. Namely, 
by evolving the stochastic localization in time, particular properties of the structure collapses 
allowing us say something about them. On the other hand, the martingale property allows us to preserve 
these properties (possibly up to some time). Hence, by balancing the two, i.e. allowing the sequence 
to evolve so it is close to a dirac measure while not evolve too long such that we lose the properties 
of the original measure, we are able to obtain results about the original measure.

In our case, we will mostly focus on one specific type of stochastic localization known as the 
linear-tilt localization. The linear-tilt localization is a special case of stochastic localizations 
in which at each time step, the measure is ``tilted'' in a random direction. This random direction 
can be chosen in a variety of ways however one choice of interest is when the random direction is 
chosen according to a Wiener process. A stochastic localization constructed this way accumulates a 
Gaussian component which becomes more and more significant as the process evolves. This is particularly 
helpful as Gaussian measures are well understood and we can use properties of the Gaussian to obtain 
results about our original measure.
\subsection{Structure of this essay}

This essay consists of an introduction to the theory of stochastic localization and subsequently presents three 
of it applications. We will now give a brief overview of these applications.

\subsubsection{Markov mixing bounds}

% We then move on to discuss the application of stochastic localization to bounding Markov 
% mixing time. Motivated by the Markov chain Monte Carlo, the Markov mixing time provides a quantitative 
% measure on how fast a Markov chain converges to its stationary distribution. 

The first application of stochastic localization we will discuss is its application to bounding Markov
mixing times. The motivation for Markov mixing bounds fundamentally comes from sampling. Suppose we wish to sample 
from some probability distribution $\mu$. A common method to achieve this to through the use of the 
Markov chain Monte Carlo (MCMC):
\begin{theorem}[MCMC]\label{thm:markov_conv}
  Given \((X_n)\) an irreducible positively recurrent homogenous Markov process on \(\mathcal{X}\) with
  stationary distribution \(\mu\), for any \(\phi : \mathcal{X} \to \mathbb{R}\) integrable, 
  \[\lim_{n \to \infty} \frac{1}{n} \sum_{k = 1}^n \phi(X_k) = \int \phi \dd \mu\]
  almost everywhere.  
\end{theorem}
With this theorem in mind, MCMC allows us to sample \(\mu\) by sampling from a Markov chain instead. It is in general 
not difficult to come up with such Markov processes, although the difficulty often arises when we want to know its rate of convergence. 
This motivates the notion of mixing bounds which quantifies the time for which the Markov process takes before 
its law is approximately stationary.

\begin{definition}[Total variation mixing time]
  Given a probability measure \(\nu \in \mathcal{M}(\mathcal{X})\), a Markov kernel \(K\) with stationary 
  distribution \(\mu\) and some \(\epsilon > 0\), the \(\epsilon\)-total variation mixing time is defined as 
  \[t_{\text{mix}}(P, \epsilon, \nu) := \inf \{t \ge 0 \mid \|K^t\nu - \mu\|_{\text{TV}} < \epsilon\},\]
  Furthermore, we denote 
  \[t_{\text{mix}}(P, \epsilon) = \sup_{x \in \mathcal{X}} t_{\text{mix}}(P, \epsilon, \delta_x)\]
  the worst mixing time starting at a point.
\end{definition}

A standard method of analyzing the mixing times of Markov chains is through the use of the spectral gap. 

\begin{definition}[Spectral gap]
  Given a Markov kernel \(K\), we define its spectral gap to be 
  \[\mathfrak{gap}(K) := 1 - \sup\{\lambda \mid \lambda \text{ is an eigenvalue of } K, \lambda \neq 1\}.\]
\end{definition}

\begin{theorem}[\cite{Levin_2017}]\label{thm:levin}
  Given a reversible and irreducible Markov chain with kernel \(K\) on the state space \(\mathcal{X}\) 
  with stationary distribution \(\mu\), denoting \(\mu_{\min} = \inf_{x \in \mathcal{X}} \mu(x)\), we have 
  \[t_{\text{mix}}(K, \epsilon) \le \left\lceil\frac{1}{\mathfrak{gap}(K)}
  \left(\frac{1}{2}\log\left(\frac{1}{\mu_{\min}}\right) + \log\left(\frac{1}{2\epsilon}\right)\right)\right\rceil.\]
\end{theorem}

We will see that the spectral gap of Markov chains which kernels can be described using stochastic localizations 
is related to how the variance of the stochastic localization evolves. Thus, by analyzing the variance of 
the stochastic localization, in particular when the stochastic localization ``conserves'' variance, we can 
bound the spectral gap and consequently the mixing time of said kernel. As an example, we will apply this 
method to the Glauber dynamics on the Boolean hypercube to obtain a mixing time bound to the Ising model. 

\subsubsection{The KLS conjecture}

We then move on to discuss a proof of Eldan's original 2013 result in \cite{Eldan_2013} which reduced 
the KLS conjecture to another seemingly weaker conjecture known as the thin-shell conjecture (up to an logarithmic factor). 
We quickly introduce these conjectures here.

\begin{definition}[log-concave measure]
  A measure \(\mu\) on \(\mathbb{R}^n\) is said to log-concave if it is of the form 
  \[\dd \mu = e^{-V}\dd\text{Leb}^n\] 
  for some convex function \(V : \mathbb{R}^n \to \mathbb{R} \cup \{\infty\}\).
\end{definition}

Straightaway, we observe that the standard Gaussian measure on \(\mathbb{R}^n\) is log-concave.
Thus, as the Gaussian measures are very well understood, we are motivated to ask how similarly do log-concave 
measures behave when compared to the Gaussian. The KLS conjecture is one such comparison which compares the 
concentration of log-concave measures with that of the Gaussian.

\begin{definition}[Concentration, \cite{Eldan_notes}]
  Let \(\mu\) be a measure on \(\mathbb{R}^n\), then \(\mu\) is said to be \(C\)-concentrated if
  for all \(1\)-Lipschitz function \(\phi : \mathbb{R}^n \to \mathbb{R}\), 
  \begin{equation}
    \text{Var}_\mu[\phi] = \text{Var}_{X \sim \mu}[\phi(X)] \le \frac{1}{C^2}.
  \end{equation}
  We denote the largest possible such \(C\) by \(C^\mu_{\text{con}}\).
\end{definition}

As it turns out, the standard Gaussian measure on \(\mathbb{R}^n\) is concentrated by a constant which 
is independent of the dimension \(n\). Thus, we might make a conjecture of the form 
``all log-concave probability measures are concentrated by a universal constant''. 
However, as currently stated, this statement is obviously false as spreading out 
the measure decreases the concentration. Furthermore, we need to be careful since, unlike the Gaussian, the 
concentration of general log-concave measures are not invariant under linear transformations. Hence, it is clear that the KLS 
conjecture would not hold without a suitable normalization. This leads us to the following.

\begin{definition}[Isotropic]
  A measure \(\mu\) on \(\mathbb{R}^n\) is isotropic if 
  \(\mathbb{E}_{X \sim \nu}[X] = 0\) and \(\text{Cov}_{X \sim \nu}(X) = \text{id}_n\).
\end{definition}

% The KLS conjecture is one such comparisons which roughly asserts 
% that the class of isotropic log-concave measures are concentrated in a similar way to the Gaussian. 
% More precisely, we have the following definitions.
 
\begin{conjecture}[Kannan-Lovász-Simonovitz, \cite{Eldan_notes}]\label{conj:KLS}
  Denoting \(\mathscr{M}^n_{\text{iso}}\) the set of all isotropic log-concave probability measures \(\mu\) on 
  \(\mathbb{R}^n\) (recall that \(\mu\) is isotropic if \(\mathbb{E}_{X \sim \mu}[X] = 0\) and \(\text{Cov}_{X \sim \mu}(X) = \text{id}_n\)),
  there exists a \textit{universal} constant \(C\) 
  (i.e. does not depend on the dimension \(n\)) such that for all 
  \(\mu \in \mathscr{M}^n_{\text{con}}\), \(\mu\) is \(C\)-concentrated.
\end{conjecture}

From a more geometric point of view, the KLS conjecture can be equivalently phrased such that it asserts the 
maximum proportion of volume by surface area of a log-concave measure is bounded by a universal constant. 
As a result, the KLS conjecture has many important consequences in convex geometry. 
In particular, by noticing that any uniform measure on a convex body of unit volume is log-concave, 
the KLS conjecture directly implies the Bourgain slicing conjecture.

\begin{conjecture}[Bourgain slicing, \cite{Bourgain}]
  For any convex body \(U \subset \mathbb{R}^n\) of unit volume, there exists a hyperplane \(S\) such 
  that \(U \cap S\) has boundary measure of at least \(C\) for some universal constant \(C\). 
\end{conjecture}

Moreover, as we shall see, the KLS conjecture in addition has applications to the Poincaré inequality, 
first moment concentration, exponential concentration and the aforementioned thin-shell conjecture. 
We will take a special look at the thin-shell conjecture which relaxes the KLS conjecture by 
only requiring the variance of the norm function to behave similar to the Gaussian.

\begin{conjecture}[Thin-shell, \cite{Eldan_2013}]
  Taking \(\mathscr{M}^n_{\text{iso}}\) as above, there exists a universal constant \(C\) such that 
  for all \(\mu \in \mathscr{M}^n_{\text{con}}\), we have
  \[\sqrt{\text{Var}_\mu[\|\cdot\|]} \le \frac{1}{C}.\]
\end{conjecture}

% Since the norm function is 1-Lipschitz, it is clear that the thin-shell conjecture is 
% weaker than that of the KLS conjecture. On the other hand, as we shall describe in the following subsections, 
% as a consequence of the theory of stochastic localization, it is possible to reduce the KLS conjecture to 
% the thin-shell conjecture up to a logarithmic factor. 

As a consequence of stochastic localization, we present a proof of the following theorem.

\begin{theorem}[Eldan, \cite{Eldan_2013}]\label{thm:KLS_to_TS}
  Denoting \(\mathscr{M}^n_{\text{iso}}\) as above, we define 
  \[C^n_{\text{con}} := \sup \left\{C \mid \text{\(\forall \mu \in \mathscr{M}^n_{\text{con}},\) 
    \(\mu\) is \(C\)-concentrated}\right\},\]
  and 
  \[C^n_{\text{TS}} := \sup \left\{C \mid \text{\(\forall \mu \in \mathscr{M}^n_{\text{con}},\) 
      \(\sqrt{\text{Var}_\mu[\|\cdot\|]} \le C^{-1}\)}\right\},\]
  we have,
  \[C^n_{\text{con}} \le C^n_{\text{TS}} \le C^n_{\text{con}}\log n.\]
\end{theorem}

% The thin-shell conjecture on the other hand relaxes the condition required for concentration and only 
% asserts that the variance of the norm to behave similarly to the Gaussian.
% \begin{conjecture}[Thin-shell, \cite{Eldan_2013}]
%   Taking \(\mathscr{M}^n_{\text{iso}}\) as above, there exists a universal constant \(C\) such that 
%   for all \(\mu \in \mathscr{M}^n_{\text{con}}\), we have
%   \[\sqrt{\text{Var}_\mu[\|\cdot\|]} \le \frac{1}{C}.\]
% \end{conjecture}

\subsubsection{Log-Sobolev inequality}

We will in the last section of this essay discuss the application of stochastic localization to the 
log-Sobolev inequality. Similar to the KLS conjecture, we are interested to compare the log-concave 
measures to the Gaussian measure. In this case, motivated by the Gaussian log-Sobolev inequality, 
we establish a similar inequality for log-concave measures bounding the entropy of said measures. 
Heuristically, similar to that of the variance, the entropy of a random variable is a measure of its 
uncertainty or randomness. 

\begin{definition}[Entropy]
  Given \(\phi : \mathcal{X} \to \mathbb{R}_{\ge 0}\) and a measure \(\mu\), we define the entropy of \(\phi\) 
  with respect to \(\mu\) to be 
  \[\text{Ent}_\mu[\phi] := \mathbb{E}_\mu\left[\phi \log\left(\frac{1}{\mathbb{E}_\mu[\phi]} \phi\right)\right]
   = \int \phi \log \phi \dd \mu - \int \phi \dd \mu \log\left(\int \phi \dd \mu\right)\]
  with the convention that \(0\log 0 = 0\).
\end{definition}

The log-Sobolev inequality is then formulated as the following.
\begin{definition}[Log-Sobolev inequality, \cite{Lee_2016}]
  For a given measure \(\mu\) on \(\mathbb{R}^n\), \(\mu\) is said to satisfy the log-Sobolev inequality with log-Sobolev 
  constant \(\rho_\mu\) if \(\rho_\mu\) is the largest \(\rho\) such that for all smooth 
  \(\phi : \mathbb{R}^n \to \mathbb{R}\) with \(\int \phi^2 \dd \mu = 1\), we have 
  \[\frac{\rho}{2} \text{Ent}_\mu[\phi^2] \le \mathbb{E}_\mu[\|\nabla \phi\|^2] = \int \|\nabla \phi\|^2 \dd \mu.\]
\end{definition}

The log-Sobolev inequality is an incredibly useful inequality while studying the concentration of measures. In 
particular, should a measure satisfy a log-Sobolev equality, by using a well-known method commonly known as Herbst's argument,
one can obtain the exponential concentration of said measure via the Chernoff bound. 
\begin{theorem}[Herbst's argument]\label{thm:herbst}
  If \(\mu\) satisfy the log-Sobolev inequality with log-Sobolev constant \(\rho_\mu\), then for all \(\phi\) with 
  uniformly bounded gradient \(\|\nabla \phi\| \le K\), we have 
  \[\psi_{\phi - \mathbb{E}_\mu[\phi]}^\mu(\lambda) \le \frac{K^2 \lambda^2}{2\rho_\mu}\]
  where \(\psi_{\phi - \mathbb{E}_\mu \phi}^\mu\) is the logarithmic moment generating function of 
  \(\phi - \mathbb{E}_\mu[\phi]\) with respect to \(\mu\).
\end{theorem}
A proof of this theorem is included in Appendix~\ref{sec:herbst}.
% https://faculty.math.illinois.edu/~psdey/math595fa19/lec14.pdf
  
It does not come as a surprise that as with the setting of the KLS conjecture, the standard Gaussian measure 
\(\gamma^n\) in \(\mathbb{R}^n\) satisfies the log-Sobolev inequality with log-Sobolev constant 
\(\rho_{\gamma^n} = 1\) (c.f. \cite{Gross_1975}). However, in contrast to the KLS conjecture which suggests that the concentration of 
log-concave measures are bounded below by a universal constant, it is known (c.f. \cite{Lee_2016}) that the log-Sobolev constant 
cannot be bounded below by a universal constant. Nonetheless, we are 
interested in the log-Sobolev constant of log-concave measure. More specifically, we will study the 
log-Sobolev constant for log-concave measures supported in a ball of fixed diameter to obtain the 
following result.

\begin{theorem}[\cite{Lee_2016}]\label{thm:Lee_Vempala}
  For any isotropic log-concave measure \(\mu\) on \(\mathbb{R}^n\) with support on a ball of diameter \(D\), 
  \(\mu\) has log-Sobolev constant \(\rho_\mu \gtrsim D^{-1}\).
\end{theorem}




% Moreover, the KLS conjecture has applications in bounding mixing times for 
% ball walks from a warm start (c.f. \cite{Lee_2016}), the central limit theorem for convex bodies (c.f. \cite{Giannopoulos}) 
% and directly implies the aforementioned thin-shell conjecture. 



% \begin{itemize}
%   \item The first section of this essay consists of setting up the theory of stochastic localization.
%     We also introduce some specific examples of stochastic localizations which will be used in subsequent 
%     sections most importantly the discrete and continuous linear-tilt localization. 
%   \item We then move on to discuss the application of stochastic localization to bounding Markov 
%     mixing time. Motivated by the Markov chain Monte Carlo, the Markov mixing time provides a quantitative 
%     measure on how fast a Markov chain converges to its stationary distribution. It turns out that 
%     the mixing time of Markov chains which kernels can be described using stochastic localizations 
%     is related to how the variance of the stochastic localization evolves. This method is then applied  
%     to the Glauber dynamics on the Boolean hypercube to obtain a mixing time bound to the Ising model. 
%   \item We also discuss a proof of Eldan's original 2013 result reducing the KLS conjecture to another seemingly 
%     weaker conjecture known as the thin-shell conjecture. Roughly speaking, the KLS conjecture asserts 
%     that the class of measures known as the \textit{isotropic log-concave} measures are concentrated in a similar 
%     way to the Gaussian. In a more geometric point of view, the KLS conjecture asserts that the 
%     maximum proportion of volume by surface area of a log-concave measure is bounded by a universal constant. 
%     As a result, the KLS conjecture has many important consequences in convex geometry. 
%     In particular, by noticing that any uniform measure on a convex body of unit volume is log-concave, 
%     the KLS conjecture also implies the Bourgain slicing conjecture which asserts that any convex body 
%     of unit volume in \(\mathbb{R}^n\) has a intersection with some hyperplane with volume bounded below 
%     by a universal constant. Moreover, the KLS conjecture has applications in bounding mixing times for 
%     ball walks from a warm start (c.f. \cite{Lee_2016}), the central limit theorem for convex bodies (c.f. \cite{Giannopoulos}) 
%     and directly implies the aforementioned thin-shell conjecture. 
  % \item Finally, inspired by the proof from the third section, we will use stochastic localization methods to 
  %   provide a bound for the log-Sobolev constant of log-concave measures. The log-Sobolev constant is 
  %   a constant relating to the log-Sobolev inequality -- a concept central to the theory of concentrations. 
  %   In a similar vain to the KLS conjecture, motivated by the Gaussian log-Sobolev inequality, we also
  %   attempt to show that the log-concave measures behaves similarly to the Gaussian in the setting of 
  %   the log-Sobolev inequality. While in general, this turns out not to be the case, by restricting 
  %   our view to log-concave measures with bounded support, we are able to obtain a bound for the 
  %   log-Sobolev constant. 
    
% \end{itemize}



\newpage
\section{Stochastic Localization Scheme}
\label{sec:stoch_loc}
We will in this section introduce the notion of stochastic localization schemes. 
To gain an intuition for these objects, we will present several examples which 
will be studied further in subsequent sections. 

We will work in general Borel spaces \((\mathcal{X}, \Sigma)\) for this section while for restrict our 
focus to either the Euclidean space \(\mathbb{R}^n\) or the Boolean hypercube \(\{-1, 1\}^n\) 
in subsequent sections. We take \((\Omega, \mathscr{F}, \mathbb{P})\) our underlying probability space 
and we introduce the notation \(\mathcal{M}(\mathcal{X})\) for the space of probability measures on 
\(\mathcal{X}\).

\begin{definition}[Prelocalization process]
  Given \(\mu \in \mathcal{M}(\mathcal{X})\), a measure-valued stochastic process 
  \((\mu_t)_{t \ge 0}\) is said to be a prelocalization of \(\mu\) if 
  \begin{enumerate}[label=(L\arabic*), start=0]
    \item \label{L0} \(\mu_0 = \mu\).
    \item \label{L1} For all \(t \ge 0\), \(\mu_t\) is a probability measure almost everywhere, i.e. 
      \(\mathbb{P}(\mu_t(\mathcal{X}) = 1) = 1\).
    \item \label{L2} For all \(A \in \Sigma\), \((\mu_t(A))_{t \ge 0}\) is a martingale with respect 
      to the natural filtration of \((\mu_t)\).
  \end{enumerate}
  Where \(\mathcal{M}(\mathcal{X})\) is equipped with the \(\sigma\)-algebra generated by maps of the form 
  \[\pi_A : \mathcal{M}(\mathcal{X}) \mapsto \mathbb{R}_{\ge 0} \cup \{\infty\} : \mu \mapsto \mu(A)\] 
  for all \(A \in \Sigma\). Equivalently, this is the Borel \(\sigma\)-algebra on \(\mathcal{M}(\mathcal{X})\) 
  using the topology induced by the total variation norm.
\end{definition}

\begin{definition}[Stochastic localization process, \cite{Chen_2022}]
  Given \(\mu \in \mathcal{M}(\mathcal{X})\), a measure-valued stochastic process 
  \((\mu_t)_{t \ge 0}\) is said to be a stochastic localization of \(\mu\) if in addition to being a 
  prelocalization of \(\mu\), \((\mu_t)_{t \ge 0}\) also satisfies
  \begin{enumerate}[label=(L\arabic*), start=3]
    \item \label{L3} For all \(A \in \Sigma\), \(\mu_t(A)\) converges almost everywhere to \(0\) 
      or \(1\) as \(t \to \infty\).
  \end{enumerate}
\end{definition}

\begin{definition}[Stochastic localization scheme, \cite{Chen_2022}]
  Denoting \(\mathcal{L}(\mu)\) the set of all stochastic localization processes of the measure
  \(\mu\), a stochastic localization scheme is a map 
  \[\Phi : \mathcal{M}(\mathcal{X}) \to \coprod_{\mu \in \mathcal{M}(\mathcal{X})} \mathcal{L}(\mu)\] 
  such that \(\Phi(\mu) \in \mathcal{L}(\mu)\) for all \(\mu \in \mathcal{M}(\mathcal{X})\).
\end{definition}

We say a stochastic localization is discrete if \(t\) takes value in \(\mathbb{N}\) and continuous 
if \(t\) takes value in \(\mathbb{R}_{\ge 0}\). For shorthand, we denote \((\mu_k)_k\) for a
discrete stochastic localization of \(\mu\).

\begin{proposition}\label{prop:stoch_loc}
  Straightaway, by the martingale property, if \((\mu_t)_{t \ge 0}\) is a stochastic localization of \(\mu\), 
  then 
  \begin{itemize}
    \item \(\mathbb{E}[\mu_t] = \mu\) for all \(t \ge 0\).
    \item taking \(X \sim \mu\) such that \(\mu_t \to \delta_X\) almost everywhere as \(t \to \infty\) 
      (here weak and total variational convergence are equivalent and so \(\to\) can mean either).
  \end{itemize}
\end{proposition}
\begin{proof}
  The first statement is immediate as for all \(A \in \Sigma\), 
  \[\mathbb{E}[\mu_t](A) \triangleq \mathbb{E}[\mu_t(A)] = \mathbb{E}[\mu_0(A)] = \mu(A).\]
  To prove the second statement, let us first parse what the claim is. Fixing a realization \(\omega\) of \(\mu_t\), 
  we have by \ref{L3} that \(\mu_t\) converges to some Dirac measure based at some \(x_\omega \in \mathcal{X}\). Thus, 
  defining the random variable \(X : \omega \mapsto x_\omega\), it suffices to show \(X \sim \mu\). 
  Indeed, by taking \(\phi : \mathcal{X} \to \mathbb{R}\) to be any bounded and continuous function, by the 
  definition of \(X\)
  \[\int \phi(x) \mu_t(\dd x) \xrightarrow{\text{a.e.}} \int \phi(x) \delta_{X}(\dd x) = \phi(X) \text{ as } t \to \infty.\]
  Thus, taking expectation on both sides, we have 
  \[\mathbb{E}[\phi(X)] = \mathbb{E}\left[\int \phi \dd \mu_t\right] = 
    \int \phi \dd \mathbb{E}[\mu_t] = \int \phi \dd \mu\]
  implying \(X \sim \mu\) as required.
\end{proof}

An example of a stochastic localization scheme is the coordinate by coordinate localization scheme 
on \(\mathcal{X} = \{-1, 1\}^n\). This scheme relates to the Glauber dynamics for which the stochastic 
localization scheme provides a mixing bound. We shall examine the property in section~\ref{sec:Glauber_loc}, 
though we will construct the scheme now. 

Given a probability measure \(\mu\) on \(\{-1, 1\}^n\), we introduce the random variable \(X \sim \mu\), and 
\(Y\) a uniform random variable over all permutations of \([n] = \{1, \cdots, n\}\) independent of \(X\). 
Then, the coordinate by coordinate stochastic localization of \(\mu\) is the process \((\mu_k)_{k}\)
such that for all \(x \in \{-1, 1\}^n\),
\[\mu_k(x) = \mathbb{P}(X = x \mid X_{Y_1}, \cdots, X_{Y_{n \wedge k}}).\]
Namely, \(\mu_k\) is the law of \(X\) conditioned on \(X_{Y_1}, \cdots, X_{Y_i}\).

\((\mu_k)_{k}\) is indeed a stochastic localization of \(\mu\). It is clear that \ref{L0} and \ref{L1} are 
satisfied. By construction of \((\mu_k)_k\), denoting 
\(\mathscr{F}_k := \sigma(X_{Y_1}, \cdots, X_{Y_{n \wedge k}})\), we have by the tower property
\[\mathbb{E}[\mu_{k + 1}(x) \mid \mathscr{F}_k] 
  = \mathbb{E}[\mathbb{E}[\mathbb{P}(X = x \mid X) \mid \mathscr{F}_{k + 1}] \mid \mathscr{F}_k]
  = \mathbb{E}[\mathbb{P}(X = x \mid X) \mid \mathscr{F}_k] = \mu_k(x)\]
implying \((\mu_k(x))\) a martingale as required for \ref{L2}. Finally, it is clear that
\[\lim_{k \to \infty} \mu_k(x) = \mu_n(x) = \mathbb{P}(X = x \mid X) = \mathbf{1}_{\{X = x\}} \in \{0, 1\}\]
implying \ref{L3}.

An analogous construction of the coordinate by coordinate stochastic localization scheme in \(\mathbb{R}^n\) 
is the random subspace localization. Similar to before, for a probability measure \(\mu\) on \(\mathbb{R}^n\),
we introduce the random variable \(X \sim \mu\) and \(Y\) a uniform random variable on \(O(n)\) 
(so the column vectors \(\{Y_1, \cdots, Y_n\}\) form an orthonormal basis of \(\mathbb{R}^n\)) 
independent of \(X\). Then, we define the random subspace stochastic localization of \(\mu\) as \((\mu_k)_k\) 
where \(\mu_k\) is the law of \(X\) conditioned on \(\langle X, Y_1\rangle, \cdots, \langle X, Y_{n \wedge k}\rangle\).

\subsection{Linear-tilt localization schemes}

An important class of stochastic localization schemes are the linear-tilt schemes. Introduced by Eldan 
in \cite{Eldan_2013}, linear-tilt schemes has been vital in the recent progress regarding the KLS 
conjecture. More recently, a discrete version of the linear-tilt scheme was introduced in \cite{Chen_2022}
and is used to provide a mixing bound for Glauber dynamics. We will in this section introduce these family 
of localizations and consider two specific examples of such linear-tilt schemes which are 
useful for our analysis later.

Informally, given a probability measure \(\mu\) on \(\mathcal{X} \subseteq \mathbb{R}^n\), the linear-tilt scheme
of \(\mu\) is constructed recursively in which at each step, we pick a random direction and multiply 
the density at this time with a linear function along this direction (i.e. a tilt along a random 
direction). 

Let \(\mu\) be a probability measure on \(\mathcal{X} \subseteq \mathbb{R}^n\), we introduce the following definition.
\begin{definition}[Barycenter]
  The barycenter of \(\mu\) with respect to the function \(F : \mathbb{R}^n \to \mathbb{R}\) is
  \[\bar{\mu}(F) := \int_{\mathcal{X}} x F(x) \mu(\dd x).\]
  In the case that \(F = \text{id}\), we simply write \(\bar{\mu} = \bar{\mu}(F) = \mathbb{E}_{X \sim \mu}[X]\).
\end{definition}

\begin{definition}[Linear-tilt localization]
  A measure-valued stochastic process \((\mu_t)_{t \ge 0}\) is said to be a linear-tilt localization of
  the probability measure \(\mu\) if 
  \begin{enumerate}
    \item \(\mu_t \ll \mu\) for each \(t \ge 0\), and 
    \item denoting \(F_t := \dd \mu_t / \dd \mu\), we have \(F_0 = 1\) and 
      \begin{equation}\label{eq:linear-tilt}
        \dd F_t(x) = \langle x - \bar{\mu}(F_t), \dd Z_t \rangle F_t(x)
      \end{equation}
      for some stochastic process \((Z_t)_{t \ge 0}\) such that \(\mathbb{E}[\dd Z_t \mid \mu_t] = 0\) 
      for all \(t \ge 0\). 
  \end{enumerate} 
\end{definition}

It is clear that \((\mu_t(A))_t\) is a martingale for all \(A \in \Sigma\) by observing that 
equation~\eqref{eq:linear-tilt} has no drift term. Hence, it follows \((\mu_t(\mathcal{X}))\) is a 
martingale. However, since \(\mu_t(\mathcal{X}) = \int F_t \dd \mu\) is differentiable by 
construction, \((\mu_t(\mathcal{X}))\) has zero quadratic variation and thus is constant in \(t\). 
With this in mind, as \(\mu_0 = \mu\) is a probability measure, it follows:

\begin{proposition}\todo{I think Eldan's proof in \cite{Eldan_2013} has a mistake}
  If \((\mu_t)_t\) is a linear-tilt localization of \(\mu\), then \(\mu_t\) is a probability measure for each \(t\).
\end{proposition}

\begin{corollary}
  A linear-tilt localization \((\mu_t)_t\) of \(\mu\) is a prelocalization of \(\mu\).
\end{corollary}

We remark that in general, a linear-tilt localization is not necessarily a stochastic localization as \ref{L3} 
might not be satisfied. It is possible to impose sufficient conditions on \((Z_t)\) for which \ref{L3} holds,
e.g. by requiring \(\|\text{Cov}(Z_t)\|_{\text{op}}\) to decrease sufficiently fast. However, for generality, 
we will not restrict ourselves to one of these conditions. Instead, we will consider \ref{L3} case by case 
in the following examples of linear-tilt schemes. 

\subsubsection{Linear-tilt localization driven by a Wiener process}\label{sec:construct}

A natural choice of \((Z_t)_{t \ge 0}\) is the standard Wiener process on \(\mathbb{R}^n\).
Denoting \((W_t)_{t \ge 0}\) a standard Wiener process on \(\mathbb{R}^n\), we define the random functions 
\((F_t)_{t \ge 0}\) to be the solution of the following infinite system of SDEs 
(existence and uniqueness is established by theorem 5.2 in \cite{Øksendal_2003}): 
\begin{equation}\label{eq:stoch_loc}
  F_0 = 1, \dd F_t(x) = \langle x - \bar{\mu}(F_t), \dd W_t \rangle F_t(x),
\end{equation}
for all \(x \in \mathbb{R}^n\). We shall from this point forward denote the random variables \(a_t := \bar{\mu}(F_t)\).

By applying Itô's formula, we make the following useful observation: for all \(x \in \mathbb{R}^n\),
\begin{equation}
  \dd \log F_t(x) = \frac{\dd F_t(x)}{F_t(x)} - \frac{\dd\hspace{0pt} [F(x)]_t}{2F_t(x)^2} 
    = \langle x - a_t, \dd W_t \rangle - \frac{1}{2}\|x - a_t\|^2 \dd t
\end{equation}
where the second equality follows by the construction of \(F\). Hence, as \(\log F_0(x) = 0\), we 
observe
\begin{align*}
  \log F_t(x) & = \int_0^t \langle x - a_s, \dd W_s \rangle - \frac{1}{2}\int_0^t \|x - a_s\|^2 \dd s\\
    & = \left(\langle x, W_t \rangle - \int_0^t \langle a_s, \dd W_s \rangle\right)
      - \left(\frac{t}{2}\|x\|^2 + \frac{1}{2}\int_0^t\|a_s\|^2 \dd s - \int_0^t \langle x, a_s \rangle \dd s\right)\\
    & = - \left(\int_0^t \langle a_s, \dd W_s \rangle + \frac{1}{2}\|a_s\|^2 \dd s\right) + 
      \langle x, a_t + W_t \rangle - \frac{t}{2}\|x\|^2.
\end{align*}
Thus, taking \(\dd z_t := \langle a_t, \dd W_t\rangle + \frac{1}{2} \|a_t\|^2 \dd t\) and 
\(v_t := a_t + W_t\), we observe \(F_t(x)\) is of the form
\begin{equation}\label{eq:stoch_loc_alt}
  F_t(x) = e^{z_t + \langle x, v_t \rangle - \frac{t}{2}\|x\|^2},
\end{equation}
for given Itô processes \((z_t), (v_t)\).

With this formulation of \(F_t(x)\) in mind, it follows \(F_t\) is non-negative, and so, 
we may define \((\mu_t)_t\) to be the process such that \(\dd \mu_t = F_t \dd \mu\). It is clear that 
\((\mu_t)_t\) is a linear-tilt localization of \(\mu\) and so, is a prelocalization of \(\mu\).
The remainder of this section is devoted to showing \((\mu_t)_t\) is furthermore a stochastic localization 
of \(\mu\) if \(\mu\) is log-concave (namely we will show \ref{L3} for this special case), and prove 
some basic properties about this process useful for our analysis later.

\begin{definition}[Log-concave measure]
  A measure \(\mu\) on \(\mathbb{R}^n\) is said to log-concave if it is of the form 
  \(\dd \mu = \exp(-H)\dd\text{Leb}^n\) for some convex function \(H : \mathbb{R}^n \to \mathbb{R} \cup \{\infty\}\)
\end{definition}

To show \((\mu_t)\) satisfies \ref{L3} if \(\mu\) is log-concave, we study the limiting behavior of 
\((\mu_t)\) as \(t \to \infty\) by considering their covariances:
\begin{equation}
  A_t := \text{Cov}[\mu_t] = \int (x - a_t) \otimes (x - a_t) \mu_t(\dd x),
\end{equation}
where \(\otimes\) denotes the Kronecker product. In particular, we will show \((A_t)_{ij} \to 0\) for all
 \(i, j \in \{1, \cdots, n\}\) as \(t \to \infty\) allowing us to conclude \((\mu_t)\) converges weakly 
 to some Dirac measure. Indeed, this is a direct consequence of the following lemma.

\begin{lemma}[Brascamp-Lieb inequality, \cite{Brascamp_1976}]\label{lem:brascamp-lieb}
  Given \(V : \mathbb{R}^n \to \mathbb{R}\) convex and \(K > 0\), if \(\nu\) is an isotropic probability 
  measure on \(\mathbb{R}^n\) of the form 
  \[\dd \nu = Ze^{-V(x) - \frac{1}{2K}\|x\|^2}\dd \text{Leb}^n\]
  with \(Z\) being the normalization constant, then \(\nu\) satisfy the Poincaré inequality, i.e. 
  for all differentiable \(\phi\),
  \[K\text{Var}_\nu[\phi] \le \int \|\nabla\phi\|^2 \dd\nu.\]
\end{lemma}

With this lemma in mind, by taking \(\nu = \mu_t\) using 
equation~\eqref{eq:stoch_loc_alt} and defining
\(\pi_i(x) := x_i\), we have by the Cauchy-Schwarz inequality
\[(A_t)_{ij} \le \sqrt{\text{Var}_{\mu_t}[\pi_i]}\sqrt{\text{Var}_{\mu_t}[\pi_j]} 
  \le \max_{k = 1, \cdots, n} \frac{1}{t}\int \|\nabla \pi_k\|^2 \dd \mu_t\]
Again, using equation~\eqref{eq:stoch_loc_alt}, we note that any realizations of \((F_t(x))\) is eventually 
decreasing in \(t\) for all \(x \neq 0\), implying 
\[\sup_{t > 0} \max_{k = 1, \cdots, n} \int \|\nabla \pi_k\|^2 \dd \mu_t = 
\sup_{t > 0} \max_{k = 1, \cdots, n} \int x_k^2 \dd \mu_t < \infty.\] 
Thus, by taking \(t \to \infty\) we have \((A_t)_{ij} \to 0\) for all \(i, j \in \{1, \cdots, n\}\) as claimed 
and we have \((\mu_t)\) satisfying \ref{L3}.

\begin{corollary}
  \((\mu_t)\) converges set-wise to some Dirac measure almost everywhere. We denote this 
  limiting (random) Dirac measure by \(\delta_{a_\infty}\) where \(a_\infty\) is some 
  \(\mathbb{R}^n\)-valued random variable.
\end{corollary}

As a result of \ref{prop:stoch_loc}, we have the following useful corollary.

\begin{corollary}\label{cor:lim_dis}
  The massive point \(a_\infty\) of the limiting Dirac measure is the limit of \(a_t\) as 
  \(t \to \infty\) and has law \(\mu\).
\end{corollary}
\begin{proof}
  Since convergence implies relatively compact, applying the Dunford-Pettis theorem it follows that 
  any realizations of \((F_t)\) is uniformly integrable. Thus, the result follows by the Vitali 
  convergence theorem.
\end{proof}

\begin{corollary}\label{cor:lim_mart}
  Similarly, taking \(\phi\) to be any continuous function (not necessarily bounded as we have uniform 
  integrability), defining \(M_t = \int \phi \dd\mu_t\), \((M_t)_t\) is a martingale and
  \begin{equation}\label{eq:lim_mart}
    M_t \xrightarrow{\text{a.e.}} M_\infty \sim \phi_* \mu
  \end{equation}
  where \(\phi_*\mu\) denotes the push-forward measure of \(\mu\) along \(\phi\).
\end{corollary}

\subsubsection{Discrete time linear-tilt localization}

We may construct an analogous version of the linear-tilt localization for discrete time. By utilizing 
the little-\(o\) notation, equation~\eqref{eq:linear-tilt} can be rewritten as 
\[\frac{\dd \mu_{t + h}}{\dd \mu}(x) = 
  \frac{\dd \mu_t}{\dd \mu}(x) + \langle x - \bar{\mu}_t, h\dd Z_t\rangle \frac{\dd \mu_t}{\dd \mu}(x) 
  + o(h).\]
Hence, an discrete analog of the linear tilt localization is defined as the following.

\begin{definition}[Discrete time linear-tilt localization]
  Given a measure \(\mu \in \mathcal{M}(\mathcal{X})\), the discrete time linear-tilt localization 
  is the sequence of random measures \((\mu_k)_k\) defined by \(\mu_0 = \mu\) and 
  \begin{equation}\label{eq:discrete_tilt}
    \dd \mu_{k + 1} = (1 + \langle x - \bar{\mu}_k, Z_k \rangle)\dd \mu_k
  \end{equation} 
  for some sequence of random variables such that \(\mathbb{E}[Z_k \mid \mu_k] = 0\) for all 
  \(k \in \mathbb{N}\).
\end{definition}

Using the discrete time linear-tilt localization, let us now provide an alternative construction of the 
coordinate by coordinate localization.

Given \(\mu\) a probability measure on \(\{-1, 1\}^n\), we recall that the coordinate by coordinate 
localization is defined by ``pinning'' an additional random coordinate after each time step. To phrase 
this as a linear-tilt localization, we take the random variables \(Z_k\) to be
\begin{equation}
  Z_k := e_{Y_k} \cdot
    \begin{cases}
      \frac{1}{1 + (\bar{\mu}_k)_{Y_k}} & \text{with probability } \frac{1 + (\bar{\mu}_k)_{Y_k}}{2} \\
      \frac{-1}{1 - (\bar{\mu}_k)_{Y_k}} & \text{with probability } \frac{1 - (\bar{\mu}_k)_{Y_k}}{2}
    \end{cases}
\end{equation}
where \(e_1, \cdots, e_n\) are the standard basis of \(\mathbb{R}^n\) and again \(Y\) is a uniform 
random variable over all permutations of \([n]\). Thus, the linear-tilt localization \((\mu_k)\) given by this 
choice of \((Z_k)\) is defined by
\[\mu_{k + 1}(\sigma) = (1 + \|Z_k\|(\sigma_{Y_k} - (\overline{\mu}_k)_{Y_k}))\mu_k(\sigma)\]
for all \(k < n\). Similar to before, we terminate the process at time \(n\) and so we extend the process 
to all times by taking \(\mu_k = \mu_{n \wedge k}\).

Let us parse this definition to see why this is equivalent to the coordinate by coordinate localization. 
Taking \(k < n\), we have at the \(k + 1\)-th step, \(Y_{k + 1}\) chooses a random axis
which had not been chosen before. Then, \(Z_k\) is chosen such that for each configuration \(\sigma\), 
the probability of \(\sigma_{Y_k}\) being \(\pm 1\) is proportional the mass of \(\mu_k\) on \(\pm e_{Y_k}\). 
This is precisely the steps needed to construct the coordinate by coordinate localization as conditioning 
on an additional axis in this case is simply a projection on to said axis. 

\newpage
\section{The KLS and Thin-Shell Conjecture}
\label{sec:KLS}
In this section, we will present the original context and motivation for the construction of stochastic 
localizations: the KLS conjecture. The KLS conjecture is a conjecture which roughly states that all 
log-concave measures on \(\mathbb{R}^n\) are concentrated in a way similar to a Gaussian measure. 
To tackle this problem, Eldan in \cite{Eldan_2013} introduced stochastic localizations and applied to 
in order to reduce the KLS conjecture to a seemingly weaker conjecture (up to a logarithmic constant): 
the thin-Shell conjecture. More recently, also using stochastic localization, Chen \cite{Chen_2020} 
was able to provide an almost constant lower bound on for the KLS conjecture. We will now describe 
a proof of Eldan's original reduction of the KLS conjecture to the thin-Shell conjecture.
The method presented in this section is due to Lee and Vempala \cite{Lee_2016} and reformulated in 
the language of concentration by Eldan \cite{Eldan_notes}.

\subsection{Concentration}

Let us quickly introduce some preliminary definitions required to state the aforementioned conjectures. 

\begin{definition}[Concentration, \cite{Eldan_notes}]
  Let \(\mu\) be a measure on \(\mathbb{R}^n\), then \(\mu\) is said to be \(C\)-(inversely)-concentrated if
  for all \(1\)-Lipschitz function \(\phi : \mathbb{R}^n \to \mathbb{R}\), 
  \begin{equation}
    \text{Var}_\mu[\phi] = \text{Var}_{X \sim \mu}[\phi(X)] \le C^2.
  \end{equation}
  We denote the least possible such \(C\) by \(C^\mu_{\text{con}}\).
\end{definition}

Heuristically, the concentration measures the relation between \(\mu\) and the Euclidean metric by 
providing a numerical control for the variance of its norm. This is perhaps best illustrated by the following proposition.

\begin{proposition}\label{prop:concentration}
  Let \(X\) be a \(\mathbb{R}^n\)-valued random variable. Then for all \(K\)-Lipschitz function 
  \(\phi : \mathbb{R}^n \to \mathbb{R}\),
  \[\text{Var}[\phi(X)] \le K^2 \text{Var}[\|X\|^2].\]
\end{proposition}
\begin{proof}
  WLOG. by subtracting its expectation from \(X\), we may assume \(\mathbb{E}[X] = 0\).
  Let \(X'\) be a i.i.d. copy of \(X\) on the same probability space. Then for all \(K\)-Lipschitz 
  function \(\phi\), we have 
  \begin{align*}
    2 \text{Var}[\phi(X)] & = \text{Var}[\phi(X) - \phi(X')] & (\text{i.i.d.})\\
      & = \mathbb{E}[(\phi(X) - \phi(X'))^2] - \mathbb{E}[\phi(X) - \phi(X')]^2 & \\
      & = \mathbb{E}[(\phi(X) - \phi(X'))^2] & (\text{identically distributed}) \\
      & \le K^2 \mathbb{E}[\|X - X'\|^2] & (\text{as \(\phi\) is \(K\)-Lipschitz}) \\
      & = K^2 \mathbb{E}[X^T X + X'^T X' - X^T X' - X'^T X] & \\
      & = 2K^2 \text{Var}[\|X\|^2] - 2K^2 \text{Cov}(X, X') = 2K^2 \text{Var}[\|X\|^2]. & (\text{independence})
  \end{align*}
  implying \(\text{Var}[\phi(X)] \le K^2 \text{Var}[\|X\|^2]\) as claimed.
\end{proof}

With this proposition in mind, it is clear that for \(\mathbb{R}\)-valued random variables \(X\), 
its law \(\mu\) has concentration \(C^\mu_{\text{con}} = \text{Var}[X]\). Furthermore, by 
considering the projection maps, it follows that the standard Gaussian measure on \(\mathbb{R}^n\) 
is 1-concentrated.

We note that the definition we are presenting here is slightly non-standard. However, utilising a 
remarkable result due to Milman, we show that this definition is equivalent to the following 
definitions in a specific sense.

\begin{definition}[Exponential concentration, \cite{Milman_2018}]
  Given a measure \(\mu\) on \(\mathbb{R}^n\), we say \(\mu\) has exponential concentration if 
  there exists some \(c, D > 0\) such that for all 1-Lipschitz function 
  \(\phi : \mathbb{R}^n \to \mathbb{R}, t > 0\), we have
  \begin{equation}
    \mu(|\phi - \mathbb{E}_\mu[\phi]| \ge t) \le c e^{-Dt}.
  \end{equation}
  Fixing \(c = 1\), we denote the largest possible \(D\) as \(D^\mu_{\text{exp}}\).
\end{definition}

\begin{definition}[First-moment concentration, \cite{Milman_2018}]
  Again, given \(\mu\) a measure on \(\mathbb{R}^n\), we say \(\mu\) has first-moment concentration 
  if there exists some \(D > 0\) such that for all 1-Lipschitz function 
  \(\phi : \mathbb{R}^n \to \mathbb{R}\), we have
  \begin{equation}
    \mathbb{E}_\mu[|\phi - \mathbb{E}_\mu[\phi]|] \le \frac{1}{D}.
  \end{equation}
  We denote the largest possible \(D\) by \(D^\mu_{\text{FM}}\).
\end{definition}

It is clear that exponential concentration implies first-moment concentration. Indeed, if \(\mu\) 
has exponential concentration with constant \(D\) (taking \(c = 1\)), then by the tail probability formula,
\begin{align*}
  \mathbb{E}_\mu[|\phi - \mathbb{E}_\mu[\phi]|] 
  & = \int_0^\infty \mu(|\phi - \mathbb{E}_\mu[\phi]| \ge t) \dd t \le \int_0^\infty e^{-Dt} \dd t 
   = \frac{1}{D}.
\end{align*}

On the other hand, Milman showed that for log-concave measures on \(\mathbb{R}^n\), exponential concentration 
and first-moment concentration are equivalent in the following sense. 

\begin{theorem}[Milman, \cite{Milman_2008}]\label{thm:milman}
  For all log-concave measure \(\mu\) on \(\mathbb{R}^n\), \(\mu\) has exponential concentration 
  if and only if \(\mu\) has first-moment concentration. Furthermore, 
  \(D^\mu_{\text{exp}} \simeq D^\mu_{\text{FM}}\) where we write \(A \simeq B\) if there exists 
  universal constants \(C_1, C_2 > 0\) such that \(C_1 A \le B \le C_2 A\).
\end{theorem}

With this theorem in mind, we establish the following correspondence.

\begin{proposition}\label{thm:equiv}
  For all measures \(\mu\) on \(\mathbb{R}^n\), we have 
  \[\text{Exponentially concentrated \(\implies\) Concentrated \(\implies\) First-moment concentrated}\]
  and \(D^\mu_{\text{exp}} \le \sqrt{2} (C^\mu_{\text{con}})^{-1}\) and 
  \((2C^\mu_{\text{con}})^{-1} \le D^\mu_{\text{FM}}\).
  Hence, if \(\mu\) is log-concave, 
  \(D^\mu_{\text{exp}} \simeq D^\mu_{\text{FM}} \simeq (C^\mu_{\text{con}})^{-1}\).
\end{proposition}
\begin{proof}
  Assume first that \(\mu\) is \(C\)-concentrated. Then by the Chebyshev inequality, we have 
  \[\mu(|\phi - \mathbb{E}_\mu[\phi]| \ge t) 
      \le \frac{1}{t^2}\text{Var}_\mu[\phi] 
      \le \frac{C^2}{t^2},\]
  for all 1-Lipschitz \(\phi\). 
  Thus, by tail probability,
  \begin{align*}\mathbb{E}_\mu[|\phi - \mathbb{E}_\mu[\phi]|] 
    & = \int_0^\infty \mu(|\phi - \mathbb{E}_\mu[\phi]| \ge t) \dd t\\ 
    & \le \inf_{a > 0} \left\{\int_0^a \mu(|\phi - \mathbb{E}_\mu[\phi]| \ge t) \dd t + C^2 \int_a^\infty \frac{1}{t^2} \dd t\right\}\\ 
    & \le \inf_{a > 0} \left\{a + \frac{C^2}{a}\right\} = 2C,
  \end{align*}
  implying \(\mu\) is first-moment concentrated with respect to the constant \((2C)^{-1}\).

  On the other hand, if \(\mu\) is exponential concentration with some constant \(D\), then 
  again by the tail probability, 
  \[\text{Var}_\mu[\phi] = \int_0^\infty \mu((\phi - \mathbb{E}_\mu[\phi])^2 \ge t) \dd t
      \le \int_0^\infty e^{-D \sqrt{t}} \dd t = \frac{2}{D^2}\]
  implying \(\mu\) is \(\sqrt{2} D^{-1}\)-concentrated.
\end{proof}

\subsection{Example: concentration of the Gaussian}

Given a class of measures, it is in general \textit{not} true that the concentration coefficient of 
measures of said class is dimension invariant. However, this turns out to be the case for the Gaussian 
measures.

\begin{theorem}[Concentration of Gaussian measures]\label{thm:gaussian_conc}
  Denoting \(\gamma^n\) the standard Gaussian measure on \(\mathbb{R}^n\), \(\gamma^n\) is \(C\)-concentrated 
  for some constant \(C\) which is independent of \(n\). That is, for all 1-Lipschitz 
  \(\phi : \mathbb{R}^n \to \mathbb{R}\), \(\text{Var}_{\gamma^n}[\phi] \le C^2\).
\end{theorem}

This fact motivate the KLS conjecture which hypothesized that 
this invariance holds for a larger class of measures known as the log-concave measures. We will 
for completeness give a brief proof of the above theorem taking \(C = \pi / 2\) 
(although one can further show Gaussian measures are 1-concentrated).

To prove this theorem we first observe the following elementary property of the Gaussian measure.

\begin{lemma}\label{lem:gaussian_inner}
  For all \(x \in \mathbb{R}^n\), we have 
  \(\mathbb{E}_{\gamma^n}[|\langle x, \cdot \rangle|^2] = \|x\|^2\).
\end{lemma}
\begin{proof}
  Defining \(f_x := \langle x, \cdot \rangle\), we have 
  \(\mathbb{E}_{\gamma^n}[|\langle x, \cdot \rangle|^2] = \mathbb{E}_{f_x^* \gamma^n}[|\cdot |^2]\) where 
  \(f_x^* \gamma^n \sim \mathcal{N}(0, \|x\|^2)\) as \(f_x^*\) is linear. Hence, the result follows as 
  \(\mathbb{E}_{f_x^* \gamma^n}[|\cdot |^2] = \text{Var}_{f_x^* \gamma^n}[\text{id}] = \|x\|^2\).
\end{proof}

With this in mind, fixing a 1-Lipschitz function \(\phi : \mathbb{R}^n \to \mathbb{R}\), we will now attempt 
to bound \(\int \int |\phi(x) - \phi(y)|^2 \gamma^n(\dd x) \gamma^n(\dd y)\) by first bounding it by 
the integral of a inner product. In particular, we observe
\begin{align*}
  |\phi(x) - \phi(y)| 
  & = \left|\int_0^{\pi / 2}\partial_\theta \phi(x \sin \theta + y \cos \theta) \dd \theta\right|
    \le \int_0^{\pi / 2}|\partial_\theta \phi(x \sin \theta + y \cos \theta)| \dd\theta\\
  & = \int_0^{\pi / 2} |\langle \nabla \phi(x \sin \theta + y \cos \theta), x \cos \theta - y \sin\theta\rangle| \dd \theta.
\end{align*}
Then, by rescaling \(\dd \theta\), we may apply Jensen's inequality resulting in
\begin{align*}
  |\phi(x) - \phi(y)|^2 
  & \le \left(\int_0^{\pi / 2} |\langle \nabla \phi(x \sin \theta + y \cos \theta), x \cos \theta - y \sin\theta\rangle| \dd \theta\right)^2\\
  & \le \frac{\pi}{2}\int_0^{\pi / 2} |\langle \nabla \phi(x \sin \theta + y \cos \theta), x \cos \theta - y \sin\theta\rangle|^2 \dd \theta
\end{align*}
Thus, we have 
\begin{align*}
  & \int \int |\phi(x) - \phi(y)|^2 \gamma^n(\dd x) \gamma^n(\dd y) \\
  & \le \frac{\pi}{2} \int_0^{\pi / 2} \dd \theta \int \int 
    |\langle \nabla \phi(x \sin \theta + y \cos \theta), x \cos \theta - y \sin\theta\rangle|^2
    \gamma^n(\dd x) \gamma^n(\dd y).
\end{align*}
Now, by substituting \(u = x \sin \theta + y\cos \theta, v = x \cos \theta - y \sin \theta\) (which Jacobian 
has determinant 1), we have
\begin{align*}
  \int \int |\phi(x) - \phi(y)|^2 \gamma^n(\dd x) \gamma^n(\dd y) 
  & \le \frac{\pi}{2} \int_0^{\pi / 2} \dd \theta \int \int |\langle \nabla \phi(u), v \rangle|^2 \gamma^n(\dd u) \gamma^n(\dd v)\\
  & \le \frac{\pi}{2}\int_0^{\pi / 2} \dd \theta \int \|\nabla \phi(u)\|^2 \gamma^n(\dd u)
\end{align*}
where the second inequality is due to lemma \ref{lem:gaussian_inner}. Hence, as \(\phi\) is 1-Lipschitz, 
\(\|\nabla \phi(u)\| \le 1\) for all \(u\) and thus, 
\[\int \int |\phi(x) - \phi(y)|^2 \gamma^n(\dd x) \gamma^n(\dd y) 
  \le \frac{\pi}{2}\int_0^{\pi / 2} \dd \theta \int \dd \gamma^n
  = \frac{\pi}{2} \cdot \frac{\pi}{2} = \frac{\pi^2}{4}.\]
The immediate consequence of this bound is that there exists some \(z \in \mathbb{R}^n\) such that 
\(\int |\phi(x) - z|^2 \gamma^n(\dd x) \le \pi^2 / 4\). Indeed, if no such \(z\) exists, 
then \(\int |\phi(x) - \phi(y)|^2 \gamma^n(\dd x) > \pi^2 / 4\) for all \(y\), implying
\(\int \int |\phi(x) - \phi(y)|^2 \gamma^n(\dd x) \gamma^n(\dd y) > \pi^2 / 4\) which 
contradicts the above bound. Hence, choosing such a \(z\), we conclude the proof of 
theorem~\ref{thm:gaussian_conc} since
\[\text{Var}_{\gamma^n}[\phi] = \min_{w \in \mathbb{R}^n} \int |\phi(x) - w|^2 \gamma^n(\dd x)
  \le \int |\phi(x) - z|^2 \gamma^n(\dd x) \le \left(\frac{\pi}{2}\right)^2\]
as required.

\subsection{The KLS and thin-shell conjecture}

As alluded to in the previous section, the KLS conjecture suggests that any log-concave measure on 
\(\mathbb{R}^n\) admits the same concentration as that of the Gaussian measure. However, unlike the Gaussian, as the 
concentration of measures is not invariant under linear functions, it is clear that the KLS 
conjecture would not hold without a suitable normalization. This leads us to the following formulation 
of the KLS conjecture.

\begin{conjecture}[Kannan-Lovász-Simonovitz, \cite{Eldan_notes}]\label{conj:KLS}
  Denoting \(\mathscr{M}^n_{\text{con}}\) the set of all log-concave probability measures \(\mu\) on 
  \(\mathbb{R}^n\) satisfying \(\text{Var}_\mu[T] \le 1\) for all 1-Lipschitz linear maps 
  \(T : \mathbb{R}^n \to \mathbb{R}\), there exists a \textit{universal} constant \(C\) such that for all 
  \(\mu \in \mathscr{M}^n_{\text{con}}\), \(\mu\) is \(C\)-concentrated.
\end{conjecture}

We remark that \(C\) is universal in the sense that it does not depend on any parameter and in 
particular is independent of the dimension \(n\).

\begin{conjecture}[Thin-shell, \cite{Eldan_2013}]
  Taking \(\mathscr{M}^n_{\text{con}}\) as above, there exists a universal constant \(C\) such that 
  for all \(\mu \in \mathscr{M}^n_{\text{con}}\), we have
  \[\sqrt{\text{Var}_\mu[\|\cdot\|]} \le C.\]
\end{conjecture}

Taking \(X \sim \mu\) in proposition~\ref{prop:concentration}, it is a priori that the thin-shell conjecture is 
weaker than that of the KLS conjecture. On the other hand, as we shall describe in the following subsections, 
as a consequence of the theory of stochastic localization, Eldan \cite{Eldan_2013} provides a reduction 
of the KLS conjecture to the thin-shell conjecture up to logarithmic factors. 

\begin{theorem}[Eldan, \cite{Eldan_2013}]\label{thm:KLS_to_TS}
  Denoting \(\mathscr{M}^n_{\text{con}}\) as above, we define 
  \[C^n_{\text{con}} := \inf \left\{C \mid \text{\(\forall \mu \in \mathscr{M}^n_{\text{con}},\) 
    \(\mu\) is \(C\)-concentrated}\right\},\]
  and 
  \[C^n_{\text{TS}} := \inf \left\{C \mid \text{\(\forall \mu \in \mathscr{M}^n_{\text{con}},\) 
      \(\sqrt{\text{Var}_\mu[\|\cdot\|]} \le C\)}\right\}
      = \sup_{\mu \in \mathscr{M}^n_{\text{con}}} \sqrt{\text{Var}_\mu[\|\cdot\|]},\]
  we have,
  \[C^n_{\text{TS}} \le C^n_{\text{con}} \lesssim C^n_{\text{TS}} \log n.\]
\end{theorem}

% We remark that while the constants in theorem~\ref{thm:KLS_to_TS} depends on the dimension \(n\), the 
% KLS conjecture is reduced to the thin-shell conjecture in the sense that it suffices to show 
% \(\sup_n C^n_{\text{TS}} \log n < \infty\). Then, the universal bound for the KLS conjecture is obtained 
% by taking the constant \(C = \sup_n C^n_{\text{con}} \le \sup_n C^n_{\text{TS}} \log n < \infty\).

The stochastic localization scheme has been wildly successful in making progress towards the KLS conjecture. 
Modifying the original arguments by Eldan, Lee and Vempala \cite{Lee_2016} obtained the bound 
\(C_{\text{con}}^n \lesssim n^{-1 / 4}\). Further modifying their arguments, a recent breakthrough by 
Chen \cite{Chen_2020} improves the bound providing the following theorem. 

\begin{theorem}[Chen, \cite{Chen_2020}]
  \(\log C_{\text{con}}^n \lesssim \sqrt{\log n \log \log n}\) and so \(C_{\text{con}}^n = n^{-o(1)}\).
\end{theorem}


\subsubsection{Equivalent formulation of the KLS conjecture}

While we have formulated the KLS conjecture using the language of concentration, the conjecture itself 
was originally formulated as an isoperimetric problem. For completeness of this exposition, we shall 
briefly present these equivalent formulations here.  

The isoperimetric problem is the problem in finding 
the set of unit volume with minimum surface area. In the case of the \(\mathbb{R}^n\) equipped with the 
Lebesgue measure, we have known since the ancient Greeks \cite{iso_hist} that the solution is the unit ball. 
With this in mind, it is natural for us to generalize the problem for arbitrary measures. 

\begin{definition}[Minkowski's boundary measure]
  Given a measure \(\mu\) on \(\mathbb{R}^n\) and a Borel set \(A \subseteq <\mathbb{R}^n\), the Minkowski's 
  boundary measure of \(A\), 
  \[\mu^+(\partial A) := \liminf_{\epsilon \downarrow 0} \frac{\mu(A_\epsilon) - \mu(A)}{\epsilon}.\]
  where \(A_\epsilon := \{x \in \mathbb{R}^n \mid \text{dist}(x, A) \le \epsilon\}\) is the 
  \(\epsilon\)-thickening of some Borel set \(A\). 
\end{definition}

The isoperimetric problem for the measure \(\mu\) then becomes the problem of finding the set \(A\) 
satisfying \(\mu(A) = 1\) with minimum \(\mu^+(\partial A)\). 

\begin{definition}[Cheeger's inequality, \cite{Milman_2008}]
  Given a measure \(\mu\) on \(\mathbb{R}^n\), we say \(\mu\) satisfy Cheeger's inequality 
  if there exists some \(D\) such that for all \(A\),
  \[\mu(A) \wedge \mu(A^c) \le D\mu^+(A).\]
  We call the largest such \(D\) the inverse Cheeger's constant (or the inverse isoperimetric constant) and 
  denote it by \(D^\mu_{\text{C}}\).
\end{definition}

With these definitions, the KLS conjecture can be equivalently reformulated as the following.

\begin{conjecture}[KLS, \cite{Eldan_2013}]
  Denoting \(\mathscr{M}^n_{\text{iso}}\) the set of all log-concave and isotropic 
  (i.e. \(\mathbb{E}_{X \sim \mu}[X] = 0\) and \(\text{Cov}_{X \sim \mu}(X) = \text{id}\)) probability measures \(\mu\) on 
  \(\mathbb{R}^n\), there exists a \textit{universal} constant \(D\) such that for all 
  \(\mu \in \mathscr{M}^n_{\text{iso}}\), \(\mu\) satisfy the Cheeger's inequality with constant \(D\).
\end{conjecture}

The equivalence of the reformulation follows by completing theorem~\ref{thm:equiv} with two additional 
equivalences.

\begin{theorem}[Milman, \cite{Milman_2008}]\label{thm:milman2}
  For all log-concave measure \(\mu\) on \(\mathbb{R}^n\), the following are equivalent
  \begin{itemize}
    \item \(\mu\) has exponential concentration with constant \(D^\mu_\text{exp}\).
    \item \(\mu\) has first-moment concentration with constant \(D^\mu_\text{FM}\).
    \item \(\mu\) satisfy the Cheeger's inequality with constant \(D^\mu_\text{C}\).
    \item \(\mu\) satisfy the Poincaré inequality: there exists some \(D > 0\) such that for all 
      differentiable \(\phi : \mathbb{R}^n \to \mathbb{R}\) satisfying \(\int \phi \dd\mu = 0\), we have
      \[D \cdot \text{Var}_\mu[\phi] \le \int \|\nabla \phi\|^2 \dd\mu.\]
      We denote the largest such \(D\) by \(D^\mu_{\text{P}}\).
  \end{itemize}
  Furthermore, \(D^\mu_{\text{exp}} \simeq D^\mu_{\text{FM}} \simeq D^\mu_{\text{P}} \simeq D^\mu_{\text{C}}\).
\end{theorem}

With this theorem and proposition \ref{thm:equiv} in mind, it is clear that the KLS conjecture can be 
instead formulated with any of these inequalities instead. We also showcase one of these formulations here 
using the Poincaré inequality:

\begin{conjecture}[KLS, \cite{Eldan_2013}]
  Denoting \(\mathscr{M}^n_{\text{iso}}\) as above, there exist a \textit{universal} constant \(D\) 
  such that for all \(\mu \in \mathscr{M}^n_{\text{iso}}\), \(\mu\) satisfy the Poincaré inequality 
  with constant \(D\).
\end{conjecture}

We remark that isotropic measures satisfy the normalization condition in conjecture~\ref{conj:KLS}. Indeed, if 
\(T : \mathbb{R}^n \to \mathbb{R}\) is a 1-Lipschitz linear function, i.e. is of the form \(v \mapsto w^T v + d\) 
for some \(w \in S^{n - 1}\) and \(d \in \mathbb{R}\), then we have
\[\text{Var}_{\mu}[T] = \text{Var}_{X \sim \mu}\left[\sum_{i = 1}^n w_i X_i + d\right] 
    = \sum_{i, j = 1}^n w_i w_j \text{Cov}_{X \sim \mu}(X_i, X_j) = \sum_{i = 1}^n w_i^2 = 1,\]
as \(\text{Cov}_{X \sim \mu}(X) = \text{id}\).


\subsection{Reduction of KLS to thin-shell}

We will now present a proof of theorem~\ref{thm:KLS_to_TS}.
As a high level overview, recall that the linear-tilt localization of a given measure is a measure-valued martingale
for which the original measure is recovered in the limit. Then, as the concentration of the measure 
relates to the covariance of said measure, we will stop the martingale before the covariance grows too large. 
This allows us to analyze the martingale in a more tractable manner. However, as the the sequence is 
a martingale, some properties are invariant in time and hence allowing us to conclude that these properties 
also hold for the original measure.

We recall the goal of theorem~\ref{thm:KLS_to_TS} is to control \(\text{Var}_\mu[\phi]\) by 
a logarithmic factor of \(\text{Var}_\mu[\|\cdot\|]\). As translating the barycenter of \(\mu\)
does not affect its variance, we may assume \(\mu\) has its barycenter \(\overline{\mu}\) at the origin.
Furthermore, we will assume \(\mu\) is supported on \(B_n(0) \subseteq \mathbb{R}^n\) with 
\(B_n(0)\) the ball at the origin of radius \(n\). The reason for this is due to a concentration 
bound for log-concave measures where one may show most of their densities lie within a compact support.
As a result, the region outside said compact set only contributes a bounded amount (in fact, it decreases 
in \(n\)) to the variance, and does not affect our computation (c.f. \cite{Klartag_2006}). 
Thus, we also have 
\[\text{supp}\ \mu_t = \text{supp}\ F_t\mu \subseteq \text{supp}\ \mu \subseteq B_n(0)\] 
for all \(t > 0\).
 
Let us fix \(\phi : \mathbb{R}^n \to \mathbb{R}\) some 1-Lipschitz function and let \((M_t)\) be the 
martingale as described in corollary~\ref{cor:lim_mart}, we have \(\text{Var}_\mu[\phi] = \text{Var}[\phi(a_\infty)]\) 
where \(a_\infty \sim \mu\).
Then, for all \(t > 0\), by the law of total variance and the martingale property we have
\begin{equation}\label{eq:variance}
  \begin{split}
    \text{Var}_\mu[\phi] = \text{Var}[M_\infty] & = \text{Var}[\mathbb{E}[M_\infty \mid \mu_t]] + \mathbb{E}[\text{Var}[M_\infty \mid \mu_t]]\\
    & = \text{Var}[M_t] + \mathbb{E}[\text{Var}[M_\infty \mid \mu_t]]
  \end{split}
\end{equation}
where we introduce the notation \(\text{Var}[X \mid \mathcal{G}] := \mathbb{E}[(X - \mathbb{E}[X \mid \mathcal{G}])^2 \mid \mathcal{G}]\)
for some random variable \(X\) and sub-\(\sigma\)-algebra \(\mathcal{G}\). Furthermore, by replacing 
\(\phi\) in corollary~\ref{cor:lim_dis} with \(\phi^2\) and denoting the resulting martingale 
\(N_t := \int \phi^2 \dd \mu_t\), we obtain \(N_t \to \phi(a_\infty)^2\) and hence,
\begin{align*}
  \text{Var}[M_\infty \mid \mu_t] & = \mathbb{E}[\phi(a_\infty)^2 \mid \mu_t] - M_t^2 = N_t - M_t^2\\ 
    & = \int \phi^2 \dd \mu_t - \left(\int \phi \dd \mu_t\right)^2 = \text{Var}_{\mu_t}[\phi]
\end{align*}
Combining this with equation~\eqref{eq:variance}, we obtain
\begin{equation}\label{eq:bound}
  \text{Var}_\mu[\phi] = \text{Var}[M_t] + \text{Var}_{\mu_t}[\phi].
\end{equation}
for any \(t \ge 0\). Furthermore, by applying the optional stopping theorem, the same equality holds 
when we take \(t\) to be a stopping time. 

At this point, by recalling proposition~\ref{prop:linear-tilt_ACV}, we recognize that the first term 
\(\text{Var}[M_t]\) is controlled by the operator norm of \(A_t\) (the covariance matrix of \(\mu_t\)).
Thus, to bound the first term, the idea is to choose an appropriate stopping time \(\tau\) to 
stop the process before \(\|A_t\|_\text{op}\) grows too large. Then, by recalling lemma~\ref{lem:brascamp-lieb},
the second term \(\text{Var}_{\mu_\tau}[\phi]\) is bounded by \(\mathbb{E}[\tau^{-1}]\) for which 
we will compute an explicit bound.

We dedicate the remainder of this section to describe said procedure in detail.

% by recalling proposition~\ref{prop:linear-tilt_ACV} and lemma~\ref{lem:brascamp-lieb}, 
% we may recognize that the first term \(\text{Var}[M_t]\) is controlled by the operator norm of \(A_t\) (the covariance 
% matrix of \(\mu_t\)) while the second term \(\text{Var}_{\mu_t}[\phi]\) is bounded by \(t^{-1}\). 
% Thus, with this in mind, the idea now is to choose such an appropriate random time \(\tau\) to stop the process such that 
% \(\mathbb{E}[M]_\tau\) is nicely bounded. Then, the result follow by bounding \(\mathbb{E}[\tau^{-1}]\). 

% At this point, by recalling by lemma~\ref{lem:brascamp-lieb}, we see that the second term can be bounded 
% by \(\mathbb{E}[\text{Var}_{\mu_\tau}[\phi]] \le \mathbb{E}[\tau^{-1}]\) for any stopping time \(\tau\)
% as \(\phi\) is 1-Lipschitz implies \(\|\nabla \phi\|^2 \le 1\).
% % (in fact \(\text{Var}_{\mu_t}[\phi] \le t^{-1} \wedge n^2\) as we have assumed \(\text{supp}\ \mu_t \subseteq B_n(0)\)).

% \subsubsection{Differential of the quadratic variation}

% To bound the term \(\mathbb{E}[M]_\tau\) we will compute its differential and bound it 
% sufficiently such that we reobtain a bound for \([M]_\tau\) after integration. 
% We will show \(\dd[] [M]_t\) is bounded by a quantity concerning \(A_t\). This should not be at all 
% surprising as both \(\dd[] [M]_t\) and \(A_t\) describes the variation of \(M_t\) in a infinitesimal time 
% neighborhood of \(t\).

% We compute
% \begin{align*}
%   \dd M_t & = d \int \phi(x) F_t(x) \mu(\dd x) = \int \phi(x) \langle x - a_t, \dd W_t \rangle \mu_t(\dd x)\\
%   & = \left\langle \int \phi(x)(x - a_t)\mu_t(\dd x), \dd W_t\right\rangle
% \end{align*}
% and so, by considering the component-wise quadratic variation, we have
% \begin{equation}\label{eq:diff_qvar}
%   \dd[] [M]_t = \left\| \int \phi(x)(x - a_t)\mu_t(\dd x) \right\|^2 \dd t.
% \end{equation}
% Then, denoting \(\theta\) the vector \(\int \phi(x)(x - a_t)\mu_t(\dd x)\) normalized to have norm 1, so 
% \[\left\langle \theta, \int \phi(x)(x - a_t)\mu_t(\dd x)\right\rangle = \left\|\int \phi(x)(x - a_t)\mu_t(\dd x)\right\|\]
% we observe,
% \begin{equation}\label{eq:red_a}
%   \begin{split}
%     \dd[] [M]_t & = \left\langle \theta, \int \phi(x)(x - a_t)\mu_t(\dd x)\right\rangle^2 \dd t
%       = \left\langle \theta, \int \left(\phi(x) - \int \phi \dd \mu_t\right)(x - a_t)\mu_t(\dd x)\right\rangle^2 \dd t\\
%     & = \left(\int \left(\phi(x) - \int \phi \dd \mu_t\right) \langle \theta, x - a_t\rangle \mu_t(\dd x)\right)^2 \dd t\\
%     & \le \left(\int \left(\phi(x) - \int \phi \dd \mu_t\right)^2 \mu_t(\dd x)\right) \left(\int \langle \theta, x - a_t\rangle^2 \mu_t(\dd x)\right) \dd t\\
%     & = \text{Var}_{\mu_t}[\phi] \left(\int \theta^T (x - a_t)^{\otimes 2} \theta \mu_t(\dd x)\right) \dd t
%       = \text{Var}_{\mu_t}[\phi] (\theta^T A_t \theta)\dd t\\ 
%     & \le \text{Var}_{\mu_t}[\phi] \|A_t\|_{\text{op}} \dd t.
%   \end{split}
% \end{equation}
% where the inequality follows by the Cauchy-Schwarz inequality and \(\|\cdot\|_{\text{op}}\) denotes the operator norm. 
% Thus, as we know \(\text{Var}_{\mu_t}[\phi] \le t^{-1}\), the problem is now reduced to that of
% bounding \(\|A_t\|_{\text{op}}\).

\subsubsection{Analysis of the covariance matrix}

As demonstrated in section~\ref{sec:construct}, we know the limiting behavior of the covariance matrices, namely 
\(A_t \to 0\) point-wise as \(t \to \infty\). This was important for us to establish the existence of the limit 
of \((a_t)\) and \((M_t)\). However, as shown above, we now require some quantitative bounds for the operator 
norm of \(A_t\). For this purpose, we first compute some useful properties of \(A_t\).

Observing 
\[\int \dd F_t(x) \mu(\dd x) = \int \langle x - a_t, \dd W_t\rangle \mu_t(\dd x) = 
  \left\langle\int x \mu_t(\dd x) - a_t, \dd W_t\right\rangle = 0,\]
we have
\begin{equation}\label{eq:diff_center}
  \begin{split}
    \dd a_t & = \dd \int x F_t(x) \mu(\dd x) = \int x \dd F_t(x) \mu(\dd x) 
        = \int (x - a_t) \dd F_t(x) \mu(\dd x)\\
      & = \int (x - a_t) \langle x - a_t, \dd W_t \rangle F_t(x) \mu(\dd x) 
        = \int (x - a_t)^{\otimes 2} \dd W_t \mu_t(\dd x) = A_t \dd W_t
  \end{split}
\end{equation}
where the second to last equality used the fact that \(v \langle v, w \rangle = v^{\otimes 2} w\) for 
any appropriate \(v, w\).

Similarly, computing using Itô's formula, we have
\begin{equation}\label{eq:diff_cov_aux}
  \begin{split}
    \dd A_t & = \dd \int (x - a_t)^{\otimes 2} F_t(x) \mu(\dd x)\\
      & = \int (x - a_t)^{\otimes 2} \dd F_t(x) + F_t(x) \dd (x - a_t)^{\otimes 2}\\
      & \hspace{1cm} - 2 (x - a_t) \otimes \dd[][a_t, F_t(x)]_t +F_t(x)\dd[][a_t]_t\mu(\dd x).
  \end{split}
\end{equation}
The second term vanishes as 
\[\int F_t(x) \dd (x - a_t)^{\otimes 2} \mu(\dd x) = -2 \dd a_t \otimes 
  \overbrace{\int (x - a_t) \mu_t(\dd x)}^{= 0} = 0.\]
Also, by equation~\eqref{eq:diff_center}, \(\dd a_t = A_t \dd W_t\) implying \(\dd[][a_t]_t = A_t^2 \dd t\).
Finally, as both \((a_t)\) and \((F_t(x))\) are martingales, \(\dd[] [a_t, F_t(x)]_t = F_t(x) A_t x \dd t\) 
and the third term becomes
\begin{align*}
  -2 \int (x - a_t) \otimes \dd[] [a_t, F_t(x)] \mu_t(\dd x) 
  & = - 2A_t \left(\int (x - a_t) \otimes x \mu_t(\dd x)\right) \dd t \\
  & = -2 A_t \left(\overbrace{\int (x - a_t)^{\otimes 2} \mu_t(\dd x)}^{A_t} + 
    \overbrace{\int (x - a_t) \mu_t(\dd x)}^{= 0} \otimes a_t \right) \dd t\\ 
  & = -2 A_t^2 \dd t.
\end{align*}
Hence, combining these and equation~\eqref{eq:stoch_loc} together in \eqref{eq:diff_cov_aux}, we have
\[\dd A_t = \int (x - a_t)^{\otimes 2} \langle x - a_t, \dd W_t \rangle \mu_t(\dd x) - A_t^2 \dd t\]
However, since we wish to bound \(A_t\) from above, as the drift term \(- A_t^2 \dd t\) only contributes 
negatively, an upper bound for the process of the form 
\(\int (x - a_t)^{\otimes 2} \langle x - a_t, \dd W_t \rangle \mu_t(\dd x)\) is also sufficient for \(A_t\).
Hence, we proceed by ignoring the drift term and redefine the process \(A_t\) such that
\begin{equation}\label{eq:diff_cov}
  \dd A_t = \int (x - a_t)^{\otimes 2} \langle x - a_t, \dd W_t \rangle \mu_t(\dd x).
\end{equation}
With this justification, we now proceed to bound the operator norm of this new \(A_t\). In particular, 
as \(A_t\) is symmetric, we recall that \(\|A_t\|_{\text{op}} = \max_{i= 1, \cdots, n} \lambda_i(t) = \|(\lambda_i(t))_{i = 1}^n\|_\infty\) 
where \(\lambda_i(t)\) denotes the distinct eigenvalues of \(A_t\). Hence, it suffices to find a bound for the potential 
\begin{equation}
  \Phi^{\alpha}(t) = \sum_{i = 1}^n |\lambda_i(t)|^{\alpha} = \|(\lambda_i(t))_{i = 1}^n\|_\alpha^\alpha
\end{equation} 
for some \(\alpha > 0\). Furthermore, as \(A_t\) is positive 
semi-definite, \(\lambda_i(t) \ge 0\) for all \(i = 1, \cdots, n\) and thus we have
\(\Phi^\alpha(t) = \sum_{i = 1}^n \lambda_i(t)^{\alpha}\). Again, to proceed, we will attempt to compute 
\(\dd \Phi^\alpha(t)\) at some \(t = t_0 > 0\) utilizing the following lemma.

\begin{lemma}\label{lem:diff_eig}
  If \(A = [a_{ij}]\) is a diagonal matrix with distinct eigenvalues \(\lambda_i, \cdots, \lambda_n\), then 
  for all \(i, j, k, l, m \in {1, \cdots n}\), we have
  \begin{itemize}
    \item \(\pdv{\lambda_i}{a_{jk}} = \delta_{ij} \delta_{ik}\);
    \item whenever \(i \neq j\), \(\pdv[2]{\lambda_i}{a_{ij}} = 2(\lambda_i - \lambda_j)^{-1}\);
    \item and for \(j \neq l, k \neq m\) or \(i \neq j\) and \(i \neq k\),
      \(\pdv[2]{\lambda_i}{a_{jk}}{a_{lm}} = 0\),
  \end{itemize}
  where \(\delta_{ij}\) denotes the Kronecker delta function.
\end{lemma}

As this lemma requires the matrix to be diagonal, denoting \(e_1, \cdots, e_n\) as the normalized 
eigenbasis of \(A_{t_0}\) (they are in fact orthonormal as \(A_{t_0}\) is positive semi-definite), 
we will consider \(A_t\) with respect to this basis by considering the entries 
\[a_{ij}(t) := \langle e_i, A_t e_j\rangle.\]
Using equation~\eqref{eq:diff_cov}, we compute 
\begin{align*}
  \dd a_{ij}(t) & = \left\langle e_i, \left(\int (x - a_t)^{\otimes 2} 
    \langle x - a_t, \dd W_t \rangle \mu_t(\dd x)\right) e_j \right\rangle\\
    & = \left\langle\int \langle e_i, (x - a_t)^{\otimes 2} e_j\rangle
      (x - a_t) \mu_t(\dd x), \dd W_t\right\rangle
      = \langle \xi_{ij}, \dd W_t\rangle
\end{align*}
where we introduce the notation \(\xi_{ij} = \int \langle e_i, (x - a_t)^{\otimes 2} e_j\rangle (x - a_t) \mu_t(\dd x).\)
Thus, combining this with lemma~\ref{lem:diff_eig}, 
denoting \(\lambda_i = \lambda_i(t_0)\), we have by Itô's formula
\begin{equation}
  \begin{split}
    \dd \lambda_i(t) 
    & = \sum_{j, k = 1}^n \pdv{\lambda_i}{a_{jk}} \dd a_{jk}(t) 
      + \frac{1}{2}\sum_{j, k = 1}^n \sum_{l, m = 1}^n \pdv[2]{\lambda_i}{a_{jk}}{a_{lm}} \dd[] [a_{jk}, a_{lm}]_t\\
    & = \langle \xi_{ii}, \dd W_t\rangle + \sum_{j \neq i} \frac{\dd[][a_{ij}]_t}{\lambda_i - \lambda_j}
      = \langle \xi_{ii}, \dd W_t\rangle + \sum_{j \neq i} \frac{\|\xi_{ij}\|^2}{\lambda_i - \lambda_{j}} \dd t.
  \end{split}
\end{equation}
at \(t = t_0\). As a result, it is also clear that \(\dd[][\lambda_i(t)]_{t_0} = \|\xi_{ii}\|^2 \dd t\). 

Again applying Itô's formula, we may finally compute
\begin{align*}
  \dd \Phi^\alpha(t) & = \sum_{i = 1}^n \pdv{\Phi^\alpha}{\lambda_i}\Big\vert_{t = t_0} \dd \lambda_i(t)
    + \frac{1}{2}\sum_{i, j = 1}^n \pdv[2]{\Phi^\alpha}{\lambda_i}{\lambda_j}\Big\vert_{t = t_0} \dd[][\lambda_i, \lambda_j]_{t}\\
  & = \alpha \sum_{i = 1}^n \lambda_i^{\alpha - 1} \dd \lambda_i(t)
    + \frac{1}{2}\alpha(\alpha - 1)\sum_{i = 1}^n \lambda_i^{\alpha - 2} \dd[][\lambda_i(t)]_{t}\\
  & = \alpha \sum_{i = 1}^n \lambda_i^{\alpha - 1} \left(\langle \xi_{ii}, \dd W_t\rangle 
    + \sum_{j \neq i} \frac{\|\xi_{ij}\|^2}{\lambda_i - \lambda_j} \dd t\right) 
    + \frac{1}{2}\alpha(\alpha - 1)\sum_{i = 1}^n \lambda_i^{\alpha - 2} \dd[][\lambda_i(t)]_{t}\\
  & = \alpha \sum_{i \neq j}\lambda_i^{\alpha - 1} \frac{\|\xi_{ij}\|^2}{\lambda_i - \lambda_j} \dd t 
    + \frac{1}{2}\alpha(\alpha - 1)\sum_{i = 1}^n \lambda_i^{\alpha - 2} \|\xi_{ii}\|^2 \dd t
    + \left\langle\underbrace{\alpha \sum_{i = 1}^n \lambda_i^{\alpha - 1} \xi_{ii}}_{=: v_t}, \dd W_t\right\rangle\\
  & = \frac{1}{2}\alpha \sum_{i \neq j}\|\xi_{ij}\|^2\frac{\lambda_i^{\alpha - 1} - \lambda_j^{\alpha -1}}{\lambda_i - \lambda_j}\dd t
    + \frac{1}{2}\alpha(\alpha - 1)\sum_{i = 1}^n \lambda_i(t)^{\alpha - 2} \|\xi_{ii}\|^2 \dd t + \langle v_t, \dd W_t\rangle\\
  & \le \frac{1}{2}\alpha(\alpha - 1) \sum_{i \neq j}\|\xi_{ij}\|^2 (\lambda_i \vee \lambda_j)^{\alpha - 2} \dd t
    + \frac{1}{2}\alpha(\alpha - 1)\sum_{i = 1}^n \lambda_i(t)^{\alpha - 2} \|\xi_{ii}\|^2 \dd t + \langle v_t, \dd W_t\rangle\\
  & = \frac{1}{2}\alpha(\alpha - 1) \sum_{i, j = 1}^n\|\xi_{ij}\|^2 (\lambda_i \vee \lambda_j)^{\alpha - 2} \dd t
    + \langle v_t, \dd W_t\rangle
    \le \alpha^2 \sum_{i, j = 1}^n\|\xi_{ij}\|^2 \lambda_i^{\alpha - 2} \dd t + \langle v_t, \dd W_t\rangle,
\end{align*}
where the first inequality holds as 
\[\frac{\lambda_i^{\alpha - 1} - \lambda_j^{\alpha -1}}{\lambda_i - \lambda_j}
  = \lambda_i^{\alpha - 2} + \lambda_i^{\alpha - 3}\lambda_j + \cdots + \lambda_i^{\alpha - 2} 
  \le (\alpha - 1)(\lambda_i \vee \lambda_j)^{\alpha - 2}.\]
Thus, we have shown 
\begin{equation}\label{eq:potential_bound}
  \dd \Phi^\alpha(t) \le \alpha^2 \sum_{i = 1}^n \lambda_i(t)^{\alpha - 2} \sum_{j = 1}^n \|\xi_{ij}\|^2 \dd t + \langle v_t, \dd W_t\rangle
\end{equation}
where \(v_t := \alpha \sum_{i = 1}^n \lambda_i^{\alpha - 1}\xi_{ii}\).

By recalling that our goal is to bound \(\|A_t\|_{\text{op}}\) from above (c.f. equation~\eqref{eq:bound} and \eqref{eq:red_a}), 
we may assume without loss of generality that \(\|A_t\|_{\text{op}} \ge 1\). Thus, applying the reverse Cauchy-Schwarz inequality to 
equation~\eqref{eq:potential_bound}, we have
\begin{align*}
  \dd \Phi^\alpha(t) & \le 2\alpha^2 \sum_{i = 1}^n \lambda_i(t)^{\alpha - 2} \sum_{j = 1}^n \|\xi_{ij}\|^2 \dd t + \langle v_t, \dd W_t\rangle\\
    & \le 2\alpha^2 \|A_t\|_{\text{op}}^2 \sum_{i = 1}^n \lambda_i(t)^{\alpha - 2} \sum_{j = 1}^n \|\xi_{ij}\|^2 \dd t + \langle v_t, \dd W_t\rangle\\
    & \lesssim 2\alpha^2\sum_{i = 1}^n \lambda_i(t)^\alpha \sum_{j = 1}^n \|\xi_{ij}\|^2 \dd t + \langle v_t, \dd W_t\rangle.
\end{align*} 
Thus, defining \(K_t := \sup_i \sum_{j = 1}^n \|\xi_{ij}\|^2\), we have the bound 
\begin{equation}\label{eq:potential_bound_2}
  \dd \Phi^\alpha(t) \lesssim 2\alpha^2 K_t \Phi^\alpha(t) \dd t + \langle v_t, \dd W_t\rangle.
\end{equation}

\subsubsection{Stopping the process early}

As outlined in the beginning of this section, we will stop the process early in order to provide a 
bound for the right hand side of equation~\eqref{eq:bound}. By observing equation~\eqref{eq:red_a}, we hypothesize that
we should stop the process once \(\|A_t\|_{\text{op}}\) grows too large. As a result we define the stopping time 
\[\tau := \inf\{t > 0 \mid \|A_t\|_{\text{op}} > 2\} \wedge 1.\]
By the optional stopping theorem we have
\begin{align*}
  [M]_\tau & = \int_0^\tau \dd[] [M]_t 
      \le \int_0^\tau \overbrace{\text{Var}_\mu[\phi]}^{\le t^{-1} \wedge n^2} \overbrace{\|A_t\|_{\text{op}}}^{\le 2} \dd t\\
    & \le 2 \int_0^\tau t^{-1} \wedge n^2 \dd t \le 2 \int_0^1 t^{-1} \wedge n^2 \dd t = 2 + 4 \log n.
\end{align*}
Combining this with equation~\eqref{eq:bound}, we obtain
\begin{equation}\label{eq:tau_bd}
  \text{Var}_\mu[\phi] \le 2 + 4 \log n + \mathbb{E}[\tau^{-1}],
\end{equation}
and it remains to find an upper bound for \(\mathbb{E}[\tau^{-1}]\). Observing that \(t < \tau\) whenever 
\(\Phi^\alpha(t) < 2^\alpha\), we define the \(\sigma\) the first time for which the potential \(\Phi^\alpha(t)\) reaches \(2^\alpha\),
namely
\[\sigma := \inf \{t > 0 \mid \Phi^\alpha(t) = 2^\alpha\},\] 
we have \(\sigma^{-1} \ge \tau^{-1}\) and so it suffices to bound \(\sigma\) from below. 

For simplicity (the general computation is similar albeit much more technical), let us ignore the stochastic term in 
equation~\eqref{eq:potential_bound_2} and regard it as an ODE. Then, by Gronwall's inequality, if we can 
find some constant \(K\) such that \(K_t \le K\) for all \(t \le \tau\), we have the bound
\[S_t \le n e^{2\alpha^2 K t}.\]
Thus, substituting \(\sigma\) into the above, we have 
\[2^\alpha = S_\sigma \le ne^{2\alpha^2 K\sigma}\]
implying 
\[\frac{\alpha \log 2 - \log n}{2\alpha^2 K} \le \sigma \le \tau.\]
Then, taking \(\alpha = 10K\log n\), it is easy to check that 
\[\frac{1}{10K \log n} \le \frac{\alpha \log 2 - \log n}{2\alpha^2 K}\]
implying \(\mathbb{E}[\tau^{-1}] \le 10K \log n\). Of course, this deduction only holds while ignoring the stochastic term 
\(\langle v_t, \dd W_t\rangle\). Nonetheless, this is justified as one can show that \(\|v_t\|_2\) is bounded 
\(\alpha \Phi^\alpha(t)\) and so the same analysis holds by applying the stochastic Gronwall's inequality
(c.f. second part of lemma 34 in \cite{Lee_2018}).

Finally, to find a bound for \((K_t)\), we employ the following lemma.

\begin{lemma}[Lemma 1.6 in \cite{Eldan_2013}]\label{lem:final_bd}
  Denoting \(C_{\text{TS}}^n\) as in theorem~\ref{thm:KLS_to_TS}, there exists a constant \(C\) such that 
  for any log-concave, isotropic probability measure \(\mu\), we have
  \[\sup_{\theta \in S^{n - 1}}\sum_{i, j = 1}^n 
    \mathbb{E}_{X \sim \mu}[\langle X, e_i\rangle \langle X, e_j\rangle \langle X, \theta\rangle]^2 \le 
    C \sum_{k = 1}^n \frac{(C_{\text{TS}}^n)^2}{k},\]
  where \(\{e_1, \cdots, e_n\}\) is any orthonormal basis on \(\mathbb{R}^n\).
\end{lemma}

Recalling that
\[\xi_{ij} = \mathbb{E}_{X + a_t \sim \mu_t}[\langle e_i, X^{\otimes 2} e_j\rangle X] = 
  \mathbb{E}_{X + a_t \sim \mu_t}[\langle X, e_i\rangle \langle X, e_j\rangle X],\]
we have by Parseval's identity 
\begin{align*}
  K_t & = \sup_i \sum_{j = 1}^n \|\xi_{ij}\|^2 
    = \sup_i \sum_{j = 1}^n \|\mathbb{E}_{X + a_t \sim \mu_t}[\langle X, e_i\rangle \langle X, e_j\rangle X]\|^2\\
  & = \sup_i \sum_{j = 1}^n \sum_{k = 1}^n 
    \left\langle \mathbb{E}_{X + a_t \sim \mu_t}[\langle X, e_i\rangle \langle X, e_j\rangle X], e_k\right\rangle^2\\
  & = \sup_i \sum_{j, k = 1}^n \mathbb{E}_{X + a_t \sim \mu_t}[\langle X, e_i\rangle \langle X, e_j\rangle \langle X, e_k\rangle]^2\\
  & \le \sup_{\theta \in S^{n - 1}}\sum_{i, j = 1}^n 
    \mathbb{E}_{X + a_t \sim \mu}[\langle X, e_i\rangle \langle X, e_j\rangle \langle X, \theta\rangle]^2.
\end{align*}
We note that we cannot direct apply lemma~\ref{lem:final_bd} at this point since the measure \(\mu_t\) 
might not be isotropic. Hence, to be able to use the lemma, we need to normalize the covariance of \(\mu_t\). 
Namely, taking \(X + a_t \sim \mu_t\), we define \(Y = A^{-1/2} X\) which by construction is isotropic. 
Thus, by observing that 
\[\mathbb{E}_{X + a_t \sim \mu}[\langle X, e_i\rangle \langle X, e_j\rangle \langle X, \theta\rangle]^2 
  \le \|A_t\|_{\text{op}}^3 \mathbb{E}_{X + a_t \sim \mu}[\langle Y, e_i\rangle \langle Y, e_j\rangle \langle Y, \theta\rangle]^2,\]
we have 
\begin{equation}\label{eq:K_bd}
  \begin{split}
    K_t & \le \sup_{\theta \in S^{n - 1}}\sum_{i, j = 1}^n 
        \mathbb{E}_{X + a_t \sim \mu}[\langle X, e_i\rangle \langle X, e_j\rangle \langle X, \theta\rangle]^2\\
      & \le \|A_t\|_{\text{op}}^3 \sup_{\theta \in S^{n - 1}}\sum_{i, j = 1}^n 
        \mathbb{E}_{X + a_t \sim \mu}[\langle Y, e_i\rangle \langle Y, e_j\rangle \langle Y, \theta\rangle]^2
        \le 8 C \sum_{k = 1}^n \frac{(C_{\text{TS}}^n)^2}{k}
      \end{split}
\end{equation}
where the last inequality follows as \(\|A_t\|_{\text{op}} \le 2\) for all \(t < \tau\). 
    
At last, combining equation~\eqref{eq:K_bd} and~\eqref{eq:tau_bd}, we have
\[\text{Var}_\mu[\phi] \le 2 + \log n\left(4 + \overbrace{80 C \sum_{k = 1}^n \frac{1}{k}}^{\Theta(\log n)}(C_{\text{TS}}^n)^2\right)
   = \Theta_n((C_{\text{TS}}^n \log n)^2)\]
implying there exists a constant \(R > 0\) such that for all 1-Lipschitz \(\phi\), 
\(\sqrt{\text{Var}_\mu[\phi]} \le R C_{\text{TS}}^n \log n\), i.e. \(\mu\) is 
\(R C_{\text{TS}}^n \log n\)-concentrated and so, \(C_{\text{con}}^n \le R C_{\text{TS}}^n \log n\)
as required.

% \subsection{Lee-Vempala's bound}

% \begin{theorem}[Lee-Vempala, \cite{Lee_2016}]
  
% \end{theorem}

% By taking \(\alpha = 2\) in equation~\eqref{eq:potential_bound}, we have 
% \[\dd \Phi^2(t) \le 4 \sum_{i, j = 1}^n \|\xi_{ij}\|^2 + \langle v_t, \dd W_t\rangle,\]
% where \(v_t = 2 \sum_{i = 1}^n \lambda_i \xi_{ii}\).

\newpage
\section{Almost Constant Bound}
\label{sec:chen}
\input{sections/chen.tex} 

\newpage
\bibliographystyle{alpha}
\bibliography{biblio}

\end{document}
