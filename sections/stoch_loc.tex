We will in this section introduce the a class of stochastic localization schemes which are the object 
central to this essay. To gain an intuition for these objects, we will present several examples which 
will be studied further in subsequent sections. 

We will work in general Borel spaces \((\mathcal{X}, \Sigma)\) for this section while for restrict our 
focus to either the Euclidean space \(\mathbb{R}^n\) or the Boolean hypercube \(\{-1, 1\}^n\) 
in subsequent sections. We take \((\Omega, \mathscr{F}, \mathbb{P})\) our underlying probability space 
and we introduce the notation \(\mathcal{M}(\mathcal{X})\) for the space of probability measures on 
\(\mathcal{X}\).

\begin{definition}[Stochastic localization process, \cite{Chen_2022}]
  Given \(\mu \in \mathcal{M}(\mathcal{X})\), a measure-valued stochastic process 
  \((\mu)_{t \ge 0}\) is said to be a stochastic localization of \(\mu\) if 
  \begin{enumerate}[label=(L\arabic*), start=0]
    \item \label{L0} \(\mu_0 = \mu\).
    \item \label{L1} For all \(t \ge 0\), \(\mu_t\) is a probability measure almost everywhere, i.e. 
      \(\mathbb{P}(\mu_t(\mathcal{X}) = 1) = 1\).
    \item \label{L2} For all \(A \in \Sigma\), \((\mu_t(A))_{t \ge 0}\) is a martingale.
    \item \label{L3} For all \(A \in \Sigma\), \(\mu_t(A)\) converges almost everywhere to \(0\) 
      or \(1\) as \(t \to \infty\).
  \end{enumerate}
\end{definition}

\begin{definition}[Stochastic localization scheme, \cite{Chen_2022}]
  Denoting \(\mathcal{L}(\mu)\) the set of all stochastic localization processes of the measure
  \(\mu\), a stochastic localization scheme is a map 
  \[\Phi : \mathcal{M}(\mathcal{X}) \to \prod_{\mu \in \mathcal{X}} \mathcal{L}(\mu)\] 
  such that \(\Phi(\mu) \in \mathcal{L}(\mu)\) for all \(\mu \in \mathcal{M}(\mathcal{X})\).
\end{definition}

We say a stochastic localization is discrete if \(t\) takes value in \(\mathbb{N}\) and continuous 
if \(t\) takes value in \(\mathbb{R}_{\ge 0}\). For shorthand, we denote \((\mu_k)_k\) for a
discrete stochastic localization of \(\mu\).

\subsection{Examples of stochastic localizations}

An example of a stochastic localization scheme is the coordinate by coordinate localization scheme 
on \(\mathcal{X} = \{-1, 1\}^n\). This scheme relates to the Glauber dynamics for which the stochastic 
localization scheme provides a mixing bound. We shall examine the property in section \todo{TODO!}, 
though we will construct the scheme now. 

Given a probability measure \(\mu\) on \(\{-1, +1\}^n\), we introduce the random variable \(X \sim \mu\), and 
\(Y\) a uniform random variable over all permutations of \([n] = \{1, \cdots, n\}\) independent of \(X\). 
Then, the coordinate by coordinate stochastic localization of \(\mu\) is the process \((\mu_k)_{k}\)
such that for all \(x \in \{-1, 1\}^n\),
\[\mu_k(x) = \mathbb{P}(X = x \mid X_{Y_1}, \cdots, X_{Y_{n \wedge k}}).\]
Namely, \(\mu_k\) is the law of \(X\) conditioned on \(X_{Y_1}, \cdots, X_{Y_i}\).

\((\mu_k)_{k}\) is indeed a stochastic localization of \(\mu\). It is clear that \ref{L0} and \ref{L1} are 
satisfied. By construction of \((\mu_k)_k\), denoting 
\(\mathscr{F}_k := \sigma(X_{Y_1}, \cdots, X_{Y_{n \wedge k}})\), we have by the tower property
\[\mathbb{E}[\mu_{k + 1}(x) \mid \mathscr{F}_k] 
  = \mathbb{E}[\mathbb{E}[\mathbb{P}(X = x \mid X) \mid \mathscr{F}_{k + 1}] \mid \mathscr{F}_k]
  = \mathbb{E}[\mathbb{P}(X = x \mid X) \mid \mathscr{F}_k] = \mu_k(x)\]
implying \((\mu_k(x))\) a martingale as required for \ref{L2}. Finally, it is clear that
\[\lim_{k \to \infty} \mu_k(x) = \mu_n(x) = \mathbb{P}(X = x \mid X) = \mathbf{1}_{\{X = x\}} \in \{0, 1\}\]
implying \ref{L3}.

An analogous construction of the coordinate by coordinate stochastic localization scheme in \(\mathbb{R}^n\) 
is the random subspace localization. Similar to before, for a probability measure \(\mu\) on \(\mathbb{R}^n\),
we introduce the random variable \(X \sim \mu\) and \(Y\) a uniform random variable on \(O(n)\) 
(so the column vectors \(\{Y_1, \cdots, Y_n\}\) form an orthonormal basis of \(\mathbb{R}^n\)) 
independent of \(X\). Then, we define the random subspace stochastic localization of \(\mu\) as \((\mu_k)_k\) 
where \(\mu_k\) is the law of \(X\) conditioned on \(\langle X, Y_1\rangle, \cdots, \langle X, Y_{n \wedge k}\rangle\).

\subsubsection{Linear-tilt localization}

An important class of stochastic localization schemes are the linear-tilt schemes. Introduced by Eldan 
in \cite{Eldan_2013}, linear-tilt schemes had been vital in the recent progress regarding the KLS 
conjecture. Informally, given a probability measure \(\mu\) on \(\mathbb{R}^n\), the linear-tilt scheme
of \(\mu\) is constructed recursively in which at each step, we pick a random direction and multiply 
the density at this time with a linear function along this direction (i.e. a tilt along a random 
direction). 

Let \(\mu\) be a probability measure on \(\mathbb{R}^n\), we introduce the following definition.
\begin{definition}[Barycenter]
  The barycenter of \(\mu\) with respect to the function \(F : \mathbb{R}^n \to \mathbb{R}\) is
  \[\bar{\mu}(F) := \int_{\mathbb{R}^n} x F(x) \mu(\dd x).\]
  In the case that \(F = \text{id}\), we simply write \(\bar{\mu} = \bar{\mu}(F) = \mathbb{E}_{X \sim \mu}[X]\).
\end{definition}

Then, given \((W_t)_{t \ge 0}\) be a standard Wiener process in \(\mathbb{R}^n\), we define the random functions 
\((F_t)_{t \ge 0}\) to be the solution of the following infinite system of SDEs 
(existence and uniqueness is established by theorem 5.2 in \cite{Øksendal_2003}): 
\begin{equation}\label{eq:stoch_loc}
  F_0 = 1, \dd F_t(x) = \langle x - \bar{\mu}(F_t), \dd W_t \rangle F_t(x),
\end{equation}
for all \(x \in \mathbb{R}^n\). We shall from this point forward denote the random variables 
\(a_t := \bar{\mu}(F_t)\).

By applying Itô's formula, we make the following useful observation: for all \(x \in \mathbb{R}^n\),
\begin{equation}
  \dd \log F_t(x) = \frac{\dd F_t(x)}{F_t(x)} - \frac{\dd\hspace{0pt} [F(x)]_t}{2F_t(x)^2} 
    = \langle x - a_t, \dd W_t \rangle - \frac{1}{2}\|x - a_t\|^2 \dd t
\end{equation}
where the second equality follows by the construction of \(F\). Hence, as \(\log F_0(x) = 0\), we 
observe
\begin{align*}
  \log F_t(x) & = \int_0^t \langle x - a_s, \dd W_s \rangle - \frac{1}{2}\int_0^t \|x - a_s\|^2 \dd s\\
    & = \left(\langle x, W_t \rangle - \int_0^t \langle a_s, \dd W_s \rangle\right)
      - \left(\frac{t}{2}\|x\|^2 + \frac{1}{2}\int_0^t\|a_s\|^2 \dd s - \int_0^t \langle x, a_s \rangle \dd s\right)\\
    & = - \left(\int_0^t \langle a_s, \dd W_s \rangle + \frac{1}{2}\|a_s\|^2 \dd s\right) + 
      \langle x, a_t + W_t \rangle - \frac{t}{2}\|x\|^2.
\end{align*}
Thus, taking \(\dd z_t := \langle a_t, \dd W_t\rangle + \frac{1}{2} \|a_t\|^2 \dd t\) and 
\(v_t := a_t + W_t\), we observe \(F_t(x)\) is of the form
\begin{equation}\label{eq:stoch_loc_alt}
  F_t(x) = e^{z_t + \langle x, v_t \rangle - \frac{t}{2}\|x\|^2},
\end{equation}
for given Itô processes \((z_t), (v_t)\).

With this formulation of \(F_t(x)\) in mind, it follows \(F_t\) is non-negative, and so, 
we may define the linear-tilt localization of \(\mu\) to be the process \((\mu_t)_t\) 
where \(\dd \mu_t = F_t \dd \mu\). The remainder of this section is devoted to showing \((\mu_t)_t\) is 
indeed a stochastic localization of \(\mu\) (\ref{L0} is clear so it remains to show \ref{L1}, \ref{L2} and \ref{L3}).

\begin{proposition}
  For all \(t \ge 0\), \(\mu_t\) is a probability measure almost everywhere (and so \ref{L1} is satisfied).
\end{proposition}
\begin{proof}
  As \(\mu\) is a probability measure, it suffices to show \(\partial_t \mu_t(\mathbb{R}^n) = 0\).
  To prove this, we first consider the discrete stochastic integral on the lattice \(\Lambda = \mathbb{Z}^n \dd t\)
  for some \(\dd t > 0\). Then, constructing \(\mu_t^{\dd t}\) on \(\Lambda\) via the same process as \(\mu_t\),
  for all \(t = k \dd t \in \Lambda\),
  \begin{align*}
    \partial_t \mu_t^{\dd t}(\mathbb{R}^n) 
      & = \int_{\mathbb{R}^n} \dd F_t(x) \mu(\dd x) 
        = \int_{\mathbb{R}^n} \langle x - a_t, \dd W_t\rangle F_t(x) \mu(\dd x)\\
      & = \left\langle \int_{\mathbb{R}^n} (x - a_t) F_t(x)\mu(\dd x), \dd W_t \right\rangle
        = \langle a_t - a_t\mu_t(\mathbb{R}^n), \dd W_t \rangle = 0
  \end{align*}
  by induction on \(k\). However, by the very construction of the stochastic integral, the densities 
  of \(\mu_t^{\dd t}\): \(F_t^{\dd t}\) converges a.e. to \(F_t\) as \(\dd t \to 0\) implying 
  \(\mu_t^{\dd t} \to \mu_t\) weakly and so, \(\mu_t^{\dd t}(\mathbb{R}^n) \to \mu_t(\mathbb{R}^n)\)
  resulting in \(\mu_t(\mathbb{R}^n) = 1\) as required.
\end{proof}

To show \((\mu_t)\) satisfies \ref{L3}, we study the limiting behavior of \((\mu_t)\) as \(t \to \infty\). 
To achieve this, we will consider the covariance matrices
\begin{equation}
  A_t := \text{Cov}[\mu_t] = \int (x - a_t) \otimes (x - a_t) \mu_t(\dd x),
\end{equation}
where \(\otimes\) denotes the Kronecker product. In particular, we will show \((A_t)_{ij} \to 0\) for all
 \(i, j \in \{1, \cdots, n\}\) as \(t \to \infty\) allowing us to conclude \((\mu_t)\) converges weakly 
 to some Dirac measure. Indeed, this is a direct consequence of the following lemma.

\begin{lemma}[Brascamp-Lieb inequality, \cite{Brascamp_1976}]\label{lem:brascamp-lieb}
  Given \(V : \mathbb{R}^n \to \mathbb{R}\) convex and \(K > 0\), if \(\nu\) is an isotropic probability 
  measure on \(\mathbb{R}^n\) of the form 
  \[\dd \nu = Ze^{-V(x) - \frac{1}{2K}\|x\|^2}\dd \text{Leb}^n\]
  with \(Z\) being the normalization constant, then \(\nu\) satisfy the Poincaré inequality, i.e. 
  for all differentiable \(\phi\),
  \[K\text{Var}_\nu[\phi] \le \int \|\nabla\phi\|^2 \dd\nu.\]
\end{lemma}

With this lemma in mind, by taking \(\nu = \mu_t\) using equation~\eqref{eq:stoch_loc_alt} and defining
\(\pi_i(x) := x_i\), we have by the Cauchy-Schwarz inequality
\[(A_t)_{ij} \le \sqrt{\text{Var}_{\mu_t}[\pi_i]}\sqrt{\text{Var}_{\mu_t}[\pi_j]} 
  \le \max_{k = 1, \cdots, n} \frac{1}{t}\int \|\nabla \pi_k\|^2 \dd \mu_t\]
Again, using equation~\eqref{eq:stoch_loc_alt}, we note that any realizations of \((F_t(x))\) is eventually 
decreasing in \(t\) for all \(x \neq 0\), implying 
\[\sup_{t > 0} \max_{k = 1, \cdots, n} \int \|\nabla \pi_k\|^2 \dd \mu_t = 
\sup_{t > 0} \max_{k = 1, \cdots, n} \int x_k^2 \dd \mu_t < \infty.\] 
Thus, by taking \(t \to \infty\) we have \((A_t)_{ij} \to 0\) for all \(i, j \in \{1, \cdots, n\}\) as claimed 
and we have \((\mu_t)\) satisfying \ref{L3}.

\begin{corollary}
  \((\mu_t)\) converges set-wise to some Dirac measure almost everywhere. We denote this 
  limiting (random) Dirac measure by \(\delta_{a_\infty}\) where \(a_\infty\) is some 
  \(\mathbb{R}^n\)-valued random variable.
\end{corollary}

Since convergence implies relatively compact, applying the Dunford-Pettis theorem it follows that 
any realizations of \((F_t)\) is uniformly integrable. Thus, we can make the following deductions 
about \(a_\infty\).

\begin{corollary}\label{cor:lim_dis}
  The massive point \(a_\infty\) of the limiting Dirac measure is the limit of \(a_t\) as 
  \(t \to \infty\) and has law \(\mu\).
\end{corollary}
\begin{proof}
  Since \((F_t)\) is uniformly integrable we have convergence of means almost everywhere, namely
  \[a_t = \int x \mu_t(\dd x) \xrightarrow{\text{a.e.}} \int x \delta_{a_\infty}(\dd x) =: a_\infty \text{ as } t \to \infty\]
  implying that \(a_t\) converges a.e. to \(a_\infty\) as \(t \to \infty\) as required. 
  
  Furthermore, taking \(\phi : \mathbb{R}^n \to \mathbb{R}\) to be any bounded continuous function, we have
  \[\int \phi(x) \mu(\dd x) = \lim_{t \to \infty} \int \phi(x) \mu_t(\dd x).\]
  Then, taking expectation on both sides, we obtain
  \begin{align*}
    \int \phi(x) \mu(\dd x) & = \mathbb{E}\left[\lim_{t \to \infty} \int \phi(x) \mu_t(\dd x)\right] & \\
    & = \lim_{t \to \infty} \mathbb{E}\left[\int \phi(x) \mu_t(\dd x)\right] & (\text{Dominated convergence})\\
    & = \lim_{t \to \infty} \mathbb{E}[\phi(a_t)] & (\text{LOTUS. theorem})\\
    & = \mathbb{E}[\phi(a_\infty)]. & (\text{Dominated convergence \& continuity of } \phi)
  \end{align*}
  Thus, \(\mathbb{E}_\mu[\phi] = \mathbb{E}[\phi(a_\infty)]\) for all bounded continuous \(\phi\) implying 
  \(a_\infty \sim \mu\).
\end{proof}

Finally, \ref{L2} follows by a straightforward computation (take \(\phi = \mathbf{1}_A\) in the following 
proposition).

\begin{proposition}
  For all \(x \in \mathbb{R}^n\), \((F_t(x))_{t \ge 0}\) is a martingale. Furthermore, for any 
  bounded measurable \(\phi : \mathbb{R}^n \to \mathbb{R}\), defining the process 
  \(M_t := \int \phi \dd \mu_t\), \((M_t)_{t \ge 0}\) is also a martingale.
\end{proposition}
\begin{proof}
  By its very construction, \((F_t(x))\) is a martingale by observing equation \ref{eq:stoch_loc} has no 
  drift term.
  
  Now, for all \(s \le t\) we have by the conditional Fubini's theorem,
  \begin{align*}
    \mathbb{E}[M_t \mid \mathscr{F}_s] 
    & = \mathbb{E}\left[\int \phi(x) F_t(x) \mu(\dd x) \middle\vert \mathscr{F}_s\right]\\
    & = \int \phi(x) \mathbb{E}[F_t(x) \mid \mathscr{F}_s] \mu(\dd x) 
    = \int \phi(x) F_s(x) \mu(\dd x)
    = M_s
  \end{align*}
  implying \((M_t)\) is also a martingale.
\end{proof}

Using the same proof as corollary~\ref{cor:lim_dis}, we observe 
\begin{equation}\label{eq:lim_mart}
  M_t \xrightarrow{\text{a.e.}} M_\infty \sim \phi_* \mu
\end{equation}
where \(\phi_*\mu\) denotes the push-forward measure of \(\mu\) along \(\phi\).