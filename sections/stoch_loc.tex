We will in this section provide a description of the stochastic localisation scheme introduced by 
Eldan \cite{Eldan_2013} and describe its application in reducing (up to a logarithmic factor) the 
KLS conjecture into the thin-shell conjecture, i.e. we will describe a proof of \ref{thm:KLS_to_TS}.

% As a broad overview, given a measure, we will construct random sequences of measures for which each realization 
% converges weakly to the some Dirac measure. Furthermore, the mass of these Dirac measures are 
% distributed according to the original measure.

\subsection{Construction and basic properties}\label{sec:construct}

\begin{definition}[Barycenter]
  Given a (probability) measure \(\mu\) on \(\mathbb{R}^n\), we define 
  its barycenter with respect to the function \(F : \mathbb{R}^n \to \mathbb{R}\) to be
  \[\bar{\mu}(F) := \int_{\mathbb{R}^n} x F(x) \mu(\dd x).\]
  In the case that \(F = \text{id}\), we simply write \(\bar{\mu} = \bar{\mu}(F) = \mathbb{E}_{X \sim \mu}[X]\).
\end{definition}

Given the above definition, we now define the following construction central to the stochastic 
localisation scheme. Let \((W_t)_{t \ge 0}\) be a standard Wiener process in \(\mathbb{R}^n\), we define the random functions 
\((F_t)_{t \ge 0}\) to be the solution of the following infinite system of SDEs: 
\begin{equation}\label{eq:stoch_loc}
  F_0 = 1, \dd F_t(x) = \langle x - \bar{\mu}(F_t), \dd W_t \rangle F_t(x),
\end{equation}
for all \(x \in \mathbb{R}^n\). We shall from this point forward denote the random variables 
\(a_t := \bar{\mu}(F_t)\).\todo{Existence and uniqueness of \(F\).}

By applying Itô's formula, we make the following useful observation: for all \(x \in \mathbb{R}^n\),
\begin{equation}
  \dd \log F_t(x) = \frac{\dd F_t(x)}{F_t(x)} - \frac{\dd\hspace{0pt} [F(x)]_t}{2F_t(x)^2} 
    = \langle x - a_t, \dd W_t \rangle - \frac{1}{2}\|x - a_t\|^2 \dd t
\end{equation}
where the second equality follows by the construction of \(F\). Hence, as \(\log F_0(x) = 0\), we 
observe
\begin{align*}
  \log F_t(x) & = \int_0^t \langle x - a_s, \dd W_s \rangle - \frac{1}{2}\int_0^t \|x - a_s\|^2 \dd s\\
    & = \left(\langle x, W_t \rangle - \int_0^t \langle a_s, \dd W_s \rangle\right)
      - \left(\frac{t}{2}\|x\|^2 + \frac{1}{2}\int_0^t\|a_s\|^2 \dd s - \int_0^t \langle x, a_s \rangle \dd s\right)\\
    & = - \left(\int_0^t \langle a_s, \dd W_s \rangle + \frac{1}{2}\|a_s\|^2 \dd s\right) + 
      \langle x, a_t + W_t \rangle - \frac{t}{2}\|x\|^2.
\end{align*}
Thus, taking \(\dd z_t := \langle a_t, \dd W_t\rangle + \frac{1}{2} \|a_t\|^2 \dd t\) and 
\(v_t := a_t + W_t\), we observe \(F_t(x)\) is of the form
\begin{equation}\label{eq:stoch_loc_alt}
  F_t(x) = e^{z_t + \langle x, v_t \rangle - \frac{t}{2}\|x\|^2},
\end{equation}
for given Itô processes \((z_t), (v_t)\).\todo{Is \(e^{z_t + \langle x, v_t \rangle}\) itself a Itô process?}
With this formulation of \(F_t(x)\) in mind, it follows \(F_t\) is non-negative, and so, 
we may define the measure-valued random variable \(\mu_t\) such that \(\mu_t = F_t \mu\), i.e. they have
Radon-Nikodym derivative \(\dd \mu_t / \dd \mu = F_t\).

\begin{proposition}
  For all \(t \ge 0\), \(\mu_t\) is a probability measure almost everywhere (a.e.), i.e. \(\mathbb{P}(\mu_t(\mathbb{R}^n) = 1) = 1\).
\end{proposition}
\begin{proof}
  As \(\mu\) is a probability measure, it suffices to show \(\partial_t \mu_t(\mathbb{R}^n) = 0\).
  To prove this, we first consider the discrete stochastic integral on the lattice \(\Lambda = \mathbb{Z}^n \dd t\)
  for some \(\dd t > 0\). Then, constructing \(\mu_t^{\dd t}\) on \(\Lambda\) via the same process as \(\mu_t\),
  for all \(t = k \dd t \in \Lambda\),
  \begin{align*}
    \partial_t \mu_t^{\dd t}(\mathbb{R}^n) 
      & = \int_{\mathbb{R}^n} \dd F_t(x) \mu(\dd x) 
        = \int_{\mathbb{R}^n} \langle x - a_t, \dd W_t\rangle F_t(x) \mu(\dd x)\\
      & = \left\langle \int_{\mathbb{R}^n} (x - a_t) F_t(x)\mu(\dd x), \dd W_t \right\rangle
        = \langle a_t - a_t\mu_t(\mathbb{R}^n), \dd W_t \rangle = 0
  \end{align*}
  by induction on \(k\). However, by the very construction of the stochastic integral, the densities 
  of \(\mu_t^{\dd t}\): \(F_t^{\dd t}\) converges a.e. to \(F_t\) as \(\dd t \to 0\) implying 
  \(\mu_t^{\dd t} \to \mu_t\) weakly and so, \(\mu_t^{\dd t}(\mathbb{R}^n) \to \mu_t(\mathbb{R}^n)\)
  resulting in \(\mu_t(\mathbb{R}^n) = 1\) as required.
\end{proof}

We would also like to study the limiting behavior of \((\mu_t)\) as \(t \to \infty\). To achieve this, 
we will consider the covariance matrices
\begin{equation}
  A_t := \text{Cov}[\mu_t] = \int (x - a_t) \otimes (x - a_t) \mu_t(\dd x),
\end{equation}
where \(\otimes\) denotes the Kronecker product. In particular, we will show \((A_t)_{ij} \to 0\) for all
 \(i, j \in \{1, \cdots, n\}\) as \(t \to \infty\) allowing us to conclude \((\mu_t)\) converges weakly 
 to some Dirac measure. Indeed, this is a direct consequence of the following lemma.

\begin{lemma}[Brascamp-Lieb inequality, \cite{Brascamp_1976}]\label{lem:brascamp-lieb}
  Given \(V : \mathbb{R}^n \to \mathbb{R}\) convex and \(K > 0\), if \(\nu\) is an isotropic probability 
  measure on \(\mathbb{R}^n\) of the form 
  \[\dd \nu = Ze^{-V(x) - \frac{1}{2K}\|x\|^2}\dd \text{Leb}^n\]
  with \(Z\) being the normalization constant, then \(\nu\) satisfy the Poincaré inequality, i.e. 
  for all differentiable \(\phi\),
  \[K\text{Var}_\nu[\phi] \le \int \|\nabla\phi\|^2 \dd\nu.\]
\end{lemma}

\begin{proof}[Proof of lemma \ref{lem:brascamp-lieb}]
  TODO. Maybe go in appendix?
\end{proof}

With this lemma in mind, by taking \(\nu = \mu_t\) using equation \ref{eq:stoch_loc_alt} and defining
\(\pi_i(x) := x_i\), we have by the Cauchy-Schwarz inequality
\[(A_t)_{ij} \le \sqrt{\text{Var}_{\mu_t}[\pi_i]}\sqrt{\text{Var}_{\mu_t}[\pi_j]} 
  \le \max_{k = 1, \cdots, n} \frac{1}{t}\int \|\nabla \pi_k\|^2 \dd \mu_t\]
Again, using equation~\ref{eq:stoch_loc_alt}, we note that any realizations of \((F_t(x))\) is eventually 
decreasing in \(t\) for all \(x \neq 0\), implying 
\[\sup_{t > 0} \max_{k = 1, \cdots, n} \int \|\nabla \pi_k\|^2 \dd \mu_t = 
\sup_{t > 0} \max_{k = 1, \cdots, n} \int x_k^2 \dd \mu_t < \infty.\] 
Thus, by taking \(t \to \infty\) we have \((A_t)_{ij} \to 0\) for all \(i, j \in \{1, \cdots, n\}\) as claimed 
and we have the following corollary.

\begin{corollary}
  \((\mu_t)\) converges weakly to some Dirac measure almost everywhere. We denote this 
  limiting (random) Dirac measure by \(\delta_{a_\infty}\) where \(a_\infty\) is some 
  \(\mathbb{R}^n\)-valued random variable.
\end{corollary}

Since convergence implies relatively compact, applying the Dunford-Pettis theorem it follows that 
any realizations of \((F_t)\) is uniformly integrable. Thus, we can make the following deductions 
about \(a_\infty\).

\begin{corollary}\label{cor:lim_dis}
  The massive point \(a_\infty\) of the limiting Dirac measure is the limit of \(a_t\) as 
  \(t \to \infty\) and has law \(\mu\).
\end{corollary}
\begin{proof}
  Since \((F_t)\) is uniformly integrable we have convergence of means almost everywhere, namely
  \[a_t = \int x \mu_t(\dd x) \xrightarrow{\text{a.e.}} \int x \delta_{a_\infty}(\dd x) =: a_\infty \text{ as } t \to \infty\]
  implying that \(a_t\) converges a.e. to \(a_\infty\) as \(t \to \infty\) as required. 
  
  Furthermore, taking \(\phi : \mathbb{R}^n \to \mathbb{R}\) to be any bounded continuous function, we have
  \[\int \phi(x) \mu(\dd x) = \lim_{t \to \infty} \int \phi(x) \mu_t(\dd x).\]
  Then, taking expectation on both sides, we obtain
  \begin{align*}
    \int \phi(x) \mu(\dd x) & = \mathbb{E}\left[\lim_{t \to \infty} \int \phi(x) \mu_t(\dd x)\right] & \\
    & = \lim_{t \to \infty} \mathbb{E}\left[\int \phi(x) \mu_t(\dd x)\right] & (\text{Dominated convergence})\\
    & = \lim_{t \to \infty} \mathbb{E}[\phi(a_t)] & (\text{LOTUS. theorem})\\
    & = \mathbb{E}[\phi(a_\infty)]. & (\text{Dominated convergence \& continuity of } \phi)
  \end{align*}
  Thus, \(\mathbb{E}_\mu[\phi] = \mathbb{E}[\phi(a_\infty)]\) for all bounded continuous \(\phi\) implying 
  \(a_\infty \sim \mu\).
\end{proof}

\begin{proposition}
  For all \(x \in \mathbb{R}^n\), \((F_t(x))_{t \ge 0}\) is a martingale. Furthermore, for any 
  continuous \(\phi : \mathbb{R}^n \to \mathbb{R}\), defining the process 
  \(M_t := \int \phi \dd \mu_t\), \((M_t)_{t \ge 0}\) is also a martingale.
\end{proposition}
\begin{proof}
  By its very construction, \((F_t(x))\) is a martingale by observing equation \ref{eq:stoch_loc} has no 
  drift term.
  
  Now, for all \(s \le t\) we have by the conditional Fubini's theorem,
  \begin{align*}
    \mathbb{E}[M_t \mid \mathscr{F}_s] 
    & = \mathbb{E}\left[\int \phi(x) F_t(x) \mu(\dd x) \middle\vert \mathscr{F}_s\right]\\
    & = \int \phi(x) \mathbb{E}[F_t(x) \mid \mathscr{F}_s] \mu(\dd x) 
    = \int \phi(x) F_s(x) \mu(\dd x)
    = M_s
  \end{align*}
  implying \((M_t)\) is also a martingale.
\end{proof}

Using the same proof as corollary~\ref{cor:lim_dis}, we observe 
\begin{equation}\label{eq:lim_mart}
  M_t \xrightarrow{\text{a.e.}} M_\infty \sim \phi_* \mu
\end{equation}
where \(\phi_*\mu\) denotes the push-forward measure of \(\mu\) along \(\phi\).

\subsection{Reduction of KLS to thin-shell}

In this section we will present a proof of theorem~\ref{thm:KLS_to_TS} as described by \cite{Lee_2016} and 
reformulated in terms of concentration in \cite{Eldan_notes}.
We recall the goal of theorem~\ref{thm:KLS_to_TS} is to control \(\text{Var}_\mu[\phi]\) by 
a logarithmic factor of \(\text{Var}_\mu[\|\cdot\|]\). As translating the barycenter of \(\mu\)
does not affect its variance, we may assume \(\mu\) has its barycenter \(\overline{\mu}\) at the origin.

Furthermore, \todo{explain} we may assume \(\mu\) is supported on \(B_n(0) \subseteq \mathbb{R}^n\) with 
\(B_n(0)\) the ball at the origin of radius \(n\). Thus, we also have 
\[\text{supp}\ \mu_t = \text{supp}\ F_t\mu \subseteq \text{supp}\ \mu \subseteq B_n(0)\] 
for all \(t > 0\).
 
Fix \(\phi : \mathbb{R}^n \to \mathbb{R}\) some 1-Lipschitz function and let \((M_t)\) be the 
martingale as described above and in particular we recall equation~\ref{cor:lim_dis} and so, 
\(\text{Var}_\mu[\phi] = \text{Var}[M_\infty]\) where \(M_t \xrightarrow{\text{a.e.}} M_\infty\).
Hence, for all \(t > 0\), by the martingale property we have
\begin{align*}
  \text{Var}[M_t] + \mathbb{E}[\text{Var}[M_\infty \mid \mathscr{F}_t]] 
    & = (\mathbb{E}[M_t^2] - \mathbb{E}[M_t]^2) 
      + \mathbb{E}\left[\mathbb{E}[M_\infty^2 \mid \mathscr{F}_t] - \mathbb{E}[M_\infty \mid \mathscr{F}_t]^2\right]\\
    & = \mathbb{E}[M_t^2] + (\mathbb{E}[\mathbb{E}[M_\infty^2 \mid \mathscr{F}_t]] - \mathbb{E}[M_t^2])\\
    & = \mathbb{E}[M_\infty^2] = \text{Var}[M_\infty],
\end{align*}
where the second equality follows as \(\mathbb{E}[M_t] = \mathbb{E}[M_\infty] = \mathbb{E}_\mu[\phi] = 0\) as 
a linear map 

On the other hand, as \((M_t)\) is a martingale, \(M_t^2 - [M]_t\) is also a martingale implying
\(\mathbb{E}[M_t^2] = \mathbb{E}[M]_t\) and so \(\text{Var}[M_t] = \mathbb{E}[M_t^2] - \mathbb{E}[M_t]^2 
  = \mathbb{E}[M]_t - \overline{\mu}^2 = \mathbb{E}[M]_t\). Hence, combining this with the above, we obtain 
the bound
\begin{equation}\label{eq:bound}
  \text{Var}_\mu[\phi] = \text{Var}[M_\infty] = \text{Var}[M]_t + \mathbb{E}[\text{Var}[M_\infty \mid \mathscr{F}_t]] 
    = \mathbb{E}[M]_t + \mathbb{E}[\text{Var}_{\mu_t}[\phi]].
\end{equation}
Now, observing that \(\phi\) is 1-Lipschitz implies \(\|\nabla \phi\|^2 \le 1\), we have by lemma~\ref{lem:brascamp-lieb} 
the bound \(\text{Var}_{\mu_t}[\phi] \le t^{-1}\) (in fact \(\text{Var}_{\mu_t}[\phi] \le t^{-1} \wedge n^2\) as 
we have assumed \(\text{supp}\ \mu_t \subseteq B_n(0)\)). Thus, the second term \(\mathbb{E}[\text{Var}_{\mu_t}[\phi]]\) 
is bounded by \(t^{-1}\). With this in mind, by choosing an appropriate random time \(\tau\) to stop the process such that 
\(\mathbb{E}[M]_\tau\) is nicely bounded, the result follow by bounding \(\mathbb{E}[\tau^{-1}]\). We dedicate 
the remainder of this section to describe said procedure in detail.

\subsubsection{Differential of the quadratic variation}

To bound the term \(\mathbb{E}[M]_\tau\) we will compute its differential and bound it 
sufficiently such that we reobtain a bound for \([M]_\tau\) after integration. 
We will show \(\dd[] [M]_t\) is bounded by a quantity concerning \(A_t\). This should not be at all 
surprising as both \(\dd[] [M]_t\) and \(A_t\) describes the variation of \(M_t\) in a infinitesimal time 
neighborhood of \(t\).

We compute
\begin{align*}
  \dd M_t & = d \int \phi(x) F_t(x) \mu(\dd x) = \int \phi(x) \langle x - a_t, \dd W_t \rangle \mu_t(\dd x)\\
  & = \left\langle \int \phi(x)(x - a_t)\mu_t(\dd x), \dd W_t\right\rangle
\end{align*}
and so, by considering the component-wise quadratic variation, we have
\begin{equation}\label{eq:diff_qvar}
  \dd[] [M]_t = \left\| \int \phi(x)(x - a_t)\mu_t(\dd x) \right\|^2 \dd t.
\end{equation}
Then, denoting \(\theta\) the vector \(\int \phi(x)(x - a_t)\mu_t(\dd x)\) normalized to have norm 1, so 
\[\left\langle \theta, \int \phi(x)(x - a_t)\mu_t(\dd x)\right\rangle = \left\|\int \phi(x)(x - a_t)\mu_t(\dd x)\right\|\]
we observe,
\begin{equation}\label{eq:red_a}
  \begin{split}
    \dd[] [M]_t & = \left\langle \theta, \int \phi(x)(x - a_t)\mu_t(\dd x)\right\rangle^2 \dd t
      = \left\langle \theta, \int (\phi(x) - a_t)(x - a_t)\mu_t(\dd x)\right\rangle^2 \dd t\\
    & = \left(\int (\phi(x) - a_t) \langle \theta, x - a_t\rangle \mu_t(\dd x)\right)^2 \dd t\\
    & \le \left(\int (\phi(x) - a_t)^2 \mu_t(\dd x)\right) \left(\int \langle \theta, x - a_t\rangle^2 \mu_t(\dd x)\right) \dd t\\
    & = \text{Var}_{\mu_t}[\phi] \left(\int \theta^T (x - a_t)^{\otimes 2} \theta \mu_t(\dd x)\right) \dd t
      = \text{Var}_{\mu_t}[\phi] (\theta^T A_t \theta)\dd t\\ 
    & \le \text{Var}_{\mu_t}[\phi] \|A_t\|_{\text{op}} \dd t.
  \end{split}
\end{equation}
where the inequality follows by the Cauchy-Schwarz inequality and \(\|\cdot\|_{\text{op}}\) denotes the operator norm. 
Thus, as we know \(\text{Var}_{\mu_t}[\phi] \le t^{-1}\), the problem is now reduced to that of
bounding \(\|A_t\|_{\text{op}}\).

\subsubsection{Analysis of the covariance matrix}

As demonstrated in section~\ref{sec:construct}, we know the limiting behavior of the covariance matrices, namely 
\(A_t \to 0\) point-wise as \(t \to \infty\). This was important for us to establish the existence of the limit 
of \((a_t)\) and \((M_t)\). However, as shown above, we now require some quantitative bounds for the operator 
norm of \(A_t\). For this purpose, we first compute some useful properties of \(A_t\).

Observing 
\[\int \dd F_t(x) \mu(\dd x) = \int \langle x - a_t, \dd W_t\rangle \mu_t(\dd x) = 
  \left\langle\int x \mu_t(\dd x) - a_t, \dd W_t\right\rangle = 0,\]
we have
\begin{equation}\label{eq:diff_center}
  \begin{split}
    \dd a_t & = \dd \int x F_t(x) \mu(\dd x) = \int x \dd F_t(x) \mu(\dd x) 
        = \int (x - a_t) \dd F_t(x) \mu(\dd x)\\
      & = \int (x - a_t) \langle x - a_t, \dd W_t \rangle F_t(x) \mu(\dd x) 
        = \int (x - a_t)^{\otimes 2} \dd W_t \mu_t(\dd x) = A_t \dd W_t
  \end{split}
\end{equation}
where the second to last equality used the fact that \(v \langle v, w \rangle = v^{\otimes 2} w\) for 
any appropriate \(v, w\).

Similarly, computing using Itô's formula, we have
\begin{equation}\label{eq:diff_cov_aux}
  \begin{split}
    \dd A_t & = \dd \int (x - a_t)^{\otimes 2} F_t(x) \mu(\dd x)\\
      & = \int (x - a_t)^{\otimes 2} \dd F_t(x) + F_t(x) \dd (x - a_t)^{\otimes 2}\\
      & \hspace{1cm} - 2 (x - a_t) \otimes \dd[][a_t, F_t(x)]_t +F_t(x)\dd[][a_t]_t\mu(\dd x).
  \end{split}
\end{equation}
The second term vanishes as 
\[\int F_t(x) \dd (x - a_t)^{\otimes 2} \mu(\dd x) = -2 \dd a_t \otimes 
  \overbrace{\int (x - a_t) \mu_t(\dd x)}^{= 0} = 0.\]
Also, by equation~\ref{eq:diff_center}, \(\dd a_t = A_t \dd W_t\) implying \(\dd[][a_t]_t = A_t^2 \dd t\).
Finally, as both \((a_t)\) and \((F_t(x))\) are martingales, \(\dd[] [a_t, F_t(x)]_t = F_t(x) A_t x \dd t\) 
and the third term becomes
\begin{align*}
  -2 \int (x - a_t) \otimes \dd[] [a_t, F_t(x)] \mu_t(\dd x) 
  & = - 2A_t \left(\int (x - a_t) \otimes x \mu_t(\dd x)\right) \dd t \\
  & = -2 A_t \left(\overbrace{\int (x - a_t)^{\otimes 2} \mu_t(\dd x)}^{A_t} + 
    \overbrace{\int (x - a_t) \mu_t(\dd x)}^{= 0} \otimes a_t \right) \dd t\\ 
  & = -2 A_t^2 \dd t.
\end{align*}
Hence, combining these and equation~\ref{eq:stoch_loc} together in \ref{eq:diff_cov_aux}, we have
\[\dd A_t = \int (x - a_t)^{\otimes 2} \langle x - a_t, \dd W_t \rangle \mu_t(\dd x) - A_t^2 \dd t\]
However, since we wish to bound \(A_t\) from above, as the drift term \(- A_t^2 \dd t\) only contributes 
negatively, an upper bound for the process of the form 
\(\int (x - a_t)^{\otimes 2} \langle x - a_t, \dd W_t \rangle \mu_t(\dd x)\) is also sufficient for \(A_t\).
Hence, we proceed by ignoring the drift term and redefine the process \(A_t\) such that
\begin{equation}\label{eq:diff_cov}
  \dd A_t = \int (x - a_t)^{\otimes 2} \langle x - a_t, \dd W_t \rangle \mu_t(\dd x).
\end{equation}
With this justification, we now proceed to bound the operator norm of this new \(A_t\). In particular, 
as \(A_t\) is symmetric, we recall that \(\|A_t\|_{\text{op}} = \max_{i= 1, \cdots, n} \lambda_i(t) = \|(\lambda_i(t))_{i = 1}^n\|_\infty\) 
where \(\lambda_i(t)\) denotes the distinct eigenvalues of \(A_t\). Hence, it suffices to find a bound for the potential 
\begin{equation}
  \Phi^{\alpha}(t) = \sum_{i = 1}^n |\lambda_i(t)|^{\alpha} = \|(\lambda_i(t))_{i = 1}^n\|_\alpha^\alpha
\end{equation} 
for some \(\alpha > 0\). Furthermore, as \(A_t\) is positive 
semi-definite, \(\lambda_i(t) \ge 0\) for all \(i = 1, \cdots, n\) and thus we have
\(\Phi^\alpha(t) = \sum_{i = 1}^n \lambda_i(t)^{\alpha}\). Again, to proceed, we will attempt to compute 
\(\dd \Phi^\alpha(t)\) at some \(t = t_0 > 0\) utilizing the following lemma.

\begin{lemma}\label{lem:diff_eig}
  If \(A = [a_{ij}]\) is a diagonal matrix with distinct eigenvalues \(\lambda_i, \cdots, \lambda_n\), then 
  for all \(i, j, k, l, m \in {1, \cdots n}\), we have
  \begin{itemize}
    \item \(\pdv{\lambda_i}{a_{jk}} = \delta_{ij} \delta_{ik}\);
    \item whenever \(i \neq j\), \(\pdv[2]{\lambda_i}{a_{ij}} = 2(\lambda_i - \lambda_j)^{-1}\);
    \item and for \(j \neq l, k \neq m\) or \(i \neq j\) and \(i \neq k\),
      \(\pdv[2]{\lambda_i}{a_{jk}}{a_{lm}} = 0\),
  \end{itemize}
  where \(\delta_{ij}\) denotes the Kronecker delta function.
\end{lemma}

As this lemma requires the matrix to be diagonal, denoting \(e_1, \cdots, e_n\) as the normalized 
eigenbasis of \(A_{t_0}\) (they are in fact orthonormal as \(A_{t_0}\) is positive semi-definite), 
we will consider \(A_t\) with respect to this basis by considering the entries 
\[a_{ij}(t) := \langle e_i, A_t e_j\rangle.\]
Using equation~\ref{eq:diff_cov}, we compute 
\begin{align*}
  \dd a_{ij}(t) & = \left\langle e_i, \left(\int (x - a_t)^{\otimes 2} 
    \langle x - a_t, \dd W_t \rangle \mu_t(\dd x)\right) e_j \right\rangle\\
    & = \left\langle\int \langle e_i, (x - a_t)^{\otimes 2} e_j\rangle
      (x - a_t) \mu_t(\dd x), \dd W_t\right\rangle
      = \langle \xi_{ij}, \dd W_t\rangle
\end{align*}
where we introduce the notation \(\xi_{ij} = \int \langle e_i, (x - a_t)^{\otimes 2} e_j\rangle (x - a_t) \mu_t(\dd x).\)
Thus, combining this with lemma~\ref{lem:diff_eig}, 
denoting \(\lambda_i = \lambda_i(t_0)\), we have by Itô's formula
\begin{equation}
  \begin{split}
    \dd \lambda_i(t) 
    & = \sum_{j, k = 1}^n \pdv{\lambda_i}{a_{jk}} \dd a_{jk}(t) 
      + \frac{1}{2}\sum_{j, k = 1}^n \sum_{l, m = 1}^n \pdv[2]{\lambda_i}{a_{jk}}{a_{lm}} \dd[] [a_{jk}, a_{lm}]_t\\
    & = \langle \xi_{ii}, \dd W_t\rangle + \sum_{j \neq i} \frac{\dd[][a_{ij}]_t}{\lambda_i - \lambda_j}
      = \langle \xi_{ii}, \dd W_t\rangle + \sum_{j \neq i} \frac{\|\xi_{ij}\|^2}{\lambda_i - \lambda_{j}} \dd t.
  \end{split}
\end{equation}
at \(t = t_0\). As a result, it is also clear that \(\dd[][\lambda_i(t)]_{t_0} = \|\xi_{ii}\|^2 \dd t\). 

Again applying Itô's formula, we may finally compute
\begin{align*}
  \dd \Phi^\alpha(t) & = \sum_{i = 1}^n \pdv{\Phi^\alpha}{\lambda_i}\Big\vert_{t = t_0} \dd \lambda_i(t)
    + \frac{1}{2}\sum_{i, j = 1}^n \pdv[2]{\Phi^\alpha}{\lambda_i}{\lambda_j}\Big\vert_{t = t_0} \dd[][\lambda_i, \lambda_j]_{t}\\
  & = \alpha \sum_{i = 1}^n \lambda_i^{\alpha - 1} \dd \lambda_i(t)
    + \frac{1}{2}\alpha(\alpha - 1)\sum_{i = 1}^n \lambda_i^{\alpha - 2} \dd[][\lambda_i(t)]_{t}\\
  & = \alpha \sum_{i = 1}^n \lambda_i^{\alpha - 1} \left(\langle \xi_{ii}, \dd W_t\rangle 
    + \sum_{j \neq i} \frac{\|\xi_{ij}\|^2}{\lambda_i - \lambda_j} \dd t\right) 
    + \frac{1}{2}\alpha(\alpha - 1)\sum_{i = 1}^n \lambda_i^{\alpha - 2} \dd[][\lambda_i(t)]_{t}\\
  & = \alpha \sum_{i \neq j}\lambda_i^{\alpha - 1} \frac{\|\xi_{ij}\|^2}{\lambda_i - \lambda_j} \dd t 
    + \frac{1}{2}\alpha(\alpha - 1)\sum_{i = 1}^n \lambda_i^{\alpha - 2} \|\xi_{ii}\|^2 \dd t
    + \underbrace{\alpha \sum_{i = 1}^n \lambda_i^{\alpha - 1} \langle \xi_{ii}, \dd W_t\rangle}_{=: N_t}\\
  & = \frac{1}{2}\alpha \sum_{i \neq j}\|\xi_{ij}\|^2\frac{\lambda_i^{\alpha - 1} - \lambda_j^{\alpha -1}}{\lambda_i - \lambda_j}\dd t
    + \frac{1}{2}\alpha(\alpha - 1)\sum_{i = 1}^n \lambda_i(t)^{\alpha - 2} \|\xi_{ii}\|^2 \dd t + N_t\\
  & \le \frac{1}{2}\alpha(\alpha - 1) \sum_{i \neq j}\|\xi_{ij}\|^2 (\lambda_i \vee \lambda_j)^{\alpha - 2} \dd t
    + \frac{1}{2}\alpha(\alpha - 1)\sum_{i = 1}^n \lambda_i(t)^{\alpha - 2} \|\xi_{ii}\|^2 \dd t + N_t\\
  & = \frac{1}{2}\alpha(\alpha - 1) \sum_{i, j = 1}^n\|\xi_{ij}\|^2 (\lambda_i \vee \lambda_j)^{\alpha - 2} \dd t
    + N_t
    \le \alpha^2 \sum_{i, j = 1}^n\|\xi_{ij}\|^2 \lambda_i^{\alpha - 2} \dd t + N_t,
\end{align*}
where the first inequality holds as 
\[\frac{\lambda_i^{\alpha - 1} - \lambda_j^{\alpha -1}}{\lambda_i - \lambda_j}
  = \lambda_i^{\alpha - 2} + \lambda_i^{\alpha - 3}\lambda_j + \cdots + \lambda_i^{\alpha - 2} 
  \le (\alpha - 1)(\lambda_i \vee \lambda_j)^{\alpha - 2}.\]
Thus, we have shown 
\begin{equation}
  \dd \Phi^\alpha(t) \le \alpha^2 \sum_{i = 1}^n \lambda_i(t)^{\alpha - 2} \sum_{j = 1}^n \|\xi_{ij}\|^2 \dd t + N_t
\end{equation}
where \((N_t)\) is a martingale.

\subsubsection{Stopping the process early}

As outlined in the beginning of this section, we will stop the process early in order to provide a 
bound for the right hand side of equation~\ref{eq:bound}. By observing equation~\ref{eq:red_a}, we hypothesize that
we should stop the process once \(\|A_t\|_{\text{op}}\) grows too large. As a result we define the stopping time 
\[\tau := \inf\{t > 0 \mid \|A_t\|_{\text{op}} > 2\} \wedge 1.\]
Then, we have by the optional stopping theorem
\begin{align*}
  [M]_\tau & = \int_0^\tau \dd[] [M]_t 
      \le \int_0^\tau \overbrace{\text{Var}_\mu[\phi]}^{\le t^{-1} \wedge n^2} \overbrace{\|A_t\|_{\text{op}}}^{\le 2} \dd t\\
    & \le 2 \int_0^\tau t^{-1} \wedge n^2 \dd t \le 2 \int_0^1 t^{-1} \wedge n^2 \dd t = 2 + 4 \log n.
\end{align*}
Combining this with equation~\ref{eq:bound}, we obtain
\begin{equation}
  \text{Var}_\mu[\phi] \le 2 + 4 \log n + \mathbb{E}[\tau^{-1}],
\end{equation}
as it remains to find an upper bound for \(\mathbb{E}[\tau^{-1}]\).


\subsection{Lee-Vempala's bound}