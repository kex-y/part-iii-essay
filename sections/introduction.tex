The notion of stochastic localization was first introduced in the 2013 paper by Eldan \cite{Eldan_2013}
in order to make progress regarding an isoperimetric problem known as the 
\textit{Kannan-Lovász-Simonovitz} (KLS) conjecture. As it turns out, stochastic localization has been also 
useful in many other adjacent areas, in particular, in sampling and Markov mixing. This essay will 
provide an introduction to stochastic localization and describe its applications in Markov mixing and 
the KLS conjecture. Furthermore, using stochastic localization, this essay will also provide an alternative 
proof of a known bound for the log-Sobolev constant of log-concave measures. 

Stochastic localization in its most general form describes a sequence of random measures (random variables 
taking values in the space of measures) which begins at a given measure and converges to dirac 
measures almost everywhere, namely it ``localizes'', and moreover, satisfy a certain martingale 
condition. These sequences of random measures are useful in studying specific measures. Namely, 
by evolving the stochastic localization in time, particular properties of the structure collapses 
allowing us say something about them. On the other hand, the martingale property allows us to preserve 
these properties (possibly up to some time). Hence, by balancing the two, i.e. allowing the sequence 
to evolve so it is close to a dirac measure while not evolve too long such that we lose the properties 
of the original measure, we are able to obtain results about the original measure.

In our case, we will mostly focus on one specific type of stochastic localization known as the 
linear-tilt localization. The linear-tilt localization is a special case of stochastic localizations 
in which at each time step, the measure is ``tilted'' in a random direction. This random direction 
can be chosen in a variety of ways however one choice of interest is when the random direction is 
chosen according to a Wiener process. A stochastic localization constructed this way accumulates a 
Gaussian component which becomes more and more significant as the process evolves. This is particularly 
helpful as Gaussian measures are well understood and we can use properties of the Gaussian to obtain 
results about our original measure.
\subsection{Structure of this essay}

This essay consists of an introduction to the theory of stochastic localization and subsequently presents three 
of it applications. We will now give a brief overview of these applications.

\subsubsection{Markov mixing bounds}

% We then move on to discuss the application of stochastic localization to bounding Markov 
% mixing time. Motivated by the Markov chain Monte Carlo, the Markov mixing time provides a quantitative 
% measure on how fast a Markov chain converges to its stationary distribution. 

The first application of stochastic localization we will discuss is its application to bounding Markov
mixing times. The motivation for Markov mixing bounds fundamentally comes from sampling. Suppose we wish to sample 
from some probability distribution $\mu$. A common method to achieve this to through the use of the 
Markov chain Monte Carlo (MCMC):
\begin{theorem}[MCMC]\label{thm:markov_conv}
  Given \((X_n)\) an irreducible positively recurrent homogenous Markov process on \(\mathcal{X}\) with
  stationary distribution \(\mu\), for any \(\phi : \mathcal{X} \to \mathbb{R}\) integrable, 
  \[\lim_{n \to \infty} \frac{1}{n} \sum_{k = 1}^n \phi(X_k) = \int \phi \dd \mu\]
  almost everywhere.  
\end{theorem}
With this theorem in mind, MCMC allows us to sample \(\mu\) by sampling from a Markov chain instead. It is in general 
not difficult to come up with such Markov processes, although the difficulty often arises when we want to know its rate of convergence. 
This motivates the notion of mixing bounds which quantifies the time for which the Markov process takes before 
its law is approximately stationary.

\begin{definition}[Total variation mixing time]
  Given a probability measure \(\nu \in \mathcal{M}(\mathcal{X})\), a Markov kernel \(K\) with stationary 
  distribution \(\mu\) and some \(\epsilon > 0\), the \(\epsilon\)-total variation mixing time is defined as 
  \[t_{\text{mix}}(P, \epsilon, \nu) := \inf \{t \ge 0 \mid \|K^t\nu - \mu\|_{\text{TV}} < \epsilon\},\]
  Furthermore, we denote 
  \[t_{\text{mix}}(P, \epsilon) = \sup_{x \in \mathcal{X}} t_{\text{mix}}(P, \epsilon, \delta_x)\]
  the worst mixing time starting at a point.
\end{definition}

A standard method of analyzing the mixing times of Markov chains is through the use of the spectral gap. 

\begin{theorem}[\cite{Levin_2017}]\label{thm:levin}
  Given a reversible and irreducible Markov chain with kernel \(K\) on the state space \(\mathcal{X}\) 
  with stationary distribution \(\mu\), denoting \(\mu_{\min} = \inf_{x \in \mathcal{X}} \mu(x)\), we have 
  \[t_{\text{mix}}(K, \epsilon) \le \left\lceil\frac{1}{\mathfrak{gap}(K)}
  \left(\frac{1}{2}\log\left(\frac{1}{\mu_{\min}}\right) + \log\left(\frac{1}{2\epsilon}\right)\right)\right\rceil,\]
  where \(\mathfrak{gap}(K)\) is the spectral gap of \(K\) and is defined as
  \[\mathfrak{gap}(K) := 1 - \sup\{\lambda \mid \lambda \text{ is an eigenvalue of } K, \lambda \neq 1\}.\]
\end{theorem}

We will see that the spectral gap of Markov chains which kernels can be described using stochastic localizations 
is related to how the variance of the stochastic localization evolves. Thus, by analyzing the variance of 
the stochastic localization, in particular when the stochastic localization ``conserves'' variance, we can 
bound the spectral gap and consequently the mixing time of said kernel. As an example, we will apply this 
method to the Glauber dynamics on the Boolean hypercube to obtain a mixing time bound to the Ising model. 

\subsubsection{The KLS conjecture}

We then move on to discuss a proof of Eldan's original 2013 result in \cite{Eldan_2013} which reduced 
the KLS conjecture to another seemingly weaker conjecture known as the thin-shell conjecture (up to an logarithmic factor). 

\begin{definition}[log-concave measure]
  A measure \(\mu\) on \(\mathbb{R}^n\) is said to log-concave if it is of the form 
  \[\dd \mu = e^{-V}\dd\text{Leb}^n\] 
  for some convex function \(V : \mathbb{R}^n \to \mathbb{R} \cup \{\infty\}\).
\end{definition}

Straightaway, we observe that the standard Gaussian measure on \(\mathbb{R}^n\) is log-concave.
Thus, as the Gaussian measures are very well understood, we are motivated to ask how similarly do log-concave 
measures behave when compared to the Gaussian. The KLS conjecture is one such comparison which compares the 
concentration of log-concave measures with that of the Gaussian.

\begin{definition}[Concentration, \cite{Eldan_notes}]
  Let \(\mu\) be a measure on \(\mathbb{R}^n\), then \(\mu\) is said to be \(C\)-concentrated if
  for all \(1\)-Lipschitz function \(\phi : \mathbb{R}^n \to \mathbb{R}\), 
  \begin{equation}
    \text{Var}_\mu[\phi] = \text{Var}_{X \sim \mu}[\phi(X)] \le \frac{1}{C^2}.
  \end{equation}
  We denote the largest possible such \(C\) by \(C^\mu_{\text{con}}\).
\end{definition}

As it turns out, the standard Gaussian measure on \(\mathbb{R}^n\) is concentrated by a constant which 
is independent of the dimension \(n\). Thus, we might make a conjecture of the form 
``all log-concave probability measures are concentrated by a universal constant''. 
However, as currently stated, this statement is obviously false as spreading out 
the measure decreases the concentration. Furthermore, we need to be careful since, unlike the Gaussian, the 
concentration of general log-concave measures are not invariant under linear functions. Hence, it is clear that the KLS 
conjecture would not hold without a suitable normalization. This leads us to the following.

\begin{definition}[Isotropic]
  A measure \(\mu\) on \(\mathbb{R}^n\) is isotropic if 
  \(\mathbb{E}_{X \sim \nu}[X] = 0\) and \(\text{Cov}_{X \sim \nu}(X) = \text{id}_n\).
\end{definition}

% The KLS conjecture is one such comparisons which roughly asserts 
% that the class of isotropic log-concave measures are concentrated in a similar way to the Gaussian. 
% More precisely, we have the following definitions.
 
\begin{conjecture}[Kannan-Lovász-Simonovitz, \cite{Eldan_notes}]\label{conj:KLS}
  Denoting \(\mathscr{M}^n_{\text{iso}}\) the set of all isotropic log-concave probability measures \(\mu\) on 
  \(\mathbb{R}^n\) (recall that \(\mu\) is isotropic if \(\mathbb{E}_{X \sim \mu}[X] = 0\) and \(\text{Cov}_{X \sim \mu}(X) = \text{id}_n\)),
  there exists a \textit{universal} constant \(C\) 
  (i.e. does not depend on the dimension \(n\)) such that for all 
  \(\mu \in \mathscr{M}^n_{\text{con}}\), \(\mu\) is \(C\)-concentrated.
\end{conjecture}

From a more geometric point of view, the KLS conjecture can be equivalently phrased such that it asserts the 
maximum proportion of volume by surface area of a log-concave measure is bounded by a universal constant. 
As a result, the KLS conjecture has many important consequences in convex geometry. 
In particular, by noticing that any uniform measure on a convex body of unit volume is log-concave, 
the KLS conjecture directly implies the Bourgain slicing conjecture.

\begin{conjecture}[Bourgain slicing, \cite{Bourgain}]
  For any convex body \(U \subset \mathbb{R}^n\) of unit volume, there exists a hyperplane \(S\) such 
  that \(U \cap S\) has boundary measure of at least \(C\) for some universal constant \(C\). 
\end{conjecture}

Moreover, as we shall see, the KLS conjecture in addition has applications to the Poincaré inequality, 
first moment concentration, exponential concentration and the aforementioned thin-shell conjecture. 
We will take a special look at the thin-shell conjecture which relaxes the KLS conjecture by 
only requiring the variance of the norm function to behave similar to the Gaussian.

\begin{conjecture}[Thin-shell, \cite{Eldan_2013}]
  Taking \(\mathscr{M}^n_{\text{iso}}\) as above, there exists a universal constant \(C\) such that 
  for all \(\mu \in \mathscr{M}^n_{\text{con}}\), we have
  \[\sqrt{\text{Var}_\mu[\|\cdot\|]} \le \frac{1}{C}.\]
\end{conjecture}

% Since the norm function is 1-Lipschitz, it is clear that the thin-shell conjecture is 
% weaker than that of the KLS conjecture. On the other hand, as we shall describe in the following subsections, 
% as a consequence of the theory of stochastic localization, it is possible to reduce the KLS conjecture to 
% the thin-shell conjecture up to a logarithmic factor. 

As a consequence of stochastic localization, we present a proof of the following theorem.

\begin{theorem}[Eldan, \cite{Eldan_2013}]\label{thm:KLS_to_TS}
  Denoting \(\mathscr{M}^n_{\text{iso}}\) as above, we define 
  \[C^n_{\text{con}} := \sup \left\{C \mid \text{\(\forall \mu \in \mathscr{M}^n_{\text{con}},\) 
    \(\mu\) is \(C\)-concentrated}\right\},\]
  and 
  \[C^n_{\text{TS}} := \sup \left\{C \mid \text{\(\forall \mu \in \mathscr{M}^n_{\text{con}},\) 
      \(\sqrt{\text{Var}_\mu[\|\cdot\|]} \le C^{-1}\)}\right\},\]
  we have,
  \[C^n_{\text{con}} \le C^n_{\text{TS}} \le C^n_{\text{con}}\log n.\]
\end{theorem}

% The thin-shell conjecture on the other hand relaxes the condition required for concentration and only 
% asserts that the variance of the norm to behave similarly to the Gaussian.
% \begin{conjecture}[Thin-shell, \cite{Eldan_2013}]
%   Taking \(\mathscr{M}^n_{\text{iso}}\) as above, there exists a universal constant \(C\) such that 
%   for all \(\mu \in \mathscr{M}^n_{\text{con}}\), we have
%   \[\sqrt{\text{Var}_\mu[\|\cdot\|]} \le \frac{1}{C}.\]
% \end{conjecture}

\subsubsection{Log-Sobolev inequality}

We will in the last section of this essay discuss the application of stochastic localization to the 
log-Sobolev inequality. Similar to the KLS conjecture, we are interested to compare the log-concave 
measures to the Gaussian measure. In this case, motivated by the Gaussian log-Sobolev inequality, 
we establish a similar inequality for log-concave measures bounding the entropy of said measures. 
Heuristically, similar to that of the variance, the entropy of a random variable is a measure of its 
uncertainty or randomness. 

\begin{definition}[Entropy]
  Given \(\phi : \mathcal{X} \to \mathbb{R}_{\ge 0}\) and a measure \(\mu\), we define the entropy of \(\phi\) 
  with respect to \(\mu\) to be 
  \[\text{Ent}_\mu[\phi] := \mathbb{E}_\mu\left[\phi \log\left(\frac{1}{\mathbb{E}_\mu[\phi]} \phi\right)\right]
   = \int \phi \log \phi \dd \mu - \int \phi \dd \mu \log\left(\int \phi \dd \mu\right)\]
  with the convention that \(0\log 0 = 0\).
\end{definition}

The log-Sobolev inequality is then formulated as the following.
\begin{definition}[Log-Sobolev inequality, \cite{Lee_2016}]
  For a given measure \(\mu\) on \(\mathbb{R}^n\), \(\mu\) is said to satisfy the log-Sobolev inequality with log-Sobolev 
  constant \(\rho_\mu\) if \(\rho_\mu\) is the largest \(\rho\) such that for all smooth 
  \(\phi : \mathbb{R}^n \to \mathbb{R}\) with \(\int \phi^2 \dd \mu = 1\), we have 
  \[\frac{\rho}{2} \text{Ent}_\mu[\phi^2] \le \mathbb{E}_\mu[\|\nabla \phi\|^2] = \int \|\nabla \phi\|^2 \dd \mu.\]
\end{definition}

The log-Sobolev inequality is an incredibly useful inequality while studying the concentration of measures. In 
particular, should a measure satisfy a log-Sobolev equality, by using a well-known method commonly known as Herbst's argument,
one can obtain the exponential concentration of said measure via the Chernoff bound. 
\begin{theorem}[Herbst's argument]
  If \(\mu\) satisfy the log-Sobolev inequality with log-Sobolev constant \(\rho_\mu\), then for all \(\phi\) with 
  uniformly bounded gradient \(\|\nabla \phi\| \le K\), we have 
  \[\psi_{\phi - \mathbb{E}_\mu[\phi]}^\mu(\lambda) \le \frac{\rho_\mu K^2 \lambda^2}{2}\]
  where \(\psi_{\phi - \mathbb{E}_\mu \phi}^\mu\) is the logarithmic moment generating function of 
  \(\phi - \mathbb{E}_\mu[\phi]\) with respect to \(\mu\).
\end{theorem}
% https://faculty.math.illinois.edu/~psdey/math595fa19/lec14.pdf
  
It does not come as a surprise that as with the setting of the KLS conjecture, the standard Gaussian measure 
\(\gamma^n\) in \(\mathbb{R}^n\) satisfies the log-Sobolev inequality with log-Sobolev constant 
\(\rho_{\gamma^n} = 1\) (c.f. \cite{Gross_1975}). However, in contrast to the Poincaré inequality where 
the KLS conjecture suggests that the Poincaré constant is bounded below by a universal constant, it is known 
\cite{Lee_2016} that the log-Sobolev constant cannot be bounded below by a universal constant. Nonetheless, we are 
interested in the log-Sobolev constant of log-concave measure. More specifically, we will study the 
log-Sobolev constant for log-concave measures supported in a ball of fixed diameter to obtain the 
following result.

\begin{theorem}[\cite{Lee_2016}]\label{thm:Lee_Vempala}
  For any isotropic log-concave measure \(\mu\) on \(\mathbb{R}^n\) with support on a ball of diameter \(D\), 
  \(\mu\) has log-Sobolev constant \(\rho_\mu \gtrsim D^{-1}\).
\end{theorem}




% Moreover, the KLS conjecture has applications in bounding mixing times for 
% ball walks from a warm start (c.f. \cite{Lee_2016}), the central limit theorem for convex bodies (c.f. \cite{Giannopoulos}) 
% and directly implies the aforementioned thin-shell conjecture. 



% \begin{itemize}
%   \item The first section of this essay consists of setting up the theory of stochastic localization.
%     We also introduce some specific examples of stochastic localizations which will be used in subsequent 
%     sections most importantly the discrete and continuous linear-tilt localization. 
%   \item We then move on to discuss the application of stochastic localization to bounding Markov 
%     mixing time. Motivated by the Markov chain Monte Carlo, the Markov mixing time provides a quantitative 
%     measure on how fast a Markov chain converges to its stationary distribution. It turns out that 
%     the mixing time of Markov chains which kernels can be described using stochastic localizations 
%     is related to how the variance of the stochastic localization evolves. This method is then applied  
%     to the Glauber dynamics on the Boolean hypercube to obtain a mixing time bound to the Ising model. 
%   \item We also discuss a proof of Eldan's original 2013 result reducing the KLS conjecture to another seemingly 
%     weaker conjecture known as the thin-shell conjecture. Roughly speaking, the KLS conjecture asserts 
%     that the class of measures known as the \textit{isotropic log-concave} measures are concentrated in a similar 
%     way to the Gaussian. In a more geometric point of view, the KLS conjecture asserts that the 
%     maximum proportion of volume by surface area of a log-concave measure is bounded by a universal constant. 
%     As a result, the KLS conjecture has many important consequences in convex geometry. 
%     In particular, by noticing that any uniform measure on a convex body of unit volume is log-concave, 
%     the KLS conjecture also implies the Bourgain slicing conjecture which asserts that any convex body 
%     of unit volume in \(\mathbb{R}^n\) has a intersection with some hyperplane with volume bounded below 
%     by a universal constant. Moreover, the KLS conjecture has applications in bounding mixing times for 
%     ball walks from a warm start (c.f. \cite{Lee_2016}), the central limit theorem for convex bodies (c.f. \cite{Giannopoulos}) 
%     and directly implies the aforementioned thin-shell conjecture. 
  % \item Finally, inspired by the proof from the third section, we will use stochastic localization methods to 
  %   provide a bound for the log-Sobolev constant of log-concave measures. The log-Sobolev constant is 
  %   a constant relating to the log-Sobolev inequality -- a concept central to the theory of concentrations. 
  %   In a similar vain to the KLS conjecture, motivated by the Gaussian log-Sobolev inequality, we also
  %   attempt to show that the log-concave measures behaves similarly to the Gaussian in the setting of 
  %   the log-Sobolev inequality. While in general, this turns out not to be the case, by restricting 
  %   our view to log-concave measures with bounded support, we are able to obtain a bound for the 
  %   log-Sobolev constant. 
    
% \end{itemize}

