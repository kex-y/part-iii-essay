The notion of stochastic localization was first introduced in the 2013 paper by Eldan \cite{Eldan_2013}
in order to make progress regarding an isoperimetric problem known as the 
\textit{Kannan-Lovász-Simonovitz} (KLS) conjecture. As it turns out, stochastic localization has been also 
useful in many other adjacent areas, in particular, in sampling and Markov mixing. This essay will 
provide an introduction to stochastic localization and describe its applications in Markov mixing and 
the KLS conjecture. Furthermore, using stochastic localization, this essay will also provide an alternative 
proof of a known bound for the log-Sobolev constant of log-concave measures. 

Stochastic localization in its most general form describes a sequence of random measures (random variables 
taking values in the space of measures) which begins at a given measure and converges to dirac 
measures almost everywhere, namely it ``localizes'', and moreover, satisfy a certain martingale 
condition. These sequences of random measures are useful in studying specific measures. Namely, 
by evolving the stochastic localization in time, particular properties of the structure collapses 
allowing us say something about them. On the other hand, the martingale property allows us to preserve 
these properties (possibly up to some time). Hence, by balancing the two, i.e. allowing the sequence 
to evolve so it is close to a dirac measure while not evolve too long such that we lose the properties 
of the original measure, we are able to obtain results about the original measure.

In our case, we will mostly focus on one specific type of stochastic localizations known as the 
linear-tilt localization. The linear-tilt localization is a special case of stochastic localizations 
in which at each time step, the measure is ``tilted'' in a random direction. This random direction 
can be chosen in a variety of ways however one choice of interest is when the random direction is 
chosen according to a Wiener process. A stochastic localization constructed this way accumulates a 
Gaussian component which becomes more and more significant as the process evolves. This is particularly 
helpful as Gaussian measures are well understood and we can use properties of the Gaussian to obtain 
results about our original measure.
\subsection{Structure of this essay}

This essay consists of four sections. The first section of this essay consists of setting up the 
theory of stochastic localization while providing some specific examples of stochastic localizations 
which will be used in subsequent sections. On the other hand, the remaining three sections will focus 
on applying this theory in different scenarios. These are 

\begin{itemize}
  \item a framework using stochastic localization to find spectral gaps of specific Markov chains and an 
    example of its application on the Glauber dynamics;
  \item a proof of Eldan's original 2013 result reducing the KLS conjecture to another conjecture known 
    as the thin-shell conjecture;
  \item and finally, inspired by the proof from the third section, we will use stochastic localization methods to 
    provide a bound for the log-Sobolev constant of log-concave measures.
\end{itemize}
