\subsection{Reduction of KLS to thin-shell}

We will now present a proof of theorem~\ref{thm:KLS_to_TS}.
As a high level overview, recall that the linear-tilt localization of a given measure is a measure-valued martingale
for which the original measure is recovered in the limit. Then, as the concentration of the measure 
relates to the covariance of said measure, we will stop the martingale before the covariance grows too large. 
This allows us to analyze the martingale in a more tractable manner. However, as the the sequence is 
a martingale, some properties are invariant in time and hence allowing us to conclude that these properties 
also hold for the original measure.

We recall the goal of theorem~\ref{thm:KLS_to_TS} is to control \(\text{Var}_\mu[\phi]\) by 
a logarithmic factor of \(\text{Var}_\mu[\|\cdot\|]\). As translating the barycenter of \(\mu\)
does not affect its variance, we may assume \(\mu\) has its barycenter \(\overline{\mu}\) at the origin.
Furthermore, \todo{explain} we may assume \(\mu\) is supported on \(B_n(0) \subseteq \mathbb{R}^n\) with 
\(B_n(0)\) the ball at the origin of radius \(n\). Thus, we also have 
\[\text{supp}\ \mu_t = \text{supp}\ F_t\mu \subseteq \text{supp}\ \mu \subseteq B_n(0)\] 
for all \(t > 0\).
 
Fix \(\phi : \mathbb{R}^n \to \mathbb{R}\) some 1-Lipschitz function and let \((M_t)\) be the 
martingale as described above and in particular we recall equation~\eqref{cor:lim_dis} and so, 
\(\text{Var}_\mu[\phi] = \text{Var}[M_\infty]\) where \(M_t \xrightarrow{\text{a.e.}} M_\infty\).
Hence, for all \(t > 0\), by the martingale property we have
\begin{align*}
  \text{Var}[M_t] + \mathbb{E}[\text{Var}[M_\infty \mid \mathscr{F}_t]] 
    & = (\mathbb{E}[M_t^2] - \mathbb{E}[M_t]^2) 
      + \mathbb{E}\left[\mathbb{E}[M_\infty^2 \mid \mathscr{F}_t] - \mathbb{E}[M_\infty \mid \mathscr{F}_t]^2\right]\\
    & = \mathbb{E}[M_t^2] + (\mathbb{E}[\mathbb{E}[M_\infty^2 \mid \mathscr{F}_t]] - \mathbb{E}[M_t^2])\\
    & = \mathbb{E}[M_\infty^2] = \text{Var}[M_\infty],
\end{align*}
where the second equality follows as \(\mathbb{E}[M_t] = \mathbb{E}[M_\infty] = \mathbb{E}_\mu[\phi] = 0\) as 
a linear map 

On the other hand, as \((M_t)\) is a martingale, \(M_t^2 - [M]_t\) is also a martingale implying
\(\mathbb{E}[M_t^2] = \mathbb{E}[M]_t\) and so \(\text{Var}[M_t] = \mathbb{E}[M_t^2] - \mathbb{E}[M_t]^2 
  = \mathbb{E}[M]_t - \overline{\mu}^2 = \mathbb{E}[M]_t\). Hence, combining this with the above, we obtain 
the bound
\begin{equation}\label{eq:bound}
  \text{Var}_\mu[\phi] = \text{Var}[M_\infty] = \text{Var}[M_t] + \mathbb{E}[\text{Var}[M_\infty \mid \mathscr{F}_t]] 
    = \mathbb{E}[M]_t + \mathbb{E}[\text{Var}_{\mu_t}[\phi]].
\end{equation}
Now, observing that \(\phi\) is 1-Lipschitz implies \(\|\nabla \phi\|^2 \le 1\), we have by lemma~\ref{lem:brascamp-lieb} 
the bound \(\text{Var}_{\mu_t}[\phi] \le t^{-1}\) (in fact \(\text{Var}_{\mu_t}[\phi] \le t^{-1} \wedge n^2\) as 
we have assumed \(\text{supp}\ \mu_t \subseteq B_n(0)\)). Thus, the second term \(\mathbb{E}[\text{Var}_{\mu_t}[\phi]]\) 
is bounded by \(t^{-1}\). With this in mind, by choosing an appropriate random time \(\tau\) to stop the process such that 
\(\mathbb{E}[M]_\tau\) is nicely bounded, the result follow by bounding \(\mathbb{E}[\tau^{-1}]\). We dedicate 
the remainder of this section to describe said procedure in detail.

\subsubsection{Differential of the quadratic variation}

To bound the term \(\mathbb{E}[M]_\tau\) we will compute its differential and bound it 
sufficiently such that we reobtain a bound for \([M]_\tau\) after integration. 
We will show \(\dd[] [M]_t\) is bounded by a quantity concerning \(A_t\). This should not be at all 
surprising as both \(\dd[] [M]_t\) and \(A_t\) describes the variation of \(M_t\) in a infinitesimal time 
neighborhood of \(t\).

We compute
\begin{align*}
  \dd M_t & = d \int \phi(x) F_t(x) \mu(\dd x) = \int \phi(x) \langle x - a_t, \dd W_t \rangle \mu_t(\dd x)\\
  & = \left\langle \int \phi(x)(x - a_t)\mu_t(\dd x), \dd W_t\right\rangle
\end{align*}
and so, by considering the component-wise quadratic variation, we have
\begin{equation}\label{eq:diff_qvar}
  \dd[] [M]_t = \left\| \int \phi(x)(x - a_t)\mu_t(\dd x) \right\|^2 \dd t.
\end{equation}
Then, denoting \(\theta\) the vector \(\int \phi(x)(x - a_t)\mu_t(\dd x)\) normalized to have norm 1, so 
\[\left\langle \theta, \int \phi(x)(x - a_t)\mu_t(\dd x)\right\rangle = \left\|\int \phi(x)(x - a_t)\mu_t(\dd x)\right\|\]
we observe,
\begin{equation}\label{eq:red_a}
  \begin{split}
    \dd[] [M]_t & = \left\langle \theta, \int \phi(x)(x - a_t)\mu_t(\dd x)\right\rangle^2 \dd t
      = \left\langle \theta, \int (\phi(x) - a_t)(x - a_t)\mu_t(\dd x)\right\rangle^2 \dd t\\
    & = \left(\int (\phi(x) - a_t) \langle \theta, x - a_t\rangle \mu_t(\dd x)\right)^2 \dd t\\
    & \le \left(\int (\phi(x) - a_t)^2 \mu_t(\dd x)\right) \left(\int \langle \theta, x - a_t\rangle^2 \mu_t(\dd x)\right) \dd t\\
    & = \text{Var}_{\mu_t}[\phi] \left(\int \theta^T (x - a_t)^{\otimes 2} \theta \mu_t(\dd x)\right) \dd t
      = \text{Var}_{\mu_t}[\phi] (\theta^T A_t \theta)\dd t\\ 
    & \le \text{Var}_{\mu_t}[\phi] \|A_t\|_{\text{op}} \dd t.
  \end{split}
\end{equation}
where the inequality follows by the Cauchy-Schwarz inequality and \(\|\cdot\|_{\text{op}}\) denotes the operator norm. 
Thus, as we know \(\text{Var}_{\mu_t}[\phi] \le t^{-1}\), the problem is now reduced to that of
bounding \(\|A_t\|_{\text{op}}\).

\subsubsection{Analysis of the covariance matrix}

As demonstrated in section~\ref{sec:construct}, we know the limiting behavior of the covariance matrices, namely 
\(A_t \to 0\) point-wise as \(t \to \infty\). This was important for us to establish the existence of the limit 
of \((a_t)\) and \((M_t)\). However, as shown above, we now require some quantitative bounds for the operator 
norm of \(A_t\). For this purpose, we first compute some useful properties of \(A_t\).

Observing 
\[\int \dd F_t(x) \mu(\dd x) = \int \langle x - a_t, \dd W_t\rangle \mu_t(\dd x) = 
  \left\langle\int x \mu_t(\dd x) - a_t, \dd W_t\right\rangle = 0,\]
we have
\begin{equation}\label{eq:diff_center}
  \begin{split}
    \dd a_t & = \dd \int x F_t(x) \mu(\dd x) = \int x \dd F_t(x) \mu(\dd x) 
        = \int (x - a_t) \dd F_t(x) \mu(\dd x)\\
      & = \int (x - a_t) \langle x - a_t, \dd W_t \rangle F_t(x) \mu(\dd x) 
        = \int (x - a_t)^{\otimes 2} \dd W_t \mu_t(\dd x) = A_t \dd W_t
  \end{split}
\end{equation}
where the second to last equality used the fact that \(v \langle v, w \rangle = v^{\otimes 2} w\) for 
any appropriate \(v, w\).

Similarly, computing using Itô's formula, we have
\begin{equation}\label{eq:diff_cov_aux}
  \begin{split}
    \dd A_t & = \dd \int (x - a_t)^{\otimes 2} F_t(x) \mu(\dd x)\\
      & = \int (x - a_t)^{\otimes 2} \dd F_t(x) + F_t(x) \dd (x - a_t)^{\otimes 2}\\
      & \hspace{1cm} - 2 (x - a_t) \otimes \dd[][a_t, F_t(x)]_t +F_t(x)\dd[][a_t]_t\mu(\dd x).
  \end{split}
\end{equation}
The second term vanishes as 
\[\int F_t(x) \dd (x - a_t)^{\otimes 2} \mu(\dd x) = -2 \dd a_t \otimes 
  \overbrace{\int (x - a_t) \mu_t(\dd x)}^{= 0} = 0.\]
Also, by equation~\eqref{eq:diff_center}, \(\dd a_t = A_t \dd W_t\) implying \(\dd[][a_t]_t = A_t^2 \dd t\).
Finally, as both \((a_t)\) and \((F_t(x))\) are martingales, \(\dd[] [a_t, F_t(x)]_t = F_t(x) A_t x \dd t\) 
and the third term becomes
\begin{align*}
  -2 \int (x - a_t) \otimes \dd[] [a_t, F_t(x)] \mu_t(\dd x) 
  & = - 2A_t \left(\int (x - a_t) \otimes x \mu_t(\dd x)\right) \dd t \\
  & = -2 A_t \left(\overbrace{\int (x - a_t)^{\otimes 2} \mu_t(\dd x)}^{A_t} + 
    \overbrace{\int (x - a_t) \mu_t(\dd x)}^{= 0} \otimes a_t \right) \dd t\\ 
  & = -2 A_t^2 \dd t.
\end{align*}
Hence, combining these and equation~\eqref{eq:stoch_loc} together in \eqref{eq:diff_cov_aux}, we have
\[\dd A_t = \int (x - a_t)^{\otimes 2} \langle x - a_t, \dd W_t \rangle \mu_t(\dd x) - A_t^2 \dd t\]
However, since we wish to bound \(A_t\) from above, as the drift term \(- A_t^2 \dd t\) only contributes 
negatively, an upper bound for the process of the form 
\(\int (x - a_t)^{\otimes 2} \langle x - a_t, \dd W_t \rangle \mu_t(\dd x)\) is also sufficient for \(A_t\).
Hence, we proceed by ignoring the drift term and redefine the process \(A_t\) such that
\begin{equation}\label{eq:diff_cov}
  \dd A_t = \int (x - a_t)^{\otimes 2} \langle x - a_t, \dd W_t \rangle \mu_t(\dd x).
\end{equation}
With this justification, we now proceed to bound the operator norm of this new \(A_t\). In particular, 
as \(A_t\) is symmetric, we recall that \(\|A_t\|_{\text{op}} = \max_{i= 1, \cdots, n} \lambda_i(t) = \|(\lambda_i(t))_{i = 1}^n\|_\infty\) 
where \(\lambda_i(t)\) denotes the distinct eigenvalues of \(A_t\). Hence, it suffices to find a bound for the potential 
\begin{equation}
  \Phi^{\alpha}(t) = \sum_{i = 1}^n |\lambda_i(t)|^{\alpha} = \|(\lambda_i(t))_{i = 1}^n\|_\alpha^\alpha
\end{equation} 
for some \(\alpha > 0\). Furthermore, as \(A_t\) is positive 
semi-definite, \(\lambda_i(t) \ge 0\) for all \(i = 1, \cdots, n\) and thus we have
\(\Phi^\alpha(t) = \sum_{i = 1}^n \lambda_i(t)^{\alpha}\). Again, to proceed, we will attempt to compute 
\(\dd \Phi^\alpha(t)\) at some \(t = t_0 > 0\) utilizing the following lemma.

\begin{lemma}\label{lem:diff_eig}
  If \(A = [a_{ij}]\) is a diagonal matrix with distinct eigenvalues \(\lambda_i, \cdots, \lambda_n\), then 
  for all \(i, j, k, l, m \in {1, \cdots n}\), we have
  \begin{itemize}
    \item \(\pdv{\lambda_i}{a_{jk}} = \delta_{ij} \delta_{ik}\);
    \item whenever \(i \neq j\), \(\pdv[2]{\lambda_i}{a_{ij}} = 2(\lambda_i - \lambda_j)^{-1}\);
    \item and for \(j \neq l, k \neq m\) or \(i \neq j\) and \(i \neq k\),
      \(\pdv[2]{\lambda_i}{a_{jk}}{a_{lm}} = 0\),
  \end{itemize}
  where \(\delta_{ij}\) denotes the Kronecker delta function.
\end{lemma}

As this lemma requires the matrix to be diagonal, denoting \(e_1, \cdots, e_n\) as the normalized 
eigenbasis of \(A_{t_0}\) (they are in fact orthonormal as \(A_{t_0}\) is positive semi-definite), 
we will consider \(A_t\) with respect to this basis by considering the entries 
\[a_{ij}(t) := \langle e_i, A_t e_j\rangle.\]
Using equation~\eqref{eq:diff_cov}, we compute 
\begin{align*}
  \dd a_{ij}(t) & = \left\langle e_i, \left(\int (x - a_t)^{\otimes 2} 
    \langle x - a_t, \dd W_t \rangle \mu_t(\dd x)\right) e_j \right\rangle\\
    & = \left\langle\int \langle e_i, (x - a_t)^{\otimes 2} e_j\rangle
      (x - a_t) \mu_t(\dd x), \dd W_t\right\rangle
      = \langle \xi_{ij}, \dd W_t\rangle
\end{align*}
where we introduce the notation \(\xi_{ij} = \int \langle e_i, (x - a_t)^{\otimes 2} e_j\rangle (x - a_t) \mu_t(\dd x).\)
Thus, combining this with lemma~\ref{lem:diff_eig}, 
denoting \(\lambda_i = \lambda_i(t_0)\), we have by Itô's formula
\begin{equation}
  \begin{split}
    \dd \lambda_i(t) 
    & = \sum_{j, k = 1}^n \pdv{\lambda_i}{a_{jk}} \dd a_{jk}(t) 
      + \frac{1}{2}\sum_{j, k = 1}^n \sum_{l, m = 1}^n \pdv[2]{\lambda_i}{a_{jk}}{a_{lm}} \dd[] [a_{jk}, a_{lm}]_t\\
    & = \langle \xi_{ii}, \dd W_t\rangle + \sum_{j \neq i} \frac{\dd[][a_{ij}]_t}{\lambda_i - \lambda_j}
      = \langle \xi_{ii}, \dd W_t\rangle + \sum_{j \neq i} \frac{\|\xi_{ij}\|^2}{\lambda_i - \lambda_{j}} \dd t.
  \end{split}
\end{equation}
at \(t = t_0\). As a result, it is also clear that \(\dd[][\lambda_i(t)]_{t_0} = \|\xi_{ii}\|^2 \dd t\). 

Again applying Itô's formula, we may finally compute\todo{Maybe move computation to appendix}
\begin{align*}
  \dd \Phi^\alpha(t) & = \sum_{i = 1}^n \pdv{\Phi^\alpha}{\lambda_i}\Big\vert_{t = t_0} \dd \lambda_i(t)
    + \frac{1}{2}\sum_{i, j = 1}^n \pdv[2]{\Phi^\alpha}{\lambda_i}{\lambda_j}\Big\vert_{t = t_0} \dd[][\lambda_i, \lambda_j]_{t}\\
  & = \alpha \sum_{i = 1}^n \lambda_i^{\alpha - 1} \dd \lambda_i(t)
    + \frac{1}{2}\alpha(\alpha - 1)\sum_{i = 1}^n \lambda_i^{\alpha - 2} \dd[][\lambda_i(t)]_{t}\\
  & = \alpha \sum_{i = 1}^n \lambda_i^{\alpha - 1} \left(\langle \xi_{ii}, \dd W_t\rangle 
    + \sum_{j \neq i} \frac{\|\xi_{ij}\|^2}{\lambda_i - \lambda_j} \dd t\right) 
    + \frac{1}{2}\alpha(\alpha - 1)\sum_{i = 1}^n \lambda_i^{\alpha - 2} \dd[][\lambda_i(t)]_{t}\\
  & = \alpha \sum_{i \neq j}\lambda_i^{\alpha - 1} \frac{\|\xi_{ij}\|^2}{\lambda_i - \lambda_j} \dd t 
    + \frac{1}{2}\alpha(\alpha - 1)\sum_{i = 1}^n \lambda_i^{\alpha - 2} \|\xi_{ii}\|^2 \dd t
    + \left\langle\underbrace{\alpha \sum_{i = 1}^n \lambda_i^{\alpha - 1} \xi_{ii}}_{=: v_t}, \dd W_t\right\rangle\\
  & = \frac{1}{2}\alpha \sum_{i \neq j}\|\xi_{ij}\|^2\frac{\lambda_i^{\alpha - 1} - \lambda_j^{\alpha -1}}{\lambda_i - \lambda_j}\dd t
    + \frac{1}{2}\alpha(\alpha - 1)\sum_{i = 1}^n \lambda_i(t)^{\alpha - 2} \|\xi_{ii}\|^2 \dd t + \langle v_t, \dd W_t\rangle\\
  & \le \frac{1}{2}\alpha(\alpha - 1) \sum_{i \neq j}\|\xi_{ij}\|^2 (\lambda_i \vee \lambda_j)^{\alpha - 2} \dd t
    + \frac{1}{2}\alpha(\alpha - 1)\sum_{i = 1}^n \lambda_i(t)^{\alpha - 2} \|\xi_{ii}\|^2 \dd t + \langle v_t, \dd W_t\rangle\\
  & = \frac{1}{2}\alpha(\alpha - 1) \sum_{i, j = 1}^n\|\xi_{ij}\|^2 (\lambda_i \vee \lambda_j)^{\alpha - 2} \dd t
    + \langle v_t, \dd W_t\rangle
    \le \alpha^2 \sum_{i, j = 1}^n\|\xi_{ij}\|^2 \lambda_i^{\alpha - 2} \dd t + \langle v_t, \dd W_t\rangle,
\end{align*}
where the first inequality holds as 
\[\frac{\lambda_i^{\alpha - 1} - \lambda_j^{\alpha -1}}{\lambda_i - \lambda_j}
  = \lambda_i^{\alpha - 2} + \lambda_i^{\alpha - 3}\lambda_j + \cdots + \lambda_i^{\alpha - 2} 
  \le (\alpha - 1)(\lambda_i \vee \lambda_j)^{\alpha - 2}.\]
Thus, we have shown 
\begin{equation}\label{eq:potential_bound}
  \dd \Phi^\alpha(t) \le \alpha^2 \sum_{i = 1}^n \lambda_i(t)^{\alpha - 2} \sum_{j = 1}^n \|\xi_{ij}\|^2 \dd t + \langle v_t, \dd W_t\rangle
\end{equation}
where \(v_t := \alpha \sum_{i = 1}^n \lambda_i^{\alpha - 1}\xi_{ii}\).

By recalling that our goal is to bound \(\|A_t\|_{\text{op}}\) from above (c.f. equation~\eqref{eq:bound} and \eqref{eq:red_a}), 
we may assume without loss of generality that \(\|A_t\|_{\text{op}} \ge 1\). Thus, applying the reverse Cauchy-Schwarz inequality to 
equation~\eqref{eq:potential_bound}, we have
\begin{align*}
  \dd \Phi^\alpha(t) & \le 2\alpha^2 \sum_{i = 1}^n \lambda_i(t)^{\alpha - 2} \sum_{j = 1}^n \|\xi_{ij}\|^2 \dd t + \langle v_t, \dd W_t\rangle\\
    & \le 2\alpha^2 \|A_t\|_{\text{op}}^2 \sum_{i = 1}^n \lambda_i(t)^{\alpha - 2} \sum_{j = 1}^n \|\xi_{ij}\|^2 \dd t + \langle v_t, \dd W_t\rangle\\
    & \lesssim 2\alpha^2\sum_{i = 1}^n \lambda_i(t)^\alpha \sum_{j = 1}^n \|\xi_{ij}\|^2 \dd t + \langle v_t, \dd W_t\rangle.
\end{align*} 
Thus, defining \(K_t := \sup_i \sum_{j = 1}^n \|\xi_{ij}\|^2\), we have the bound 
\begin{equation}\label{eq:potential_bound_2}
  \dd \Phi^\alpha(t) \lesssim 2\alpha^2 K_t \Phi^\alpha(t) \dd t + \langle v_t, \dd W_t\rangle.
\end{equation}

\subsubsection{Stopping the process early}

As outlined in the beginning of this section, we will stop the process early in order to provide a 
bound for the right hand side of equation~\eqref{eq:bound}. By observing equation~\eqref{eq:red_a}, we hypothesize that
we should stop the process once \(\|A_t\|_{\text{op}}\) grows too large. As a result we define the stopping time 
\[\tau := \inf\{t > 0 \mid \|A_t\|_{\text{op}} > 2\} \wedge 1.\]
By the optional stopping theorem we have
\begin{align*}
  [M]_\tau & = \int_0^\tau \dd[] [M]_t 
      \le \int_0^\tau \overbrace{\text{Var}_\mu[\phi]}^{\le t^{-1} \wedge n^2} \overbrace{\|A_t\|_{\text{op}}}^{\le 2} \dd t\\
    & \le 2 \int_0^\tau t^{-1} \wedge n^2 \dd t \le 2 \int_0^1 t^{-1} \wedge n^2 \dd t = 2 + 4 \log n.
\end{align*}
Combining this with equation~\eqref{eq:bound}, we obtain
\begin{equation}\label{eq:tau_bd}
  \text{Var}_\mu[\phi] \le 2 + 4 \log n + \mathbb{E}[\tau^{-1}],
\end{equation}
and it remains to find an upper bound for \(\mathbb{E}[\tau^{-1}]\). Observing that \(t < \tau\) whenever 
\(\Phi^\alpha(t) < 2^\alpha\), we define the \(\sigma\) the first time for which the potential \(\Phi^\alpha(t)\) reaches \(2^\alpha\),
namely
\[\sigma := \inf \{t > 0 \mid \Phi^\alpha(t) = 2^\alpha\},\] 
we have \(\sigma^{-1} \ge \tau^{-1}\) and so it suffices to bound \(\sigma\) from below. 

For simplicity\todo{Maybe not ignore the martingale term if we want a complete proof}, let us ignore the stochastic term in 
equation~\eqref{eq:potential_bound_2} and regard it as an ODE. Then, by Gronwall's inequality, if we can 
find some constant \(K\) such that \(K_t \le K\) for all \(t \le \tau\), we have the bound
\[S_t \le n e^{2\alpha^2 K t}.\]
Thus, substituting \(\sigma\) into the above, we have 
\[2^\alpha = S_\sigma \le ne^{2\alpha^2 K\sigma}\]
implying 
\[\frac{\alpha \log 2 - \log n}{2\alpha^2 K} \le \sigma \le \tau.\]
Then, taking \(\alpha = 10K\log n\), it is easy to check that 
\[\frac{1}{10K \log n} \le \frac{\alpha \log 2 - \log n}{2\alpha^2 K}\]
implying \(\mathbb{E}[\tau^{-1}] \le 10K \log n\). Of course, this deduction only holds while ignoring the stochastic term 
\(\langle v_t, \dd W_t\rangle\). Nonetheless, this is justified as one can show that \(\|v_t\|_2\) is bounded 
\(\alpha \Phi^\alpha(t)\) and so the same analysis holds by applying the stochastic Gronwall's inequality
(c.f. second part of lemma 34 in \cite{Lee_2018}).

Finally, to find a bound for \((K_t)\), we employ the following lemma.

\begin{lemma}[Lemma 1.6 in \cite{Eldan_2013}]\label{lem:final_bd}
  Denoting \(C_{\text{TS}}^n\) as in theorem~\ref{thm:KLS_to_TS}, there exists a constant \(C\) such that 
  for any log-concave, isotropic probability measure \(\mu\), we have
  \[\sup_{\theta \in S^{n - 1}}\sum_{i, j = 1}^n 
    \mathbb{E}_{X \sim \mu}[\langle X, e_i\rangle \langle X, e_j\rangle \langle X, \theta\rangle]^2 \le 
    C \sum_{k = 1}^n \frac{(C_{\text{TS}}^n)^2}{k},\]
  where \(\{e_1, \cdots, e_n\}\) is any orthonormal basis on \(\mathbb{R}^n\).
\end{lemma}

Recalling that
\[\xi_{ij} = \mathbb{E}_{X + a_t \sim \mu_t}[\langle e_i, X^{\otimes 2} e_j\rangle X] = 
  \mathbb{E}_{X + a_t \sim \mu_t}[\langle X, e_i\rangle \langle X, e_j\rangle X],\]
we have by Parseval's identity 
\begin{align*}
  K_t & = \sup_i \sum_{j = 1}^n \|\xi_{ij}\|^2 
    = \sup_i \sum_{j = 1}^n \|\mathbb{E}_{X + a_t \sim \mu_t}[\langle X, e_i\rangle \langle X, e_j\rangle X]\|^2\\
  & = \sup_i \sum_{j = 1}^n \sum_{k = 1}^n 
    \left\langle \mathbb{E}_{X + a_t \sim \mu_t}[\langle X, e_i\rangle \langle X, e_j\rangle X], e_k\right\rangle^2\\
  & = \sup_i \sum_{j, k = 1}^n \mathbb{E}_{X + a_t \sim \mu_t}[\langle X, e_i\rangle \langle X, e_j\rangle \langle X, e_k\rangle]^2\\
  & \le \sup_{\theta \in S^{n - 1}}\sum_{i, j = 1}^n 
    \mathbb{E}_{X + a_t \sim \mu}[\langle X, e_i\rangle \langle X, e_j\rangle \langle X, \theta\rangle]^2.
\end{align*}
We note that we cannot direct apply lemma~\ref{lem:final_bd} at this point since the measure \(\mu_t\) 
might not be isotropic. Hence, to be able to use the lemma, we need to normalize the covariance of \(\mu_t\). 
Namely, taking \(X + a_t \sim \mu_t\), we define \(Y = A^{-1/2} X\) which by construction is isotropic. 
Thus, by observing that 
\[\mathbb{E}_{X + a_t \sim \mu}[\langle X, e_i\rangle \langle X, e_j\rangle \langle X, \theta\rangle]^2 
  \le \|A_t\|_{\text{op}}^3 \mathbb{E}_{X + a_t \sim \mu}[\langle Y, e_i\rangle \langle Y, e_j\rangle \langle Y, \theta\rangle]^2,\]
we have 
\begin{equation}\label{eq:K_bd}
  \begin{split}
    K_t & \le \sup_{\theta \in S^{n - 1}}\sum_{i, j = 1}^n 
        \mathbb{E}_{X + a_t \sim \mu}[\langle X, e_i\rangle \langle X, e_j\rangle \langle X, \theta\rangle]^2\\
      & \le \|A_t\|_{\text{op}}^3 \sup_{\theta \in S^{n - 1}}\sum_{i, j = 1}^n 
        \mathbb{E}_{X + a_t \sim \mu}[\langle Y, e_i\rangle \langle Y, e_j\rangle \langle Y, \theta\rangle]^2
        \le 8 C \sum_{k = 1}^n \frac{(C_{\text{TS}}^n)^2}{k}
      \end{split}
\end{equation}
where the last inequality follows as \(\|A_t\|_{\text{op}} \le 2\) for all \(t < \tau\). 
    
At last, combining equation~\eqref{eq:K_bd} and~\eqref{eq:tau_bd}, we have
\[\text{Var}_\mu[\phi] \le 2 + \log n\left(4 + \overbrace{80 C \sum_{k = 1}^n \frac{1}{k}}^{\Theta(\log n)}(C_{\text{TS}}^n)^2\right)
   = \Theta_n((C_{\text{TS}}^n \log n)^2)\]
implying there exists a constant \(R > 0\) such that for all 1-Lipschitz \(\phi\), 
\(\sqrt{\text{Var}_\mu[\phi]} \le R C_{\text{TS}}^n \log n\), i.e. \(\mu\) is 
\(R C_{\text{TS}}^n \log n\)-concentrated and so, \(C_{\text{con}}^n \le R C_{\text{TS}}^n \log n\)
as required.

% \subsection{Lee-Vempala's bound}

% \begin{theorem}[Lee-Vempala, \cite{Lee_2016}]
  
% \end{theorem}

% By taking \(\alpha = 2\) in equation~\eqref{eq:potential_bound}, we have 
% \[\dd \Phi^2(t) \le 4 \sum_{i, j = 1}^n \|\xi_{ij}\|^2 + \langle v_t, \dd W_t\rangle,\]
% where \(v_t = 2 \sum_{i = 1}^n \lambda_i \xi_{ii}\).