Sampling from a given distribution is a fundamental problem arising naturally in many fields 
such as bandit theory, machine learning, Bayesian statistics, etc. Among these problems, sampling from the class of 
distributions known as the log-concave distributions are of particular interest.
These distributions characterizes the uniform distributions on convex bodies and is related to the volumes
of said bodies. Beyond this, sampling from the log-concave measures is useful for optimization, e.g. 
minimizing convex functions. 

Sampling from these distributions has been studied for a long time and there are many algorithms 
devised to do so. 





\subsection{Structure of this essay}

We will in this essay examine three related applications of the stochastic localization scheme. In particular,
we will consider its application to providing Markov mixing bounds, the relation of the KLS and thin-shell 
conjectures, and its application to proving a log-Sobolev inequality. 

\textit{Markov mixing bounds}: 

\textit{KLS and thin-shell conjectures}:

\textit{Log-Sobolev inequality}: The log-Sobolev inequality is a class of inequalities fundamental in the 
theory of concentration of measures. 